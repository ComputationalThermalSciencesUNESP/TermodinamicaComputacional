\chapter{Introdução aos Ciclos Termodinâmicos}
\label{chap:thermodynamicCycles}

    Verificamos em capítulos anteriores que poderíamos, desde que não
    violássemos a primeira e a segunda lei da termodinâmica, obter trabalho (ou
    minimizar o trabalho) em um processo cíclico trocando calor com uma fonte
    quente e uma fonte fria, daí o conceito de máquina térmica e refrigerador.
    Aprendemos também que o ciclo de Carnot, composto de processos
    completamente reversíveis, é o que apresenta o maior rendimento entre todos
    os ciclos que operam entre as mesmas temperaturas. Então, nada mais natural
    que procuremos emular o melhor que pudermos o ciclo de Carnot.
    Descobriremos sem muito esforço que, para um rendimento razoável do nosso
    ciclo, precisaremos lançar mão de quatro processos básicos, sendo que dois
    deles operariam trocando calor à temperatura constante alta e baixa do
    ciclo respectivamente e os outros dois processos fariam a conexão que
    fecharia o ciclo.

    \section{Ciclo de Rankine}

    Quando se observa o comportamento das substâncias puras, chama-nos a
    atenção a região de saturação na qual a temperatura permanece constante
    enquanto a pressão não se altera. Então, na região de saturação, se
    evaporarmos, de um lado, e condensarmos, do outro, uma substância pura e
    conectarmos, pelos processos mais reversíveis que pudermos obter, o lado de
    baixa pressão com o lado de alta pressão, poderíamos ter um ciclo realizado
    na prática e com eficiência razoável.  Quando a evaporação no evaporador
    (ou gerador de vapor) se dá no lado de alta pressão; a queda da pressão de
    alta para a baixa é transformada em trabalho mecânico, dentro de uma
    turbina adiabática; a condensação acontece no lado de baixa pressão, no
    condensador; e o ciclo é fechado por meio de uma bomba adiabática que repõe
    o fluido de trabalho na pressão de alta, temos o ciclo de Rankine.

    \section{Ciclo de Refrigeração por Compressão de Vapor}

    Se o evaporador troca calor no lado de baixa pressão; o fluido de trabalho
    gasoso é comprimido para a pressão de alta no compressor adiabático; o
    condensador troca calor na pressão de alta e, finalmente, o fluido líquido
    é estrangulado para a pressão de baixa através de uma válvula, então temos
    o ciclo de refrigeração por compressão de vapor.  Trata-se de um
    refrigerador, se o espaço de alta pressão (e, portanto, temperatura) for o
    meio-ambiente, este considerado um reservatório térmico em T0, e de uma
    bomba de calor se o meio-ambiente for o espaço de temperatura mais baixa.

    \section{Ciclos Padrão de Ar de Otto, Diesel e Stirling}

    Se considerarmos agora um gás perfeito, veremos que podemos construir um
    ciclo de Carnot em um sistema fechado composto por duas isotérmicas a
    volume constante e duas expansões adiabáticas. Este é o ciclo de Otto, que
    simula o comportamento de um motor de combustão interna de ignição por
    centelha.  Se substituirmos o processo de calor isovolumétrico que
    substitui a combustão, por uma troca de calor à pressão constante, obtemos
    o ciclo de Diesel, que simula um motor de combustão interna de ignição por
    compressão.  Se tivermos transferência de calor também durante os dois
    processos a volume constante, além do calor trocado nos dois processos a
    temperatura constante, obteremos o ciclo de Stirling.

    \section{Ciclo Padrão de Ar de Brayton de Turbina a Gás}

    Consideremos agora um ciclo do gás perfeito que ocorra através de quatro
    processos em ciclo aberto e regime permanente. Então, haverá uma compressão
    adiabática, uma troca de calor para o fluido à pressão constante, uma
    expansão adiabática e uma rejeição de calor do fluido de trabalho para o
    ambiente. Com esse procedimento, criamos o ciclo de Brayton, que simula
    motores a jato, turbo-hélice e turbinas a gás.

    \section{Ciclos Mistos}

    Existem muitas situações em que é desejável combinarmos dois ciclos, tanto
    os de potência quanto  os de refrigeração, em série, tal que o evaporador
    de um é parte do condensador do outro e assim por diante, nos chamados
    ciclos combinados.
