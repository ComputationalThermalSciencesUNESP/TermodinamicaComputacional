\chapter*{A Abertura}
\addcontentsline{toc}{chapter}{A Abertura}


    Os textos tradicionais de Termodinâmica geralmente contém tabelas de uma ou
    dupla entrada que representam os valores das principais propriedades
    termodinâmicas para um número selecionado de substâncias puras, usualmente
    com valores escalonados da sua temperatura e/ou pressão. Mediante uma, duas
    ou mais interpolações lineares, o aluno poderá então obter o valor das
    propriedades sob valores intermediários. De fato, essa abordagem abriga
    alguns graves problemas. Em primeiro lugar, o aluno ficará com uma sensação
    de que as tabelas foram entregues aos humanos por alguma divindade, quando
    na verdade elas provêm de equações cujos valores foram calculados em um
    passado recente por meio de \enquote{grandes} computadores. Deve ser
    recordado aqui que ainda nas décadas de 80-90, o computador
    \enquote{pessoal} era um objeto raro e caro. Em segundo lugar, o aluno será
    praticamente incapaz de resolver problemas cujo modelagem vá além de um
    sistema com uma equação --- uma incógnita (ou eventualmente duas equações -
    duas incógnitas), o que não lhe permite vislumbrar o alcance e as
    potencialidades oferecidas a ele pelo Postulado dos Estados e pelo conceito
    de propriedade termodinâmica de substância pura como função
    \enquote{conhecida} de apenas duas propriedades termodinâmicas
    independentes (no caso de sistemas compressíveis simples).  Em terceiro
    lugar, para o aluno lidar com uma substância pura e suas propriedades já
    parece complicado, como poderá este investigar problemas mais complexos que
    contenham misturas reativas e não reativas, com ou sem equilíbrio de fases,
    ideais ou modeladas como pseudo-substâncias puras? Em quarto lugar, e de
    certa forma como um corolário de todas as razões anteriores, ao aluno
    ficarão vedados os problemas que envolvem otimização e simulação de
    sistemas termodinâmicos, ou seja, exatamente aqueles que são mais
    interessantes e com os quais ele certamente irá se defrontar na sua vida
    real!

    Sabe-se que hoje em dia os computadores portáteis possuem muito mais poder
    computacional (sob qualquer métrica!) dos que os mainframes de outrora e
    que, além disso, são acessíveis a todo mundo, de modo que as tabelas
    termodinâmicas tradicionais deveriam seguir a mesma rota das tabelas
    trigonométricas e logarítmicas --- a extinção. Para contribuir com esse
    processo, no presente texto nós pretendemos resgatar um certo
    fundamentalismo, pelo qual as propriedades termodinâmicas serão
    determinadas a partir de uma equação de estado generalizada, utilizando-se
    extensivamente os conceitos de fator de compressibilidade, coeficiente de
    fugacidade e propriedades residuais. Além disso, modelos pseudo-críticos
    apropriados serão utilizados no caso de misturas homogêneas não ideais.

    Dentre todas as equações de estado generalizadas disponíveis na literatura,
    optamos pela implementação da equação de estado de Lee-Kesler com a
    correção dada pelo fator acêntrico \gls{PitzerAcentricFactor} de Pitzer e
    pelo fator \gls{WuPolarFactor} de Wu-Stiel, quando disponível, para uma
    correção adicional que se aplica a moléculas polares, como a água. Obtendo
    erros relativos da ordem de no máximo \SI{5}{\percent} quando comparados
    com as tabelas, não estamos tão preocupados, no momento, com uma extrema
    acuracidade dos valores obtidos, mas com o desenvolvimento de um
    procedimento geral e rigoroso para o tratamento de sistemas termodinâmicos
    com o uso de computador.

    Por outro lado, escolhemos para a implementação a linguagem Python, por ser
    dinâmica, interpretada, de fácil sintaxe e aprendizado, muito alto nível,
    orientada a objetos, multiplataforma e de fonte aberta. Com as bibliotecas
    apropriadas, Scientific Python é cada vez mais empregado no meio
    científico, seja em substituição ao proprietário e dispendioso MATLAB, seja
    como linguagem de ligação para as linguagens compiladas, como Fortran, C e
    C++.  De fato, pode-se afirmar que, segundo a métrica do conhecido TIOBE,
    Python corresponde à quinta ou sexta linguagem de programação mais
    utilizada no mundo.

    Este texto é fruto de uma concepção cujo objetivo foi fornecer ao leitor
    cujo tempo é limitado e precioso, um substrato rigoroso, embora sucinto às
    vezes, omisso em partes, sobre o qual ele ou ela poderá, em primeiro lugar,
    modelar e estruturar as possíveis soluções para problemas de Termodinâmica
    Clássica de complexidade acima do trivial e, em segundo lugar, construir um
    caminho para o aprofundamento em qualquer direção que mais lhe interessar
    ou necessitar, sempre utilizando o computador como a ferramenta principal.

    Isto posto, dividimos a obra em dez capítulos e uma Introdução. No Capítulo
    de Introdução (\cref{chap:introduction}), a partir dos conceitos mais
    fundamentais de sistema, vizinhança, fronteira, propriedades e interações
    de energia e massa com a vizinhança na fronteira: trabalho, calor e vazão
    de massa, definimos a Termodinâmica de forma a satisfazer os nossos
    propósitos. No \cref{chap:pureSubstances}, apresentamos de forma intuitiva
    o comportamento das propriedades termodinâmicas mensuráveis pressão,
    temperatura e volume específico da água considerada como uma substância
    pura seja em uma única fase, seja na na forma de mistura heterogênea de
    fases, com grande ênfase em líquido e vapor.  Elaboramos a respeito das
    equações de estado, definimos o fator de compressibilidade e introduzimos o
    conceito de que existem regiões onde a substância pura pode ser considerada
    gás perfeito.  Finalmente, generalizamos o comportamento de líquido e vapor
    para quaisquer substâncias puras e apresentamos a equação de estado
    generalizada. De posse dessas ideias, analisamos o comportamento geral das
    substâncias puras, dispensando algum espaço para a região em torno da
    saturação líquido-vapor e salientamos, a partir do conceito de equação de
    estado e de gás perfeito, que, para todas as substâncias puras haverá uma
    região de comportamento de gás (quase) perfeito. Apresentamos em seguida,
    no \cref{chap:theFirstLaw}, a Primeira Lei para sistema fechado e aberto,
    com a qual definimos as propriedades energia interna e entalpia. Para que o
    leitor se convença, de uma vez por todas, que uma lei não existe sem a
    outra, abordamos imediatamente, no \cref{chap:theSecondLaw}, a Segunda Lei
    para sistema fechado e aberto, abrindo caminho para a importante definição
    de entropia. Nada mais natural, portanto, que analisemos a combinação da
    Primeira e da Segunda lei, em seguida, no \cref{chap:exergyAnalysis},
    quando apresentamos os conceitos de trabalho reversível, irreversibilidade
    (trabalho perdido) e das propriedades --- combinadas com a vizinhança ---
    exergia, tanto para fluxo, como não fluxo. Embora a fundamentação anterior
    seja suficientemente geral, para que possamos resolver problemas de
    modelagem de problemas que envolvem engenharia de energia,
    particularizamos, no \cref{chap:thermodynamicProperties}, o nosso universo
    de trabalho, para sistemas compressíveis simples e, com a ajuda do
    Postulado dos Estados e de umas poucas ferramentas do Cálculo e de suas
    consequências para a Termodinâmica, desenvolvemos a importante noção de
    derivadas parciais, de relação funcional e de integral de linha entre as
    propriedades termodinâmicas dependentes e independentes.  Definimos as
    propriedades calores específicos respectivamente a pressão e a volume
    contantes, expansividade  voumétrica, compressibilidades isotérmica e
    adiabática, respectivamente e as relações entre elas. Com o objetivo de
    determinarmos as propriedades não mensuráveis, obtemos então, no
    \cref{chap:thermodynamicPropComputation}, as formas algébricas para as
    diferenças (e os desvios residuais) das propriedades termodinâmicas para
    qualquer substância pura: energia interna, entalpia e entropia, em
    coordenadas adimensionais generalizadas, as propriedades reduzidas.
    Desenvolvemos, assim, com o apoio da Regra dos Estados (quase)
    Correspondentes, o fator de compressibilidade. Apresentamos algumas
    equações de estado e, por escolha pessoal, nos fixamos na equação
    generalizada de Lee-Kesler, que nos leva imediatamente ao fator acêntrico
    de Pitzer e ao fator de correção \gls{WuPolarFactor} de Wu-Stiel, como uma
    das estratégias para obter resultados que correspondam melhor aos dados
    experimentais para as diversas substâncias puras.  Aproveitamos para
    definir a fugacidade (e o coeficiente de fugacidade) de substâncias puras.
    Imaginando que o leitor já está dotado da capacidade de obter as
    (diferenças de) propriedades termodinâmicas de qualquer substância pura,
    bem como as fugacidades, desde que lhe sejam fornecidos os seus dados do
    ponto crítico, o fator de Pitzer e a correção de Wu-Stiel, apresentamos o
    programa \command{\command{LK\_WS\_NR}} (substância pura generalizada) e o
    pacote \command{\command{LK\_proptermo}}. Este útimo  foi desenvolvido a
    partir da noção de que as propriedades termodinâmicas não mensuráveis de
    qualquer substância pura não têm um valor absoluto e portanto dependem de
    uma base arbitrária.  Utilizando uma base de dados efetivamente embutida no
    programa e extensível, abrimos assim ao aluno um universo enorme de
    aplicações envolvendo centenas de substâncias puras, sem a necessidade de
    lançar mão de uma única tabela sequer. Como uma sequência natural,
    abordamos, no \cref{chap:homogeneousMixtures}, as misturas homogêneas não
    reativas de substâncias puras de uma forma bem geral, a partir do Postulado
    Expandido dos Estados e da importante noção de propriedade molar parcial
    (ou parcial molar) e da fugacidade do componente na mistura, todos como
    propriedades intensivas da mistura. Com isso, abrimos espaço para a
    apresentação dos modelos de soluções ideais e de mistura de gases
    perfeitos, como casos particulares.  Para o caso destes modelos não se
    aplicarem, ou até mesmo para a simplificação da resolução de problemas,
    apresentamos ainda ao leitor o conceito de   pseudo-substância pura, com a
    escolha de algum modelo pseudo-crítico desde o mais simples, como a regra
    de Kay, ao razoavelmente sofisticado modelo de Lee-Kesler, que inclui os
    fatores acêntricos e até mesmo os fatores de Wu-Stiel dos componentes da
    mistura.  Sob o tópico de misturas reativas, no
    \cref{chap:homogeneousReactiveMixtures}, discutimos o grau de reação e as
    reações químicas e a necessidade de bases como as da entalpia de formação e
    da entropia absoluta. Tomando como exemplo a combustão, incluímos em
    detalhes uma discussão sobre reações reversíveis, equilíbrio químico e a
    minimização da função de Gibbs. Assim, mostramos a aplicação do método dos
    multiplicadores de Lagrange à mistura de produtos de combustão em
    equilíbrio.  Definimos e discutimos os potenciais químicos e a sua relação
    com as outras propriedades das misturas, bem como sua determinação por meio
    das propriedades das substâncias puras em algum estado padrão. Estudamos em
    seguida o método das constantes de equilíbrio e suas importantes
    consequências, como o comportamento do equilíbrio químico sob alterações de
    pressão e temperatura.  Posteriormente, analisamos com algum detalhe, no
    \cref{chap:heterogeneousMixtures}, novamente a partir da minimização da
    função de Gibbs, o equilíbrio de fases em misturas heterogêneas
    multicomponentes, exibindo exemplos de diagramas de fases para misturas
    binárias. Nesse ponto relembramos a propriedade fugacidade do componente na
    mistura. Com isso, podemos  discutir o modelo de solução ideal para cada
    fase da mistura e, finalmente, analisar o modelo extremamente simplificado
    da regra de Raoult-gás perfeito, o qual pode ser utilizado no equilíbrio de
    fases líquido-vapor  sob pressões moderadas. No
    \cref{chap:thermodynamicCycles}, abordamos sucintamente o tópico de Ciclos
    Termodinâmicos, que deve ser, na verdade, considerado como aplicações de
    tudo o que foi visto nos capítulos anteriores. Dividimos, então, em: ciclos
    com mudança de fase de Rankine e de refrigeração por compressão de vapor;
    os ciclos padrão de ar de Otto, Diesel e Stirling, que modelam sob
    determinadas limitações os motores de combustão interna; o ciclo padrão de
    ar de Brayton, que modela os motores de jatos e as turbinas a gás; e os
    ciclos mistos ou combinados.

    Se seremos bem-sucedidos em nosso propósito de apresentar um texto compacto
    de Termodinâmica Clássica que auxilie o leitor não especialista na
    modelagem estruturada dos aspectos termodinâmicos não triviais da realidade
    e que lhe desperte o interesse para aprender muito mais, só o tempo e a
    experiência dirão. Porém, se isso porventura ocorrer, remetemos o leitor
    para a curta Bibliografia comentada ao final da obra para auxiliar na
    jornada.

    Agradecemos aos colegas professores do Departamento de Engenharia Mecânica,
    pelo apoio e pela oportunidade que tivemos de elaborar e submeter as
    versões preliminares do presente texto a uma audiência que, embora de
    formação em ciências exatas, ainda assim era relativamente heterogênea.
    Agradecemos especialmente aos nossos amigos e ex-alunos, hoje cientistas,
    Marcos e Colmeia por haverem respondido tão prontamente às demandas que
    lhes impusemos quando do início do desenvolvimento dos códigos
    computacionais em Python. Cremos que, além de
    \command{\command{LK\_proptermo}} e talvez a clássica Refprop, ainda hoje
    não existem outras bibliotecas para a determinação de propriedades
    termodinâmicas de código completamente aberto, cuja fundamentação está ao
    alcance de qualquer pessoa. Finalmente, estendemos nossos agradecimentos às
    várias centenas de estudantes das disciplinas Termodinâmica I e II, do
    Curso de Engenharia Mecânica, que, ao utilizarem o presente texto, nos
    forneceram a oportunidade de continuamente aperfeiçoá-lo, assim como ao
    código computacional correspondente.  \command{LK\_proptermo} é de fato uma
    API de alto nível sobre o código original \command{\command{LK\_WS\_NR}}
    cujo desenvolvimento foi instigado pelas demandas dos usuários.  Assim como
    \command{\command{LK\_WS\_NR}}, é importante repetir que o código-fonte de
    \command{\command{LK\_proptermo}} esteve desde o início disponível a todos
    que o quisessem. Esperamos sinceramente corresponder às expectativas de
    todos.

    \begin{center}
        \textit{Ilha Solteira, aos 07 dias de Janeiro de 2018.}

        \textbf{Emanuel Rocha Woiski}
    \end{center}
