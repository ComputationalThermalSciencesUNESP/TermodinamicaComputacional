\chapter{Misturas Homogêneas Reativas}
\label{chap:homogeneousReactiveMixtures}

    Até agora lidamos com misturas cujos componentes não reagem entre si. Se
    existir esta possibilidade para a mistura, então novos componentes poderão
    surgir magicamente. Embora o processo deva atender à conservação de massa e
    às leis da Termodinâmica, poderá haver uma alteração significativa das
    propriedades termodinâmicas e da composição da mistura ao seu fim. Então o
    problema-chave aqui é: se algumas substâncias puras da mistura reagirem
    entre si, como aplicaremos as leis da termodinâmica e da conservação da
    massa? Como seremos capazes de escrever as equações das diferenças das
    propriedades termodinâmicas, se as substâncias puras envolvidas poderão ser
    completamente substituídas por outras distintas?

    A resposta está na conservação da identidade de todas as espécies químicas
    envolvidas, pois as reações químicas sempre preservam os núcleos dos
    átomos. Devemos ser capazes, também por causa disso, de desenvolver bases
    para as propriedades termodinâmicas, cujo grau de arbitrariedade deve
    refletir e ser consistente com esta conservação.


    \section{Uma Questão de Massa} \label{sec:chemicalReactions}

    As reações químicas representam, devido à conservação das espécies químicas
    envolvidas, simplesmente uma proporção entre os integrantes da reação.
    Sabe-se então que a modificação da quantidade de qualquer um dos
    componentes da reação refletirá obrigatoriamente sobre a quantidade de
    todos os outros exatamente na proporção dos seus coeficientes
    estequiométricos naquela reação.

    Portanto, é importante que fique claro que os coeficientes estequiométricos
    da reação não têm, em princípio, nada a ver com a massa (ou número de
    moles) instantânea dos componentes da mistura, sejam eles integrantes das
    diversas reações que podem ocorrer entre eles na mistura ou completamente
    inertes nas condições dadas da mistura.

    Nas condições da mistura, uma reação pode ser irreversível, quando pelo
    menos um dos reagentes se esgota ao fim do processo da reação química, ou
    então reversível, quando todos os envolvidos na reação, sejam
    \enquote{reagentes} e \enquote{produtos} passam ao final de um certo tempo
    de relaxação, a coexistir em equilíbrio dinâmico e, para as restrições
    estabelecidas, este representa o fim do processo. Equilíbrio dinâmico
    significa que a taxa de destruição de cada componente da mistura torna-se
    igual a sua taxa de produção, de forma que a composição macroscópica da
    mistura não se altera. As condições para que exista um equilíbrio dinâmico
    em uma reação química, ou seja, para que ela seja reversível, serão objeto
    de nosso estudo. Iniciaremos trabalhando sobre o exemplo detalhado da
    reação de combustão do metano, porém sem considerar que os produtos por sua
    vez podem se converter em outros, dissociando-se ou se recombinando.


    \section{O Grau de Reação} \label{sec:reactionExtent}

    A qualquer reação química pode ser atribuído \emph{um grau (ou extensão) de
    reação \gls{reactionExtent}}, através do qual se pode avaliar o avanço da
    reação de reagentes para produtos, ou vice-versa.

    Para exemplificar, seja uma mistura em um sistema fechado contendo
    inicialmente \state{\propcomp{numberMoles}{\ch{CH4}}}{0} moles de \ch{CH4},
    \state{\propcomp{numberMoles}{\ch{O2}}}{0} moles de \ch{O2}, associados à
    quantidade adequada de ar puro seco,
    \state{\propcomp{numberMoles}{\ch{CO2}}}{0} moles de \ch{CO2} e também
    \state{\propcomp{numberMoles}{\ch{H2O}}}{0} kmoles de \ch{H2O}. Vamos
    avaliar o avanço da reação e discutir o resultado final.

    Consideremos a reação química de combustão do metano com oxigênio puro, nas
    proporções estequiométricas, sem sobra de oxigênio no produtos e sem
    dissociação dos produtos:
    %
    \begin{equation} \label{eq:8.1}
        \ch{1 CH4 + 2 O2 -> 1 CO2 + 2 H2O}
    \end{equation}

    Quantos \gls{numberMoles} moles de ar são necessários para
    \propcomp{numberMoles}{\ch{O2}} moles de \ch{O2}? O ar padrão é definido
    como \SI{21}{\percent} de \ch{O2} e \SI{79}{\percent} de \ch{N2}, em base
    molar. Para cada mol de \ch{O2} temos que ter \SI{4.76}{\mol} de ar (por
    quê?). Portanto, $\gls{numberMoles} = \propcomp{numberMoles}{\ch{O2}} +
    \num{3.76}\propcomp{numberMoles}{\ch{N2}} =
    \num{4.76}\propcomp{numberMoles}{Ar}$ moles de ar .

    Para analisar em detalhes, estudemos a \cref{tab:reactionProgress}.

    \begin{table}[!htb]
        \caption{%
            Efeito do grau de avanço sobre cada componente da
            reação
        }

        \begin{tabular}{c c c c c}
            \toprule
                {Coef.}
            &   {1}
            &   {2}
            &   {1}
            &   {2}\\
            \midrule
                {Inicial}
            &   \state{\propcomp{numberMoles}{\ch{CH4}}}{0}
            &   \state{\propcomp{numberMoles}{\ch{O2}}}{0}
            &   \state{\propcomp{numberMoles}{\ch{CO2}}}{0}
            &   \state{\propcomp{numberMoles}{\ch{H2O}}}{0}\\
            \cmidrule(lr){1-5}
                {Avanço}
            &   -1\gls{reactionExtent}
            &   -2\gls{reactionExtent}
            &    1\gls{reactionExtent}
            &    2\gls{reactionExtent}\\
            \cmidrule(lr){1-5}
                {Local}
            &   \state{\propcomp{numberMoles}{\ch{CH4}}}{0} - 1\gls{reactionExtent}
            &   \state{\propcomp{numberMoles}{\ch{O2}}}{0}  - 2\gls{reactionExtent}
            &   \state{\propcomp{numberMoles}{\ch{CO2}}}{0} + 1\gls{reactionExtent}
            &   \state{\propcomp{numberMoles}{\ch{H2O}}}{0} + 2\gls{reactionExtent}\\
            \cmidrule(lr){1-5}
                {Final}
            &   0
            &   $
                    \state{\propcomp{numberMoles}{\ch{O2}}}{0}
                    -
                    2\state{\propcomp{numberMoles}{\ch{CH4}}}{0}
                $
            &   $
                    \state{\propcomp{numberMoles}{\ch{CO2}}}{0}
                    +
                    \state{\propcomp{numberMoles}{\ch{CH4}}}{0}
                $
            &   $
                    \state{\propcomp{numberMoles}{\ch{H2O}}}{0}
                    +
                    2\state{\propcomp{numberMoles}{\ch{CH4}}}{0}
                $\\
            \bottomrule
        \end{tabular}
        \label{tab:reactionProgress}
    \end{table}

    Na \cref{tab:reactionProgress} a linha \emph{Coef.} representa os
    coeficientes estequiométricos da reação original \cref{eq:8.1}; a linha
    \emph{Inicial}, os valores que iniciaram a reação química, fornecidos
    externamente pelas condições iniciais da mistura; a linha \emph{Avanço}
    representa o avanço da reação definido pelo produto dos coeficientes
    estequiométricos e o grau de avanço \gls{reactionExtent}; a linha
    \emph{Local}, os valores instantâneos dos números de moles de cada
    componente da reação em função de \gls{reactionExtent}; a linha
    \emph{Final}, os valores dos números de kmoles de cada componente da
    reação, supondo-se que esta se deslocou totalmente para a direita,
    caracterizando uma combustão completa, a qual esgotou pelo menos um dos
    reagentes no processo, no caso, o metano.

    O número instantâneo de moles, bem como a sua variação, para cada um dos
    componentes da mistura será dado por:
    %
    \begin{equation} \label{eq:8.2}
        \begin{aligned}
        &\propcomp{numberMoles}{\ch{CH4}}
        \functionOf{
            \gls{reactionExtent}
        }
        =
        \state{\propcomp{numberMoles}{\ch{CH4}}}{0}
        -
        \gls{reactionExtent}
        \,\,\,\,
        \rightarrow
        \,\,\,\,
        \diff{\propcomp{numberMoles}{\ch{CH4}}}
        \functionOf{
            \gls{reactionExtent}
        }
        =
        -
        \diff{\gls{reactionExtent}}\\
        %
        &\propcomp{numberMoles}{\ch{O2}}
        \functionOf{
            \gls{reactionExtent}
        }
        =
        \state{\propcomp{numberMoles}{\ch{O2}}}{0}
        -
        2\gls{reactionExtent}
        \,\,\,\,
        \rightarrow
        \,\,\,\,
        \diff{\propcomp{numberMoles}{\ch{O2}}}
        \functionOf{
            \gls{reactionExtent}
        }
        =
        -2\diff{\gls{reactionExtent}}\\
        %
        &\propcomp{numberMoles}{\ch{CO2}}
        \functionOf{
            \gls{reactionExtent}
        }
        =
        \state{\propcomp{numberMoles}{\ch{CO2}}}{0}
        +
        \gls{reactionExtent}
        \,\,\,\,
        \rightarrow
        \,\,\,\,
        \diff{\propcomp{numberMoles}{\ch{CO2}}}
        \functionOf{
            \gls{reactionExtent}
        }
        =
        \diff{\gls{reactionExtent}}\\
        %
        &\propcomp{numberMoles}{\ch{H2O}}
        \functionOf{
            \gls{reactionExtent}
        }
        =
        \state{\propcomp{numberMoles}{\ch{H2O}}}{0}
        +
        2\gls{reactionExtent}
        \,\,\,\,
        \rightarrow
        \,\,\,\,
        \diff{\propcomp{numberMoles}{\ch{H2O}}}
        \functionOf{
            \gls{reactionExtent}
        }
        =
        2\diff{\gls{reactionExtent}}\,.\\
        \end{aligned}
    \end{equation}

    Como vemos, a variação do número de moles é proporcional ao grau de avanço
    da reação, através do respectivo coeficiente estequiométrico.

    O número instantâneo de moles da mistura será dado por:
    %
    \begin{equation} \label{eq:8.3}
        \gsub{numberMoles}{mixture}(\gls{reactionExtent})
        =
        \state{\propcomp{numberMoles}{\ch{CH4}}}{\gls{standardState}}
        +
        \state{\propcomp{numberMoles}{\ch{O2}}}{\gls{standardState}}
        +
        \state{\propcomp{numberMoles}{\ch{CO2}}}{\gls{standardState}}
        +
        \state{\propcomp{numberMoles}{\ch{H2O}}}{\gls{standardState}}
        +
        3,76\propcomp{numberMoles}{\ch{O2}}
        -
        \gls{reactionExtent}
        -
        2\gls{reactionExtent}
        +
        \gls{reactionExtent}
        +
        2\gls{reactionExtent}\,.
    \end{equation}

    O número de moles instantâneo da mistura,
    \gsub{numberMoles}{mixture}\functionOf{\gls{reactionExtent}}, deve incluir
    também o número de moles de nitrogênio \ch{N2}, que, embora inerte na
    reação química em questão, pertence também à mistura.

    As frações molares dos componentes da mistura são dadas por:
    %
    \begin{equation} \label{eq:8.4}
        \propcomp{moleFraction}{\ch{CH4}}
        =
        \frac{
            \state{\propcomp{numberMoles}{\ch{CH4}}}{\gls{standardState}}
            -
            \gls{reactionExtent}
        }{
            \gsub{numberMoles}{mixture}
            \functionOf{
                \gls{reactionExtent}
            }
        }\,,
    \end{equation}
    %
    \begin{equation} \label{eq:8.5}
        \propcomp{moleFraction}{\ch{O2}}
        =
        \frac{
            \state{\propcomp{numberMoles}{\ch{O2}}}{\gls{standardState}}
            -
            2\gls{reactionExtent}
        }{
            \gsub{numberMoles}{mixture}
            \functionOf{
                \gls{reactionExtent}
            }
        }\,,
    \end{equation}
    %
    \begin{equation} \label{eq:8.6}
        \propcomp{moleFraction}{\ch{CO2}}
        =
        \frac{
            \state{\propcomp{numberMoles}{\ch{CO2}}}{\gls{standardState}}
            +
            \gls{reactionExtent}
        }{
            \gsub{numberMoles}{mixture}
            \functionOf{
                \gls{reactionExtent}
            }
        }\,,
    \end{equation}
    %
    \begin{equation} \label{eq:8.7}
        \propcomp{moleFraction}{\ch{H2O}}
        =
        \frac{
            \state{\propcomp{numberMoles}{\ch{H2O}}}{\gls{standardState}}
            +
            2\gls{reactionExtent}
        }{
            \gsub{numberMoles}{mixture}
            \functionOf{
                \gls{reactionExtent}
            }
        }\,,
    \end{equation}
    %
    e, também,
    %
    \begin{equation} \label{eq:8.8}
        \propcomp{moleFraction}{\ch{N2}}
        =
        \frac{
            \num{3.76}
            \state{\propcomp{numberMoles}{\ch{O2}}}{\gls{standardState}}
        }{
            \gsub{numberMoles}{mixture}
            \functionOf{
                \gls{reactionExtent}
            }
        }\,.
    \end{equation}

    Se todo o metano for queimado, então  $\state{\gls{reactionExtent}}{1} =
    \state{\propcomp{numberMoles}{\ch{CH4}}}{\gls{standardState}}$ e basta
    substituirmos este  valor de \gls{reactionExtent} para obtermos que o
    oxigênio restante será
    %
    \begin{equation} \label{eq:8.9}
        \state{\propcomp{numberMoles}{\ch{O2}}}{1}
        =
        \state{\propcomp{numberMoles}{\ch{O2}}}{\gls{standardState}}
        -
        2\state{\propcomp{numberMoles}{\ch{CH4}}}{\gls{standardState}}\,.
    \end{equation}

    Observe que qualquer que seja o número de moles, nunca pode ser negativo,
    de modo que as frações molares das \cref{eq:8.4} a \cref{eq:8.8} bem como a
    \cref{eq:8.9}, devem satisfazer essa restrição para qualquer valor de
    \gls{reactionExtent}.

    Define-se a relação ar-combustível molar (ou em massa) \gls{airFuelRatio}
    como o número de moles (ou massa) de ar por mol (ou por unidade de massa)
    de combustível. O mínimo número de moles \gls{numberMoles} de ar que
    garante a combustão completa na reação fornecida será aquele que extingue
    também o oxigênio. Este é denominado o \emph{ar teórico ou estequiométrico}
    \gls{stoichiometricAirFuelRatio} desta reação. Portanto:

    \begin{equation} \label{eq:8.10}
        \gls{stoichiometricAirFuelRatio}
        =
        \min{
            \left(
                \frac{
                    \gls{numberMoles}
                }{
                    \state{\propcomp{numberMoles}{\ch{CH4}}}{0}
                }
            \right)
        }
        =
        \frac{
            \num{4.76}
            \left(
                2\state{\propcomp{numberMoles}{\ch{CH4}}}{0}
            \right)
        }{
                \state{\propcomp{numberMoles}{\ch{CH4}}}{0}
        }
        =
        \num{9.52}\,\,\,\,
        \text{moles de ar/mol de \ch{CH4}}\,.
    \end{equation}

    Naturalmente, se $\state{\propcomp{numberMoles}{\ch{CH4}}}{0} = 1$, então
    $\min{\gls{numberMoles}} = \num{9.52}$ e o número inicial de moles de
    oxigênio será simplesmente 2, como era de se esperar.

    A relação combustível-ar \gls{fuelAirRatio} para uma dada reação de
    combustão é implesmente a inversa de \gls{airFuelRatio}, assim como
    \gls{stoichiometricFuelAirRatio} é a inversa de
    \gls{stoichiometricAirFuelRatio}. Define-se adicionalmente a relação de
    equivalência \gls{equivalenceRatio} como:
    %
    \begin{equation} \label{eq:8.11}
        \gls{equivalenceRatio}
        =
        \frac{
            \gls{fuelAirRatio}
        }{
            \gls{stoichiometricFuelAirRatio}
        }
        =
        \frac{
            \gls{stoichiometricAirFuelRatio}
        }{
            \gls{airFuelRatio}
        }\,.
    \end{equation}

    O inverso da relação de equivalência \gls{equivalenceRatio} é conhecido na
    indústria por \gls{rEquivalenceRatio}. A relação de equivalência,
    \cref{eq:8.11}, é um parâmetro importante porque se \gls{equivalenceRatio}
    é menor do que 1, a reação de combustão, qualquer que seja ela, contém mais
    oxigênio do que o estequiométrico e portanto, haverá sobra de oxigênio nos
    produtos. Se \gls{equivalenceRatio} for maior do que 1, por outro lado, o
    número de moles de oxigênio será menor do que o estequiométrico e a reação
    se torna muito mais complicada.

    Na realidade, a Termodinâmica clássica é capaz de dizer se alguma reação
    pode ou não ocorrer, mas não pode predizer o que vai acontecer, ou que
    produtos serão formados.

    Precisamos agora aplicar a conservação de massa e as Leis da Termodinâmica.
    Para isso, vamos retomar o assunto das bases arbitrárias para as
    propriedades termodinâmicas.


    \section{A Base da Entalpia de Formação} \label{sec:formationEnthalpy}

    Seja a seguinte reação química em regime permanente, com reagentes
    separados à temperatura padrão \gls{standardTemperature} de
    \SI{298.15}{\kelvin} e à pressão padrão \gls{standardPressure} de
    \SI{0.1}{\mega\pascal} e o único produto também a \SI{298.15}{\kelvin} e
    \SI{0.1}{\mega\pascal}:
    %
    \begin{equation} \label{eq:8.12}
        \ch{C_{(s)} + O2_{g} -> CO2_{(g)}}\,.
    \end{equation}

    Se aplicarmos a primeira lei, obteremos
    %
    \begin{equation} \label{eq:8.13}
        \gsub{heatTransfer}{controlVolume}
        =
        \propcomp{enthalpy}{\ch{CO2}}
        -
        \propcomp{enthalpy}{\ch{C(s)}}
        -
        \propcomp{enthalpy}{\ch{O2}}
        =
        \molalpropcomp{intEnthalpy}{\ch{CO2}}
        -
        \molalpropcomp{intEnthalpy}{\ch{C(s)}}
        -
        \molalpropcomp{intEnthalpy}{\ch{O2}}\,.
    \end{equation}

    Observe que \gsub{heatTransfer}{controlVolume}, que é mensurável e
    tabelado, não pode depender das bases arbitrariamente escolhidas para as
    entalpias. Então, se adotarmos uma base arbitrária para o carbono \ch{C(s)}
    e outra para o oxigênio \ch{O2}, a entalpia do \ch{CO2} será fixada por
    estas duas bases e pelo valor de \gsub{heatTransfer}{controlVolume}.

    Ora, se podemos escolher as bases, vamos adotar
    $\propcomp{formationEnthalpy}{\ch{C(s)}} =
    \propcomp{formationEnthalpy}{\ch{O2}} = 0$, ou seja, vamos atribuir valor
    zero à entalpia de todos os elementos (em sua forma isotópica mais comum) a
    $\gls{standardTemperature} = \SI{25}{\celsius}$ e $\gls{standardPressure} =
    \SI{0.1}{\mega\pascal}$. Além disso, o valor de \gls{formationEnthalpy}
    será tabelado para os gases a \SI{25}{\celsius}, considerados perfeitos.
    Esta é a fórmula que define a \emph{base da entalpia de formação
    \gls{formationEnthalpy}}.

    Em consequência da primeira lei e das bases escolhidas, a entalpia do
    \ch{CO2} na base da entalpia de formação fica fixada e igual a
    %
    \begin{equation}   \label{eq:8.14}
        \propcomp{formationEnthalpy}{\ch{CO2}}
        =
        \gsub{heatTransfer}{controlVolume}
        =
        \SI{-393522}{\kilo\joule\per\kilo\mol}\,.
    \end{equation}

    Os valores de \gls{formationEnthalpy} para as substâncias puras são
    determinados por medidas calorimétricas e com a ajuda da termodinâmica
    estatística e podem ser obtidos em tabelas. Para manter a consistência,
    \emph{todas as entalpias envolvidas na reação precisam ser convertidas para
    a mesma base da entalpia de formação}.

    Como qualquer conversão de base, funciona da seguinte forma, onde a e b são
    duas bases distintas para a propriedade termodinâmica
    \gls{intThermodynamicProperty}:
    %
    \begin{equation} \label{eq:8.15}
        {\Delta \gls{intThermodynamicProperty}}_\text{base a}
        =
        {\Delta \gls{intThermodynamicProperty}}_\text{base b}
        \,\,\,\,
        \text{ou}
        \,\,\,\,
        \left(
           \state{\gls{intThermodynamicProperty}}{2}
           -
           \state{\gls{intThermodynamicProperty}}{1}
        \right)_{\text{base a}}
        =
        \left(
           \state{\gls{intThermodynamicProperty}}{2}
           -
           \state{\gls{intThermodynamicProperty}}{1}
        \right)_{\text{base b}}\,.
    \end{equation}

    É muito importante que lembremos que na \cref{eq:8.15} existe um valor da
    propriedade termodinâmica \state{\gls{intThermodynamicProperty}}{i} na base
    a e um valor  distinto para a mesma propriedade
    \state{\gls{intThermodynamicProperty}}{i} na base b, mas estas correspondem
    ao mesmo estado termodinâmico $(i)$ e portanto, à mesma fase. Assim, se o
    estado $(i)$ for considerado como gás perfeito, as duas propriedades
    \gls{intThermodynamicProperty} devem se referir ao gás perfeito em $(i)$.

    Para converter qualquer entalpia molar a
    $(\gls{temperature},\gls{pressure})$ para a base da entalpia de formação
    usamos então a expressão:
    %
    \begin{equation} \label{eq:8.16}
        \molar{\gls{intEnthalpy}}
        \functionOf{
            \gls{temperature},
            \gls{pressure}
        }
        =
        \gls{formationEnthalpy}
        +
        \underset{
            \gls{standardTemperature},
            \gls{standardPressure}
            \rightarrow
            \gls{temperature},
            \gls{pressure}
        }{
            \Delta\molar{\gls{intEnthalpy}}
        }{}\,.
    \end{equation}

    Todas as transformações ficaram encerradas em $\Delta
    \molar{\gls{intEnthalpy}}$ que, como já sabemos, pode ser expressa em
    qualquer base. Vamos supor que o estado $\gls{temperature}, \gls{pressure}$
    de uma determinada substância pura não esteja na região de gás perfeito.
    Estamos interessados em expressar
    $\molar{\gls{intEnthalpy}}\functionOf{\gls{temperature},\gls{pressure}}$ na
    base da entalpia de formação. Sejam T0 e P0 os parâmetros para o estado
    padrão.  Então, utilizando para $\Delta \molar{\gls{intEnthalpy}}$ a
    expansão das entalpias em coordenadas generalizadas,
    %
    \begin{equation} \label{eq:8.17}
        \molar{\gls{intEnthalpy}}
        \functionOf{
            \gls{temperature},
            \gls{pressure}
        }
        =
        \gls{formationEnthalpy}
        +
        \left[
            \molalidealgasprop{intEnthalpy}
            \functionOf{
                \gls{temperature}
            }
            -
            \molalidealgasprop{intEnthalpy}
            \functionOf{
                \gls{standardTemperature}
            }
        \right]
        -
        \left(
            \molalidealgasprop{intEnthalpy}
            -
            \molar{\gls{intEnthalpy}}
        \right)
        \functionOf{
            \gsub{temperature}{reduced},
            \gsub{pressure}{reduced},
            \gls{PitzerAcentricFactor},
            \gls{WuPolarFactor}
        }
        +
        \state{
            \left(
                \molalidealgasprop{intEnthalpy}
                -
                \molar{\gls{intEnthalpy}}
            \right)
        }{\gls{standardState}}
        \functionOf{
            \gsub{standardTemperature}{reduced},
            \gsub{standardPressure}{reduced},
            \gls{PitzerAcentricFactor},
            \gls{WuPolarFactor}
        }\,.
    \end{equation}

    Observe que na \cref{eq:8.17} todas as entalpias devem corresponder umas às
    outras, de forma que %
    $\molalidealgasprop{intEnthalpy}
    \functionOf{
        \gls{temperature}
    }$ %
    e %
    $\molar{\gls{intEnthalpy}}
    \functionOf{
        \gls{temperature},
        \gls{pressure}
    }$%
    ambas \enquote{cancelam} %
    $(
        \molalidealgasprop{intEnthalpy}
        -
        \molar{\gls{intEnthalpy}}
    )
    \functionOf{
        \gsub{temperature}{reduced},
        \gsub{pressure}{reduced},
        \gls{PitzerAcentricFactor},
        \gls{WuPolarFactor}
    }$, %
    ao passo que %
    $
        \state{{\molalidealgasprop{intEnthalpy}}}{\gls{standardState}}
        \functionOf{\gls{standardTemperature}}
    $ %
    e %
    $\gls{formationEnthalpy}
    \functionOf{
        \gls{standardTemperature},
        \gls{standardPressure}
    }$ %
    ambas \enquote{cancelam} %
    $\state{(
        \molalidealgasprop{intEnthalpy}
        -
        \molar{\gls{intEnthalpy}}
    )}{\gls{standardState}}
    \functionOf{
        \gsub{standardTemperature}{reduced},
        \gsub{standardPressure}{reduced},
        \gls{PitzerAcentricFactor},
        \gls{WuPolarFactor}
    }$. Por outro lado, o valor de
    \gls{formationEnthalpy}\functionOf{\gls{standardTemperature},
    \gls{standardPressure}} deve ser fornecido ou obtido de tabelas.

    Você consegue modificar a \cref{eq:8.16} se consideramos
    \idealgasprop{formationEnthalpy}\functionOf{\gls{standardTemperature}}? E
    se adicionalmente
    \molalidealgasprop{intEnthalpy}\functionOf{\gls{temperature}}, ou seja a
    substância pura puder ser considerada como gás perfeito a
    \gls{temperature},\gls{pressure}? Você precisa se sentir à vontade com cada
    uma destas conversões antes de prosseguir (faça!).


    \section{A Base da Entropia Absoluta} \label{sec:absoluteEntropy}

    Nas reações químicas, ao aplicarmos a Segunda Lei, nos deparamos com uma
    situação análoga a da entalpia. Poderíamos, então, utilizar uma solução
    semelhante a da entalpia de formação. Entretanto, no caso da entropia,
    costumamos adotar uma outra base: \emph{a da entropia absoluta}.

    Sabemos que a entropia tem um valor \emph{nulo} para um cristal
    absolutamente perfeito na temperatura de zero absoluto. Este é um dos
    enunciados da \emph{Terceira Lei da Termodinâmica}, devido a Nernst e Max
    Planck, no começo do século \romannumber{20}.

    Portanto, se determinarmos o valor da entropia para os elementos químicos
    empregando a base da entropia absoluta, poderíamos utilizar estes valores
    para garantir a consistência da entropia dos componentes das reações
    químicas.

    Para efeito de tabelas, costuma-se adotar a referência padrão
    $\gls{standardPressure} = \SI{0.1}{\mega\pascal}$ e a temperatura padrão
    $\gls{standardTemperature} = \SI{25}{\celsius}$, da mesma forma que
    procedemos para a entalpia de formação. Novamente, não se deve confundir
    $(\gsub{pressure}{environmentState}, \gsub{temperature}{environmentState})$
    condições do meio-ambiente padrão, com $(\gls{standardTemperature},
    \gls{standardPressure})$, \emph{temperatura e pressão do estado padrão},
    embora muitas vezes seus valores possam coincidir por conveniência.

    A mudança de base para a entropia absoluta segue os mesmos princípios que
    estabelecemos para a entalpia de formação. A conversão de base será
    %
    \begin{equation} \label{eq:8.18}
        \molar{\gls{intEntropy}}
        \functionOf{
            \gls{temperature},
            \gls{pressure}
        }
        =
        \gls{absoluteEntropy}
        \functionOf{
            \gls{standardTemperature},
            \gls{standardPressure}
        }
        +
        \underset{
            \gls{standardTemperature},
            \gls{standardPressure}
            \rightarrow
            \gls{temperature},
            \gls{pressure}
        }{
            \Delta\molar{\gls{intEntropy}}
        }{}\,.
    \end{equation}

    % Comment: não seria melhor indicar a entropia molar no estado padrao com a
    % nomeclatura de estado? s^(0)? Isso evitaria confusão com a entropia
    % absoluta não?
    Da mesma forma que para a entalpia, a transformação da entropia está
    encerrada em $\Delta \molar{\gls{intEntropy}}$. A formulação completa em
    coordenadas generalizadas é a seguinte, para o caso em que
    $\molar{\gls{intEntropy}}\functionOf{\gls{temperature},\gls{pressure}}$ e
    $
        \molar{\gls{intEntropy}}_{\gls{standardTemperature}}^{\gls{standardState}}
        =
        \molar{\gls{intEntropy}}
        \functionOf{
            \gls{standardTemperature},
            \gls{standardPressure}
        }
    $ não sejam gases perfeitos:
    %
    % Comment: a entropia absoluta deve mesmo ser indicada com subscrito T0?
    \begin{equation} \label{eq:8.19}
        \begin{aligned}
        \molar{\gls{intEntropy}}
        \functionOf{
            \gls{temperature},
            \gls{pressure}
        }
        &=
        \molar{\gls{intEntropy}}_{\gls{standardTemperature}}^{\gls{standardState}}
        %\gls{absoluteEntropy}
        \functionOf{
            \gls{standardTemperature},
            \gls{standardPressure}
        }
        -
        \left(
            \molalidealgasprop{intEntropy}
            -
            \molar{\gls{intEntropy}}
        \right)
        \functionOf{
            \gsub{temperature}{reduced},
            \gsub{pressure}{reduced},
            \gls{PitzerAcentricFactor},
            \gls{WuPolarFactor}
        }
        +
        \state{
            \left(
                \molalidealgasprop{intEntropy}
                -
                \molar{\gls{intEntropy}}
            \right)
        }{\gls{standardState}}
        \functionOf{
            \gsub{standardTemperature}{reduced},
            \gsub{standardPressure}{reduced},
            \gls{PitzerAcentricFactor},
            \gls{WuPolarFactor}
        }\\
        &+
        \left[
            \molalidealgasprop{intEntropy}
            \functionOf{
                \gls{temperature},
                \gls{pressure}
            }
            -
            \molalidealgasprop{intEntropy}
            \functionOf{
                \gls{standardTemperature},
                \gls{standardPressure}
            }
        \right]\,.
        \end{aligned}
    \end{equation}

    Devemos ficar particularmente atentos para as pressões envolvidas, pois a
    entropia do gás perfeito, ao contrário da entalpia e energia interna,
    depende da temperatura e da pressão. Da mesma forma que a expansão da
    entalpia, para verificarmos se a Eq. 8.19 está correta deve haver uma
    correspondência entre as entropias no mesmo estado. Assim, ambos os termos
    $
        \molar{\gls{intEntropy}}_{\gls{standardTemperature}}^{\gls{standardState}}
        \functionOf{
            \gls{standardTemperature},
            \gls{standardPressure}
        }
    $ e %
    $
        \molalidealgasprop{intEntropy}
        \functionOf{
            \gls{standardTemperature},
            \gls{standardPressure}
        }
    $ devem \enquote{cancelar} %
    $\state{(
        \molalidealgasprop{intEntropy}
        -
        \molar{\gls{intEntropy}}
    }{\gls{standardState}}
    \functionOf{
        \gsub{standardTemperature}{reduced},
        \gsub{standardPressure}{reduced},
        \gls{PitzerAcentricFactor},
        \gls{WuPolarFactor}
    }$,
    ao passo que, da mesma forma, os termos %
    $
        \molar{\gls{intEntropy}}
        \functionOf{
            \gls{temperature},
            \gls{pressure}
        }
    $ e %
    $
        \molalidealgasprop{intEntropy}
        \functionOf{
            \gls{temperature},
            \gls{pressure}
        }
    $ devem \enquote{cancelar} %
    $(
        \molalidealgasprop{intEntropy}
        -
        \molar{\gls{intEntropy}}
    )\functionOf{
        \gsub{temperature}{reduced},
        \gsub{pressure}{reduced},
        \gls{PitzerAcentricFactor},
        \gls{WuPolarFactor}
    }$. Como fica a \cref{eq:8.19} para %
    $
        {\molalidealgasprop{intEntropy}}_{\gls{standardTemperature}}^{\gls{standardState}}
        \functionOf{
            \gls{standardTemperature},
            \gls{standardPressure}
        }
    $? E se adicionalmente,
    $\molalidealgasprop{intEntropy}\functionOf{\gls{temperature},
    \gls{pressure}}$? Você não deveria prosseguir até se sentir confortável com
    todas as conversões para a base da entropia absoluta.


    \section{Revendo o Exemplo agora com Reações Químicas}

    Seja o exemplo da \cref{fig:nonResistedExpansion}, agora com A = \ch{CH4} e
    B = ar puro. Resolva novamente o problema da expansão não resistida,
    incluindo agora a reação química de combustão de \propcomp{numberMoles}{A}
    moles de metano com \propcomp{numberMoles}{B} moles de oxigênio
    provenientes da quantidade adequada de ar puro, depois que se rompe a
    membrana.

    Da nossa discussão anterior sobre grau de reação (lembra-se?), já
    determinamos que o número final de moles de \ch{O2} será
    $\propcomp{numberMoles}{B} - 2\propcomp{numberMoles}{A}$. Então, a reação
    completa, sem dissociação, será
    %
    \begin{equation} \label{eq:8.20}
        \ch{
            \propcomp{numberMoles}{A} CH4
            +
            \propcomp{numberMoles}{B} O2
            +
            3.76 \propcomp{numberMoles}{B} N2
            ->
            \propcomp{numberMoles}{A} CO2
            +
            (
                \propcomp{numberMoles}{B}
                -
                2 \propcomp{numberMoles}{A}
            ) O2
            +
            2 \propcomp{numberMoles}{A} H2O
            +
            3.76 \propcomp{numberMoles}{B} N2
        }\,.
    \end{equation}

    Da mesma forma que antes, você deve mostrar que a Primeira Lei para o
    processo neste sistema será dada pela \cref{eq:7.20} e a Segunda Lei pela
    \cref{eq:7.21}. O lado A é composto de metano a
    \state{\propcomp{temperature}{A}}{1},\state{\propcomp{pressure}{A}}{1} e o
    lado B conterá ar puro seco a
    \state{\propcomp{temperature}{B}}{1},\state{\propcomp{pressure}{B}}{1}. A
    energia interna do sistema em 1 será, então
    %
    \begin{equation} \label{eq:8.21}
        \state{\gls{internalEnergy}}{1}
        =
        \propcomp{numberMoles}{A}
        \state{\molalpropcomp{intInternalEnergy}{\ch{CH4}}}{1}
        \functionOf{
            \state{\propcomp{temperature}{A}}{1},
            \state{\propcomp{pressure}{A}}{1}
        }
        +
        \propcomp{numberMoles}{B}
        \state{\partmolalprop{internalEnergy}{\ch{O2}}}{1}
        +
        \num{3.76}\propcomp{numberMoles}{B}
        \state{\partmolalprop{internalEnergy}{\ch{N2}}}{1}\,,
    \end{equation}
    %
    ao passo que a energia interna do sistema em 2 ficará
    %
    \begin{equation} \label{eq:8.22}
        \state{\gls{internalEnergy}}{2}
        =
        \propcomp{numberMoles}{A}
        \state{\partmolalprop{internalEnergy}{\ch{CO2}}}{2}
        +
        (\propcomp{numberMoles}{B}-2\propcomp{numberMoles}{A})
        \state{\partmolalprop{internalEnergy}{\ch{O2}}}{2}
        +
        2\propcomp{numberMoles}{A}
        \state{\partmolalprop{internalEnergy}{\ch{H2O}}}{2}
        +
        \num{3.76}\propcomp{numberMoles}{B}
        \state{\partmolalprop{internalEnergy}{\ch{N2}}}{2}\,.
    \end{equation}

    Vamos supor em 1 que o ar é um gás perfeito, uma hipótese bem realista para
    o ar e que o metano é um gás real. Já a mistura no estado 2 será assumida
    como uma solução ideal. Então:
    %
    \begin{equation} \label{eq:8.23}
        \state{\gls{internalEnergy}}{1}
        =
        \propcomp{numberMoles}{A}
        \state{\molalpropcomp{intInternalEnergy}{\ch{CH4}}}{1}
        \functionOf{
            \state{\propcomp{temperature}{A}}{1},
            \state{\propcomp{pressure}{A}}{1}
        }
        +
        \propcomp{numberMoles}{B}
        \state{
            {\molalidealgasprop{intInternalEnergy}}_{\ch{O2}}
        }{1}
        \functionOf{
            \state{\propcomp{temperature}{B}}{1}
        }
        +
        \num{3.76}\propcomp{numberMoles}{B}
        \state{
            {\molalidealgasprop{intInternalEnergy}}_{\ch{N2}}
        }{1}
        \functionOf{
            \state{\propcomp{temperature}{B}}{1}
        }\,.
    \end{equation}

    A energia interna no estado 2 será expressa como
    %
    \begin{equation} \label{eq:8.24}
        \state{\gls{internalEnergy}}{2}
        =
        \propcomp{numberMoles}{A}
        \state{\molalpropcomp{intInternalEnergy}{\ch{CO2}}}{2}
        +
        (\propcomp{numberMoles}{B}-2\propcomp{numberMoles}{A})
        \state{\molalpropcomp{intInternalEnergy}{\ch{O2}}}{2}
        +
        2\propcomp{numberMoles}{A}
        \state{\molalpropcomp{intInternalEnergy}{\ch{H2O}}}{2}
        +
        \num{3.76}\propcomp{numberMoles}{B}
        \state{\molalpropcomp{intInternalEnergy}{\ch{N2}}}{2}\,.
    \end{equation}

    Substituindo-se as energias internas por entalpias, para o estado 1:
    %
    \begin{equation} \label{eq:8.25}
        \begin{aligned}
        \state{\gls{internalEnergy}}{1}
        =&\,\,
        \propcomp{numberMoles}{A}
        \state{\molalpropcomp{intEnthalpy}{\ch{CH4}}}{1}
        \functionOf{
            \state{\propcomp{temperature}{A}}{1},
            \state{\propcomp{pressure}{A}}{1}
        }
        +
        \propcomp{numberMoles}{B}
        \state{
            {\molalidealgasprop{intEnthalpy}}_{\ch{O2}}
        }{1}
        \functionOf{
            \state{\propcomp{temperature}{B}}{1}
        }
        +
        \num{3.76}\propcomp{numberMoles}{B}
        \state{
            {\molalidealgasprop{intEnthalpy}}_{\ch{N2}}
        }{1}
        \functionOf{
            \state{\propcomp{temperature}{B}}{1}
        }\\
        &-
        \propcomp{numberMoles}{A}
        \state{\propcomp{compressibilityFactor}{\ch{CH4}}}{1}
        \functionOf{
            \state{\propcomp{temperature}{A}}{1},
            \state{\propcomp{pressure}{A}}{1}
        }
        \gls{universalGasConstant}
        \state{\propcomp{temperature}{A}}{1}
        -
        \num{4.76}\propcomp{numberMoles}{B}
        \gls{universalGasConstant}
        \state{\propcomp{temperature}{B}}{1}\,,
        \end{aligned}
    \end{equation}

    e para o estado 2:

    \begin{equation} \label{eq:8.26}
        \begin{aligned}
        \state{\gls{internalEnergy}}{2}
        =&\,\,
        \propcomp{numberMoles}{A}
        \state{\molalpropcomp{intEnthalpy}{\ch{CO2}}}{2}
        +
        (\propcomp{numberMoles}{B}-2\propcomp{numberMoles}{A})
        \state{\molalpropcomp{intEnthalpy}{\ch{O2}}}{2}
        +
        2\propcomp{numberMoles}{A}
        \state{\molalpropcomp{intEnthalpy}{\ch{H2O}}}{2}
        +
        \num{3.76}\propcomp{numberMoles}{B}
        \state{\molalpropcomp{intEnthalpy}{\ch{N2}}}{2}\\
        &-
        \left[
            \propcomp{numberMoles}{A}
            \state{\propcomp{compressibilityFactor}{\ch{CO2}}}{2}
            \functionOf{
                \state{\gls{temperature}}{2},
                \state{\gls{pressure}}{2}
            }
            +
            (\propcomp{numberMoles}{B} - 2\propcomp{numberMoles}{B})
            \state{\propcomp{compressibilityFactor}{\ch{O2}}}{2}
            \functionOf{
                \state{\gls{temperature}}{2},
                \state{\gls{pressure}}{2}
            }
        \right]
        \gls{universalGasConstant}
        \state{\gls{temperature}}{2}\\
        &-
        \left[
            2\propcomp{numberMoles}{A}
            \state{\propcomp{compressibilityFactor}{\ch{H2O}}}{2}
            \functionOf{
                \state{\gls{temperature}}{2},
                \state{\gls{pressure}}{2}
            }
            +
            \num{3.76}\propcomp{numberMoles}{B}
            \state{\propcomp{compressibilityFactor}{\ch{N2}}}{2}
            \functionOf{
                \state{\gls{temperature}}{2},
                \state{\gls{pressure}}{2}
            }
        \right]
        \gls{universalGasConstant}
        \state{\gls{temperature}}{2}\,.
        \end{aligned}
    \end{equation}

    Agora, como acabamos de aprender, cada entalpia nas \cref{eq:8.25,eq:8.26}
    deve ser escrita em uma base comum, a base da entalpia de formação e na
    forma das entalpias residuais e coordenadas reduzidas, conforme a
    \cref{eq:8.17} de conversão de base:

    \begin{equation} \label{eq:8.27}
        \molalpropcomp{intEnthalpy}{\ch{CH4}}
        \functionOf{
            \state{\propcomp{temperature}{A}}{1},
            \state{\propcomp{pressure}{A}}{1}
        }
        =
        \idealgasprop{formationEnthalpy}_{\ch{CH4}}
        -
        \state{
            \left(
                \molalidealgasprop{intEnthalpy}
                -
                \molar{\gls{intEnthalpy}}
            \right)
        }{1}_{\ch{CH4}}
        \functionOf{
            \state{\gsub{temperature}{reduced}}{1},
            \state{\gsub{pressure}{reduced}}{1},
            \gls{PitzerAcentricFactor},
            \gls{WuPolarFactor}
        }_{\ch{CH4}}
        +
        \left[
            \molalidealgasprop{intEnthalpy}
            \functionOf{
                \state{\propcomp{temperature}{A}}{1}
            }
            -
            \molalidealgasprop{intEnthalpy}
            \functionOf{
                \gls{standardTemperature}
            }
        \right]_{\ch{CH4}}\,,
    \end{equation}
    %
    \begin{equation} \label{eq:8.28}
        \molalidealgasprop{intEnthalpy}_{\ch{O2}}
        \functionOf{
            \state{\propcomp{temperature}{B}}{1}
        }
        =
        \left[
            \molalidealgasprop{intEnthalpy}
            \functionOf{
                \state{\propcomp{temperature}{B}}{1}
            }
            -
            \molalidealgasprop{intEnthalpy}
            \functionOf{
                \gls{standardTemperature}
            }
        \right]_{\ch{O2}}\,,
    \end{equation}
    %
    \begin{equation} \label{eq:8.29}
        \molalidealgasprop{intEnthalpy}_{\ch{N2}}
        \functionOf{
            \state{\propcomp{temperature}{B}}{1}
        }
        =
        \left[
            \molalidealgasprop{intEnthalpy}
            \functionOf{
                \state{\propcomp{temperature}{B}}{1}
            }
            -
            \molalidealgasprop{intEnthalpy}
            \functionOf{
                \gls{standardTemperature}
            }
        \right]_{\ch{N2}}\,,
    \end{equation}
    %
    \begin{equation} \label{eq:8.30}
        \molalpropcomp{intEnthalpy}{\ch{CO2}}
        \functionOf{
            \state{\gls{temperature}}{2},
            \state{\gls{pressure}}{2}
        }
        =
        \idealgasprop{formationEnthalpy}_{\ch{CO2}}
        -
        \state{
            \left(
                \molalidealgasprop{intEnthalpy}
                -
                \molar{\gls{intEnthalpy}}
            \right)
        }{2}_{\ch{CO2}}
        \functionOf{
            \state{\gsub{temperature}{reduced}}{2},
            \state{\gsub{pressure}{reduced}}{2},
            \gls{PitzerAcentricFactor},
            \gls{WuPolarFactor}
        }_{\ch{CO2}}
        +
        \left[
            \molalidealgasprop{intEnthalpy}
            \functionOf{
                \state{\gls{temperature}}{2}
            }
            -
            \molalidealgasprop{intEnthalpy}
            \functionOf{
                \gls{standardTemperature}
            }
        \right]_{\ch{CO2}}\,,
    \end{equation}
    %
    \begin{equation} \label{eq:8.31}
        \molalpropcomp{intEnthalpy}{\ch{H2O}}
        \functionOf{
            \state{\gls{temperature}}{2},
            \state{\gls{pressure}}{2}
        }
        =
        \idealgasprop{formationEnthalpy}_{\ch{H2O}}
        -
        \state{
            \left(
                \molalidealgasprop{intEnthalpy}
                -
                \molar{\gls{intEnthalpy}}
            \right)
        }{2}_{\ch{H2O}}
        \functionOf{
            \state{\gsub{temperature}{reduced}}{2},
            \state{\gsub{pressure}{reduced}}{2},
            \gls{PitzerAcentricFactor},
            \gls{WuPolarFactor}
        }_{\ch{H2O}}
        +
        \left[
            \molalidealgasprop{intEnthalpy}
            \functionOf{
                \state{\gls{temperature}}{2}
            }
            -
            \molalidealgasprop{intEnthalpy}
            \functionOf{
                \gls{standardTemperature}
            }
        \right]_{\ch{H2O}}\,,
    \end{equation}
    %
    \begin{equation} \label{eq:8.32}
        \molalpropcomp{intEnthalpy}{\ch{O2}}
        \functionOf{
            \state{\gls{temperature}}{2},
            \state{\gls{pressure}}{2}
        }
        =
        -\state{
            \left(
                \molalidealgasprop{intEnthalpy}
                -
                \molar{\gls{intEnthalpy}}
            \right)
        }{2}_{\ch{O2}}
        \functionOf{
            \gsub{temperature}{reduced},
            \gsub{pressure}{reduced},
            \gls{PitzerAcentricFactor},
            \gls{WuPolarFactor}
        }_{\ch{O2}}^{(2)}
        +
        \left[
            \molalidealgasprop{intEnthalpy}
            \functionOf{
                \state{\gls{temperature}}{2}
            }
            -
            \molalidealgasprop{intEnthalpy}
            \functionOf{
                \gls{standardTemperature}
            }
        \right]_{\ch{O2}}
    \end{equation}
    %
    \begin{equation} \label{eq:8.33}
        \molalpropcomp{intEnthalpy}{\ch{N2}}
        \functionOf{
            \state{\gls{temperature}}{2},
            \state{\gls{pressure}}{2}
        }
        =
        -\state{
            \left(
                \molalidealgasprop{intEnthalpy}
                -
                \molar{\gls{intEnthalpy}}
            \right)
        }{2}_{\ch{N2}}
        \functionOf{
            \gsub{temperature}{reduced},
            \gsub{pressure}{reduced},
            \gls{PitzerAcentricFactor},
            \gls{WuPolarFactor}
        }_{\ch{N2}}^{(2)}
        +
        \left[
            \molalidealgasprop{intEnthalpy}
            \functionOf{
                \state{\gls{temperature}}{2}
            }
            -
            \molalidealgasprop{intEnthalpy}
            \functionOf{
                \gls{standardTemperature}
            }
        \right]_{\ch{N2}}\,,
    \end{equation}
    %
    cujos valores devem ser substituídos nas \cref{eq:8.25,eq:8.26}. Em cada
    uma das Eqs. (\ref{eq:8.27}) a (\ref{eq:8.33}), os valores de
    \gsub{temperature}{reduced}, \gsub{pressure}{reduced} devem ser
    determinados conhecendo-se os respectivos valores das propriedades críticas
    \gsub{temperature}{critical} e \gsub{critical}, além, é claro, de
    \gls{PitzerAcentricFactor}, \gls{WuPolarFactor}, \gls{molecularMass} e
    \idealgasprop{constPressureSpecificHeat}\functionOf{\gls{temperature}},
    para cada uma das substâncias puras.

    Entretanto, da mesma forma que o problema original, não conhecemos os
    valores de \state{\gls{temperature}}{2} e de \state{\gls{pressure}}{2} e,
    se não considerarmos as equações das propriedades reduzidas (pois são
    simplesmente proporcionais às propriedades criticas), resta-nos apenas a
    \cref{eq:7.20}. Como anteriormente, a outra equação será dada pela
    conservação do volume do sistema, \cref{eq:7.26}.

    As equações para os volumes dos estados 1 e 2 respectivamente, ficarão:
    %
    \begin{equation} \label{eq:8.34}
        \state{\gls{volume}}{1}
        =
        \state{\propcomp{volume}{A}}{1}
        +
        \state{\propcomp{volume}{B}}{1}
        =
        \frac{
            \propcomp{numberMoles}{A}
            \propcomp{compressibilityFactor}{\ch{CH4}}
            \functionOf{
                \gsub{temperature}{reduced},
                \gsub{pressure}{reduced},
                \gls{PitzerAcentricFactor},
                \gls{WuPolarFactor}
            }_{\ch{CH4}}^{(1)}
            \gls{universalGasConstant}
            \state{\propcomp{temperature}{A}}{1}
        }{
            \state{\propcomp{pressure}{A}}{1}
        }
        +
        \frac{
            \num{4.76}\propcomp{numberMoles}{B}
            \gls{universalGasConstant}
            \state{\propcomp{temperature}{B}}{1}
        }{
            \state{\propcomp{pressure}{B}}{1}
        }\,,
    \end{equation}

    \begin{equation} \label{eq:8.35}
        \begin{aligned}
        &\left[
            \propcomp{numberMoles}{A}
            \state{\propcomp{compressibilityFactor}{\ch{CO2}}}{2}
            \functionOf{
                \state{\gls{temperature}}{2},
                \state{\gls{pressure}}{2}
            }
            +
            (\propcomp{numberMoles}{B} - 2\propcomp{numberMoles}{B})
            \state{\propcomp{compressibilityFactor}{\ch{O2}}}{2}
            \functionOf{
                \state{\gls{temperature}}{2},
                \state{\gls{pressure}}{2}
            }
        \right]
        \frac{
            \gls{universalGasConstant}
            \state{\gls{temperature}}{2}
        }{
            \state{\gls{pressure}}{2}
        }\\
        -
        &\left[
            2\propcomp{numberMoles}{A}
            \state{\propcomp{compressibilityFactor}{\ch{H2O}}}{2}
            \functionOf{
                \state{\gls{temperature}}{2},
                \state{\gls{pressure}}{2}
            }
            +
            \num{3.76}\propcomp{numberMoles}{B}
            \state{\propcomp{compressibilityFactor}{\ch{N2}}}{2}
            \functionOf{
                \state{\gls{temperature}}{2},
                \state{\gls{pressure}}{2}
            }
        \right]
        \frac{
            \gls{universalGasConstant}
            \state{\gls{temperature}}{2}
        }{
            \state{\gls{pressure}}{2}
        }
        =
        \state{\gls{volume}}{2}\,.
        \end{aligned}
    \end{equation}

    Uma vez resolvidos os valores de \state{\gls{temperature}}{2} e
    \state{\gls{pressure}}{2}, podemos aplicar na Segunda Lei da Termodinâmica,
    \cref{eq:7.21}, para obtermos a produção irreversível de entropia
    \fprocess{entropyCreated}{1}{2}{}. O processo é uma expansão não resistida,
    que é irreversível, associada a uma reação de combustão, que também é
    altamente irreversível. Estamos, portanto, interessados em avaliar a
    produção irreversível de entropia (ou irreversibilidade). Para isto,
    precisamos aplicar a Segunda Lei, cuja formulação básica não sofreu
    alteração, pois nada foi alterado na fronteira do sistema. A entropia no
    estado 1 será
    %
    \begin{equation} \label{eq:8.36}
        \state{\gls{entropy}}{1}
        =
        \propcomp{numberMoles}{A}
        \state{\molalpropcomp{intEntropy}{\ch{CH4}}}{1}
        \functionOf{
            \state{\propcomp{temperature}{A}}{1},
            \state{\propcomp{pressure}{A}}{1}
        }
        +
        \propcomp{numberMoles}{B}
        \state{\partmolalprop{entropy}{\ch{O2}}}{1}
        +
        \num{3.76}\propcomp{numberMoles}{B}
        \state{\partmolalprop{entropy}{\ch{N2}}}{1}\,,
    \end{equation}
    %
    e no estado 2:
    %
    \begin{equation} \label{eq:8.37}
        \state{\gls{entropy}}{2}
        =
        \propcomp{numberMoles}{A}
        \state{\partmolalprop{entropy}{\ch{CO2}}}{2}
        +
        (\propcomp{numberMoles}{B}-2\propcomp{numberMoles}{A})
        \state{\partmolalprop{entropy}{\ch{O2}}}{2}
        +
        2\propcomp{numberMoles}{A}
        \state{\partmolalprop{entropy}{\ch{H2O}}}{2}
        +
        \num{3.76}\propcomp{numberMoles}{B}
        \state{\partmolalprop{entropy}{\ch{N2}}}{2}\,.
    \end{equation}

    Considerando ar puro seco como gás perfeito e metano como gás real em 1 e
    solução ideal em 2, obtemos para a entropia em 1:
    %
    \begin{equation} \label{eq:8.38}
        \begin{aligned}
        \state{\gls{entropy}}{1}
        =
        &\,\,\propcomp{numberMoles}{A}
        \state{\molalpropcomp{intEntropy}{\ch{CH4}}}{1}
        \functionOf{
            \state{\propcomp{temperature}{A}}{1},
            \state{\propcomp{pressure}{A}}{1}
        }
        +
        \propcomp{numberMoles}{B}
        \left[
            \state{{\molalidealgasprop{intEntropy}}}{1}_{\ch{O2}}
            \functionOf{
                \state{\propcomp{temperature}{B}}{1},
                \state{\propcomp{pressure}{B}}{1}
            }
            -
            \gls{universalGasConstant}
            \ln{
                \state{\propcomp{moleFraction}{\ch{O2}}}{1}
            }
        \right]\\
        &+
        \num{3.76}\propcomp{numberMoles}{B}
        \left[
            \state{{\molalidealgasprop{intEntropy}}}{1}_{\ch{N2}}
            \functionOf{
                \state{\propcomp{temperature}{B}}{1},
                \state{\propcomp{pressure}{B}}{1}
            }
            -
            \gls{universalGasConstant}
            \ln{
                \state{\propcomp{moleFraction}{\ch{N2}}}{1}
            }
        \right]\,,
    \end{aligned}
    \end{equation}
    %
    e para a entropia em 2:

    \begin{equation} \label{eq:8.39}
    \begin{aligned}
        \state{\gls{entropy}}{2}
        =
        &\,\,\propcomp{numberMoles}{A}
        \left[
            \state{\molalpropcomp{intEntropy}{\ch{CO2}}}{2}
            \functionOf{
                \state{\gls{temperature}}{2},
                \state{\gls{pressure}}{2}
            }
            -
            \gls{universalGasConstant}
            \ln{
                \state{\propcomp{moleFraction}{\ch{CO2}}}{2}
            }
        \right]
        +
        (\propcomp{numberMoles}{B} - 2\propcomp{numberMoles}{A})
        \left[
            \state{\molalpropcomp{intEntropy}{\ch{O2}}}{2}
            \functionOf{
                \state{\gls{temperature}}{2},
                \state{\gls{pressure}}{2}
            }
            -
            \gls{universalGasConstant}
            \ln{
                \state{\propcomp{moleFraction}{\ch{O2}}}{2}
            }
        \right]\\
        &+
        2\propcomp{numberMoles}{A}
        \left[
            \state{\molalpropcomp{intEntropy}{\ch{H2O}}}{2}
            \functionOf{
                \state{\gls{temperature}}{2},
                \state{\gls{pressure}}{2}
            }
            -
            \gls{universalGasConstant}
            \ln{
                \state{\propcomp{moleFraction}{\ch{H2O}}}{2}
            }
        \right]
        +
        \num{3.76}\propcomp{numberMoles}{B}
        \left[
            \state{\molalpropcomp{intEntropy}{\ch{N2}}}{2}
            \functionOf{
                \state{\gls{temperature}}{2},
                \state{\gls{pressure}}{2}
            }
            -
            \gls{universalGasConstant}
            \ln{
                \state{\propcomp{moleFraction}{\ch{N2}}}{2}
            }
        \right]\,.
    \end{aligned}
    \end{equation}

    De forma semelhante às entalpias, precisamos converter cada uma das
    entropias para a base da entropia absoluta.  Para isso, lançaremos mão das
    entropias residuais. Assim,
    %
    \begin{equation} \label{eq:8.40}
        \molalpropcomp{intEntropy}{\ch{CH4}}
        \functionOf{
            \state{\propcomp{temperature}{A}}{1},
            \state{\propcomp{pressure}{A}}{1}
        }
        =
        \idealgasprop{absoluteEntropy}_{\ch{CH4}}
        -
        \left(
            \molalidealgasprop{intEntropy}
            -
            \molar{\gls{intEntropy}}
        \right)_{\ch{CH4}}
        \functionOf{
            \gsub{temperature}{reduced},
            \gsub{pressure}{reduced},
            \gls{PitzerAcentricFactor},
            \gls{WuPolarFactor}
        }_{\ch{CH4}}^{(1)}
        +
        \left[
            \molalidealgasprop{intEntropy}
            \functionOf{
                \state{\propcomp{temperature}{A}}{1},
                \state{\propcomp{pressure}{A}}{1}
            }
            -
            \molalidealgasprop{intEntropy}
            \functionOf{
                \gls{standardTemperature},
                \gls{standardPressure}
            }
        \right]_{\ch{CH4}}\,,
    \end{equation}
    %
    \begin{equation} \label{eq:8.41}
        \molalpropcomp{intEntropy}{\ch{O2}}
        \functionOf{
            \state{\propcomp{temperature}{B}}{1},
            \state{\propcomp{pressure}{B}}{1}
        }
        =
        \idealgasprop{absoluteEntropy}_{\ch{O2}}
        -
        \left(
            \molalidealgasprop{intEntropy}
            -
            \molar{\gls{intEntropy}}
        \right)_{\ch{O2}}
        \functionOf{
            \gsub{temperature}{reduced},
            \gsub{pressure}{reduced},
            \gls{PitzerAcentricFactor},
            \gls{WuPolarFactor}
        }_{\ch{O2}}^{(1)}
        +
        \left[
            \molalidealgasprop{intEntropy}
            \functionOf{
                \state{\propcomp{temperature}{B}}{1},
                \state{\propcomp{pressure}{B}}{1}
            }
            -
            \molalidealgasprop{intEntropy}
            \functionOf{
                \gls{standardTemperature},
                \gls{standardPressure}
            }
        \right]_{\ch{O2}}\,,
    \end{equation}
    %
    \begin{equation} \label{eq:8.42}
        \molalpropcomp{intEntropy}{\ch{N2}}
        \functionOf{
            \state{\propcomp{temperature}{B}}{1},
            \state{\propcomp{pressure}{B}}{1}
        }
        =
        \idealgasprop{absoluteEntropy}_{\ch{N2}}
        -
        \left(
            \molalidealgasprop{intEntropy}
            -
            \molar{\gls{intEntropy}}
        \right)_{\ch{N2}}
        \functionOf{
            \gsub{temperature}{reduced},
            \gsub{pressure}{reduced},
            \gls{PitzerAcentricFactor},
            \gls{WuPolarFactor}
        }_{\ch{N2}}^{(1)}
        +
        \left[
            \molalidealgasprop{intEntropy}
            \functionOf{
                \state{\propcomp{temperature}{B}}{1},
                \state{\propcomp{pressure}{B}}{1}
            }
            -
            \molalidealgasprop{intEntropy}
            \functionOf{
                \gls{standardTemperature},
                \gls{standardPressure}
            }
        \right]_{\ch{N2}}\,,
    \end{equation}
    %
    \begin{equation} \label{eq:8.43}
        \molalpropcomp{intEntropy}{\ch{CO2}}
        \functionOf{
            \state{\gls{temperature}}{2},
            \state{\gls{pressure}}{2}
        }
        =
        \idealgasprop{absoluteEntropy}_{\ch{CO2}}
        -
        \left(
            \molalidealgasprop{intEntropy}
            -
            \molar{\gls{intEntropy}}
        \right)_{\ch{CO2}}
        \functionOf{
            \gsub{temperature}{reduced},
            \gsub{pressure}{reduced},
            \gls{PitzerAcentricFactor},
            \gls{WuPolarFactor}
        }_{\ch{CO2}}^{(2)}
        +
        \left[
            \molalidealgasprop{intEntropy}
            \functionOf{
                \state{\gls{temperature}}{2},
                \state{\gls{pressure}{B}}{2}
            }
            -
            \molalidealgasprop{intEntropy}
            \functionOf{
                \gls{standardTemperature},
                \gls{standardPressure}
            }
        \right]_{\ch{CO2}}\,,
    \end{equation}
    %
    \begin{equation} \label{eq:8.44}
        \molalpropcomp{intEntropy}{\ch{H2O}}
        \functionOf{
            \state{\gls{temperature}}{2},
            \state{\gls{pressure}}{2}
        }
        =
        \idealgasprop{absoluteEntropy}_{\ch{H2O}}
        -
        \left(
            \molalidealgasprop{intEntropy}
            -
            \molar{\gls{intEntropy}}
        \right)_{\ch{H2O}}
        \functionOf{
            \gsub{temperature}{reduced},
            \gsub{pressure}{reduced},
            \gls{PitzerAcentricFactor},
            \gls{WuPolarFactor}
        }_{\ch{H2O}}^{(2)}
        +
        \left[
            \molalidealgasprop{intEntropy}
            \functionOf{
                \state{\gls{temperature}}{2},
                \state{\gls{pressure}{B}}{2}
            }
            -
            \molalidealgasprop{intEntropy}
            \functionOf{
                \gls{standardTemperature},
                \gls{standardPressure}
            }
        \right]_{\ch{H2O}}\,,
    \end{equation}
    %
    \begin{equation} \label{eq:8.45}
        \molalpropcomp{intEntropy}{\ch{O2}}
        \functionOf{
            \state{\gls{temperature}}{2},
            \state{\gls{pressure}}{2}
        }
        =
        \idealgasprop{absoluteEntropy}_{\ch{O2}}
        -
        \left(
            \molalidealgasprop{intEntropy}
            -
            \molar{\gls{intEntropy}}
        \right)_{\ch{O2}}
        \functionOf{
            \gsub{temperature}{reduced},
            \gsub{pressure}{reduced},
            \gls{PitzerAcentricFactor},
            \gls{WuPolarFactor}
        }_{\ch{O2}}^{(2)}
        +
        \left[
            \molalidealgasprop{intEntropy}
            \functionOf{
                \state{\gls{temperature}}{2},
                \state{\gls{pressure}{B}}{2}
            }
            -
            \molalidealgasprop{intEntropy}
            \functionOf{
                \gls{standardTemperature},
                \gls{standardPressure}
            }
        \right]_{\ch{O2}}\,,
    \end{equation}
    %
    \begin{equation} \label{eq:8.46}
        \molalpropcomp{intEntropy}{\ch{N2}}
        \functionOf{
            \state{\gls{temperature}}{2},
            \state{\gls{pressure}}{2}
        }
        =
        \idealgasprop{absoluteEntropy}_{\ch{N2}}
        -
        \left(
            \molalidealgasprop{intEntropy}
            -
            \molar{\gls{intEntropy}}
        \right)_{\ch{N2}}
        \functionOf{
            \gsub{temperature}{reduced},
            \gsub{pressure}{reduced},
            \gls{PitzerAcentricFactor},
            \gls{WuPolarFactor}
        }_{\ch{N2}}^{(2)}
        +
        \left[
            \molalidealgasprop{intEntropy}
            \functionOf{
                \state{\gls{temperature}}{2},
                \state{\gls{pressure}{B}}{2}
            }
            -
            \molalidealgasprop{intEntropy}
            \functionOf{
                \gls{standardTemperature},
                \gls{standardPressure}
            }
        \right]_{\ch{N2}}\,,
    \end{equation}
    %
    cujos valores devem ser substituídos nas \cref{eq:8.38,eq:8.39}. A
    propósito, se pudéssemos separar a irreversibilidade causada somente pela
    expansão não resistida, poderíamos avaliar os efeitos da combustão sobre a
    irreversibilidade total (Como você poderia realizar essa separação?).

    Se a mistura em 2 fosse uma mistura de gases perfeitos, então todos os
    termos $(\molalidealgasprop{intEnthalpy} -
    \molar{\gls{intEnthalpy}})\functionOf{\gsub{temperature}{reduced},
        \gsub{pressure}{reduced}, \gls{PitzerAcentricFactor},
    \gls{WuPolarFactor}}$ e $(\molalidealgasprop{intEntropy} -
    \molar{\gls{intEntropy}})\functionOf{\gsub{temperature}{reduced},
        \gsub{pressure}{reduced}, \gls{PitzerAcentricFactor},
    \gls{WuPolarFactor}}$ seriam nulos e todos os \gls{compressibilityFactor}
    seriam iguais a 1 (faça como um exercício).

    Agora, vamos considerar a mistura em 2 como uma pseudo-substância pura,
    então a energia interna, a entropia e o volume do estado 1 não se alteram.
    A equação da energia interna em 2 ficará:
    %
    \begin{equation} \label{eq:8.47}
        \begin{aligned}
        \state{\gls{internalEnergy}}{2}
        =&\,\,
        \propcomp{numberMoles}{A}
        \state{{\molalidealgasprop{intEnthalpy}}}{2}_{\ch{CO2}}
        \functionOf{
            \state{\gls{temperature}}{2}
        }
        +
        (\propcomp{numberMoles}{B}-2\propcomp{numberMoles}{A})
        \state{{\molalidealgasprop{intEnthalpy}}}{2}_{\ch{O2}}
        \functionOf{
            \state{\gls{temperature}}{2}
        }
        +
        2\propcomp{numberMoles}{A}
        \state{{\molalidealgasprop{intEnthalpy}}}{2}_{\ch{H2O}}
        \functionOf{
            \state{\gls{temperature}}{2}
        }
        +
        \num{3.76}\propcomp{numberMoles}{B}
        \state{{\molalidealgasprop{intEnthalpy}}}{2}_{\ch{N2}}
        \functionOf{
            \state{\gls{temperature}}{2}
        }\\
        &+
        \left[
            \propcomp{numberMoles}{A}
            +
            (\propcomp{numberMoles}{B}-2\propcomp{numberMoles}{A})
            +
            2\propcomp{numberMoles}{A}
            +
            \num{3.76}\propcomp{numberMoles}{B}
        \right]
        \left[
            -\gls{universalGasConstant}
            \state{\gls{temperature}}{2}
            +
            \left(
                \molalidealgasprop{intEnthalpy}
                -
                \molar{\gls{intEnthalpy}}
            \right)_{\gls{mixture}}
            \functionOf{
                \gsub{temperature}{reduced},
                \gsub{pressure}{reduced},
                \gls{PitzerAcentricFactor},
                \gls{WuPolarFactor}
            }_{\gls{mixture}}^{(2)}
        \right]\,.
        \end{aligned}
    \end{equation}

    Como em todos os casos de modelos pseudo-críticos, consideramos uma mistura
    de gases perfeitos em 2 e corrigimos para a mistura como se fosse uma
    pseudo-substância pura. Uma vez determinados os valores de
    \gsub{temperature}{critical,mixture}, \gsub{pressure}{critical,mixture},
    \gsub{molecularMass}{mixture}, \gsub{PitzerAcentricFactor}{mixture}
    e \gsub{WuPolarFactor}{mixture}, avaliamos as propriedades reduzidas com
    subscrito \gls{mixture}.

    A equação para a igualdade dos volumes se torna:
    %
    \begin{equation} \label{eq:8.48}
        \begin{aligned}
        &\left[
            \propcomp{numberMoles}{A}
            \propcomp{compressibilityFactor}{\ch{CH4}}
            \functionOf{
                \gsub{temperature}{reduced},
                \gsub{pressure}{reduced},
                \gls{PitzerAcentricFactor},
                \gls{WuPolarFactor}
            }
            \frac{
                \gls{universalGasConstant}
                \state{\propcomp{temperature}{A}}{1}
            }{
                \state{\propcomp{pressure}{A}}{1}
            }
            +
            \num{4.76}\propcomp{numberMoles}{B}
            \frac{
                \gls{universalGasConstant}
                \state{\propcomp{temperature}{B}}{1}
            }{
                \state{\propcomp{pressure}{B}}{1}
            }
        \right]\\
        =
        &\left[
            \propcomp{numberMoles}{A}
            +
            (\propcomp{numberMoles}{B}-2\propcomp{numberMoles}{A})
            +
            2\propcomp{numberMoles}{A}
            +
            \num{3.76}\propcomp{numberMoles}{B}
        \right]
        \propcomp{compressibilityFactor}{\gls{mixture}}
        \functionOf{
            \gsub{temperature}{reduced},
            \gsub{pressure}{reduced},
            \gls{PitzerAcentricFactor},
            \gls{WuPolarFactor}
        }_{\gls{mixture}}^{(2)}
        \frac{
            \gls{universalGasConstant}
            \state{\gls{temperature}}{2}
        }{
            \state{\gls{pressure}}{2}
        }\,.
        \end{aligned}
    \end{equation}

    A entropia no estado 2 se torna
    %
    \begin{equation} \label{eq:8.49}
    \begin{aligned}
        \state{\gls{entropy}}{2}
        =
        &\,\,\propcomp{numberMoles}{A}
        \left[
            \state{{\molalidealgasprop{intEntropy}}_{\ch{CO2}}}{2}
            \functionOf{
                \state{\gls{temperature}}{2},
                \state{\gls{pressure}}{2}
            }
            -
            \gls{universalGasConstant}
            \ln{
                \state{\propcomp{moleFraction}{\ch{CO2}}}{2}
            }
        \right]
        +
        (\propcomp{numberMoles}{B} - 2\propcomp{numberMoles}{A})
        \left[
            \state{{\molalidealgasprop{intEntropy}}_{\ch{O2}}}{2}
            \functionOf{
                \state{\gls{temperature}}{2},
                \state{\gls{pressure}}{2}
            }
            -
            \gls{universalGasConstant}
            \ln{
                \state{\propcomp{moleFraction}{\ch{O2}}}{2}
            }
        \right]\\
        &+
        2\propcomp{numberMoles}{A}
        \left[
            \state{{\molalidealgasprop{intEntropy}}_{\ch{H2O}}}{2}
            \functionOf{
                \state{\gls{temperature}}{2},
                \state{\gls{pressure}}{2}
            }
            -
            \gls{universalGasConstant}
            \ln{
                \state{\propcomp{moleFraction}{\ch{H2O}}}{2}
            }
        \right]
        +
        \num{3.76}\propcomp{numberMoles}{B}
        \left[
            \state{{\molalidealgasprop{intEntropy}}_{\ch{N2}}}{2}
            \functionOf{
                \state{\gls{temperature}}{2},
                \state{\gls{pressure}}{2}
            }
            -
            \gls{universalGasConstant}
            \ln{
                \state{\propcomp{moleFraction}{\ch{N2}}}{2}
            }
        \right]\\
        &+
        \left[
            \propcomp{numberMoles}{A}
            +
            (\propcomp{numberMoles}{B}-2\propcomp{numberMoles}{A})
            +
            2\propcomp{numberMoles}{A}
            +
            \num{3.76}\propcomp{numberMoles}{B}
        \right]
        \left(
            \molalidealgasprop{intEntropy}
            -
            \molar{\gls{intEntropy}}
        \right)_{\gls{mixture}}
        \functionOf{
            \gsub{temperature}{reduced},
            \gsub{pressure}{reduced},
            \gls{PitzerAcentricFactor},
            \gls{WuPolarFactor}
        }_{\gls{mixture}}^{(2)}\,.
        \end{aligned}
    \end{equation}

    As equações de mudança de base para a entalpia de formação são as mesmas,
    exceto que os termos
    $(\molalidealgasprop{intEnthalpy} -
    \molar{\gls{intEnthalpy}})\functionOf{\gsub{temperature}{reduced},
        \gsub{pressure}{reduced}, \gls{PitzerAcentricFactor},
    \gls{WuPolarFactor}}$ e $(\molalidealgasprop{intEntropy} -
    \molar{\gls{intEntropy}})\functionOf{\gsub{temperature}{reduced},
        \gsub{pressure}{reduced}, \gls{PitzerAcentricFactor},
    \gls{WuPolarFactor}}$ são todos nulos, como no caso de gás perfeito.
    Precisamos da composição da mistura ao final da combustão, no estado 2. O
    total de moles da mistura \state{\gls{numberMoles}}{2} será a soma do
    número de moles dos componentes da mistura em 2. Assim,
    %
    \begin{equation} \label{eq:8.50}
        \state{\gls{numberMoles}}{2}
        =
        \propcomp{numberMoles}{A}
        +
        (\propcomp{numberMoles}{B}-2\propcomp{numberMoles}{A})
        +
        2\propcomp{numberMoles}{A}
        +
        \num{3.76}\propcomp{numberMoles}{B}
        =
        \propcomp{numberMoles}{A}
        +
        \num{4.76}\propcomp{numberMoles}{B}\,.
    \end{equation}

    Note que o número de moles no estado 2 é igual ao do estado 1, mas,
    cuidado, essa é uma situação particular para essa reação (por quê?). A
    composição da mistura em 2 torna-se
    %
    \begin{equation} \label{eq:8.51}
        \propcomp{moleFraction}{\ch{CO2}}
        =
        \frac{
            \propcomp{numberMoles}{A}
        }{
            \propcomp{numberMoles}{A}
            +
            \num{4.76}\propcomp{numberMoles}{B}
        }\,,
    \end{equation}
    %
    \begin{equation} \label{eq:8.52}
        \propcomp{moleFraction}{\ch{O2}}
        =
        \frac{
            \propcomp{numberMoles}{B}
            -
            2\propcomp{numberMoles}{A}
        }{
            \propcomp{numberMoles}{A}
            +
            4,76\propcomp{numberMoles}{B}
        }\,,
    \end{equation}
    %
    \begin{equation} \label{eq:8.53}
        \state{\propcomp{moleFraction}{\ch{H2O}}}{2}
        =
        \frac{
            2\propcomp{numberMoles}{A}
        }{
            \propcomp{numberMoles}{A}
            +
            \num{4.76}\propcomp{numberMoles}{B}
        }\,,
    \end{equation}
    %
    \begin{equation} \label{eq:8.54}
        \state{\propcomp{moleFraction}{\ch{N2}}}{2}
        =
        \frac{
            \num{3.76}\propcomp{numberMoles}{A}
        }{
            \propcomp{numberMoles}{A}
            +
            \num{4.76}\propcomp{numberMoles}{B}
        }\,.
    \end{equation}

    Para que o problema tenha sentido físico, todas as frações molares devem
    ser maiores ou iguais a zero. Neste caso, basta que
    $\propcomp{numberMoles}{B}-2\propcomp{numberMoles}{A} \ge 0$. Se
    $\propcomp{numberMoles}{B} = 2\propcomp{numberMoles}{A}$ então não haverá
    \ch{O2} na mistura final e, portanto, teremos queimado o metano
    precisamente com o ar teórico.

    Por outro lado, se fixarmos a temperatura e a pressão da mistura final,
    então, dependendo dos valores escolhidos para \state{\gls{temperature}}{2},
    \state{\gls{pressure}}{2} talvez haja uma (ou mais de uma) solução física
    para as únicas incógnitas \propcomp{numberMoles}{A} e
    \propcomp{numberMoles}{B}.


    \section{A Temperatura Adiabática de Chama e o Poder Calorífico}

    Como no presente exemplo, o processo é adiabático, pode-se prever que a
    temperatura e a pressão finais serão muito elevadas ($\approx
    \SI{3000}{\kelvin}$), pois toda a energia de combustão estará sendo
    utilizada para aquecer os produtos.

    A temperatura que ocorre sob um processo adiabático de combustão é muitas
    vezes chamada de \emph{temperatura adiabática de chama}, \gls{tac}, posto
    que a combustão se propaga como uma chama ao longo da câmara. Para se
    reduzir a temperatura final você acha que precisaria aumentar ou diminuir a
    quantidade de ar?

    Seja agora um sistema aberto (volume de controle) não adiabático no qual o
    combustível (1) entra nas condições padrão a \gls{standardTemperature},
    \gls{standardPressure} e o ar estequiométrico (2) também a
    \gls{standardTemperature}, \gls{standardPressure}. Suponha que os produtos
    de combustão (sem dissociação) saem separadamente, cada um deles a
    \gls{standardTemperature}, \gls{standardPressure}.  Para que isso aconteça,
    os produtos de combustão devem perder calor para o ambiente. A Primeira Lei
    aplicada a este sistema, considerando-se regime permanente, sem trabalho
    mecânico, desprezando-se as energias cinéticas e potenciais e se usarmos a
    vazão molar de combustível como referência de escala, ficará:
    %
    \begin{equation} \label{eq:8.55}
        \gsupsub{enthalpy}{standardState}{chProducts}
        -
        \gsupsub{enthalpy}{standardState}{chReactants}
        =
        \gls{heatTransfer}
        =
        \molar{\gls{intEnthalpy}}_{\gls{chReactants}\gls{chProducts}}^{\gls{standardState}}\,,
    \end{equation}
    %
    onde os subscritos \gls{chReactants} e \gls{chProducts} se referem a
    reagentes e produtos respectivamente e o sobrescrito refere-se ao estado
    padrão.

    Para fixar ideias, vamos assumir daqui em diante um combustível cuja
    composição molar equivalente pode ser escrita como
    \gls{chemicalGenericFuel}. Você é capaz de obter a quantidade
    estequiométrica de ar necessária para um mol desse combustível? Para isso,
    considere que os únicos produtos de combustão devem ser \ch{CO2}, \ch{H2O},
    \ch{O2} e \ch{N2} e conte o número de átomos de elementos em ambos os lados
    da seguinte reação de combustão:
    %
    % Comment: neste caso a razão de equivalencia aparece na equação?
    \begin{equation} \label{eq:8.56}
        \ch{
            \gls{chemicalGenericFuel}
            +
            $\frac{\gls{molarAirFuelRatio}}{\gls{equivalenceRatio}}$
            (
                O2
                +
                3.76 N2
            )
            ->
            \gls{numberMoles}_{ CO2 } CO2
            +
            \gls{numberMoles}_{ H2O } H2O
            +
            \propcomp{numberMoles}{O2} O2
            +
            \propcomp{numberMoles}{N2} N2
        }\,,
    \end{equation}
    %
    onde
    %
    \begin{equation} \label{eq:8.57}
        \gls{airFuelRatio}
        =
        \num{4.76}\frac{\gls{molarAirFuelRatio}}{\gls{equivalenceRatio}}\,,
    \end{equation}
    %
    e obterá então
    %
    \begin{equation} \label{eq:8.58}
        \begin{aligned}
            \ch{C}&\,:
            \gls{carbonQuantity} - \propcomp{numberMoles}{\ch{CO2}} = 0\,,\\
            %
            \ch{H}&\,:
            \gls{hidrogenQuantity} - 2\propcomp{numberMoles}{\ch{H2O}} = 0\,,\\
            %
            \ch{O}&\,:
            \gls{oxygenQuantity}
            +
            2\frac{\gls{molarAirFuelRatio}}{\gls{equivalenceRatio}}
            -
            \left(
                2\propcomp{numberMoles}{\ch{CO2}}
                +
                \propcomp{numberMoles}{\ch{H2O}}
                +
                2\propcomp{numberMoles}{\ch{O2}}
            \right) = 0\,,\\
            \ch{N}&\,:
            \gls{nitrogenQuantity}
            +
            \num{3.76}(2\gls{molarAirFuelRatio})
            -
            2\propcomp{numberMoles}{\ch{N2}} = 0\,,\\
        \end{aligned}
    \end{equation}
    %
    de onde encontramos que
    %
    \begin{equation} \label{eq:8.59}
        \gls{molarAirFuelRatio}
        =
        \gls{carbonQuantity}
        +
        \frac{\gls{hidrogenQuantity}}{4}
        -
        \frac{\gls{oxygenQuantity}}{2}\,,
    \end{equation}
    %
    e portanto, a reação estequiométrica ficará
    %
    % Comment: assumindo combustão completa neste caso?
    \begin{equation} \label{eq:8.60}
        \begin{aligned}
        \ch{
            \gls{chemicalGenericFuel}
            &+
            \bigg(
                \gls{carbonQuantity}
                +
                $\frac{\gls{hidrogenQuantity}}{4}$
                -
                $\frac{\gls{oxygenQuantity}}{2}$
            \bigg)
            (
                O2
                +
                3.76 N2
            )
            ->\\
            &\gls{carbonQuantity} CO2
            +
            $\frac{\gls{oxygenQuantity}}{2}$ H2O
            +
            \bigg[
                $\num{3.76}$
                \bigg(
                    \gls{carbonQuantity}
                    +
                    $\frac{\gls{hidrogenQuantity}}{4}$
                    -
                    $\frac{\gls{oxygenQuantity}}{2}$
                \bigg)
                +
                $\frac{\gls{nitrogenQuantity}}{2}$
            \bigg] N2
        }\,.
        \end{aligned}
    \end{equation}
    %
    Lembrando-nos que a base da entalpia de formação para \ch{O2} e \ch{N2} é
    zero, a \cref{eq:8.55} se torna
    %
    \begin{equation}
        \gls{combustionEnthalpy}
        =
        \gls{carbonQuantity}
        \propcomp{formationEnthalpy}{\ch{CO2}}
        +
        \frac{\gls{oxygenQuantity}}{2}
        \propcomp{formationEnthalpy}{\ch{H2O}}
        -
        \propcomp{formationEnthalpy}{\gls{chemicalGenericFuel}}\,,
    \end{equation}
    %
    onde \gls{combustionEnthalpy} é conhecido como \emph{calor de combustão},
    ou entalpia de combustão, um valor negativo (por quê?). Seu valor em módulo
    é denominado poder calorífico do combustível no estado padrão, em
    \si{\kilo\joule\per\kilo\mol} de combustível. Você pode mostrar que o valor
    de \gls{combustionEnthalpy} será alterado dependendo da fase em que o
    combustível se encontra e também dependendo da fase da água nos produtos.
    Se á agua estiver em estado de vapor, trata-se do \gls{pci}, ao passo que
    se a água estiver na fase condensada, está-se referindo ao \gls{pcs} (qual
    deles é maior? Por quê?).

    Já estabelecemos os fundamentos pelos quais utilizamos as Leis da
    Termodinâmica para alguns casos de misturas homogêneas reativas, incluindo
    as reações químicas que consomem um dos reagentes. Mas e as reações
    reversíveis? Como veremos, elas formam a essência do equilíbrio químico.


    \section{As Reações Químicas Reversíveis e O Equilíbrio Qu\'imico}

    Se os produtos de combustão de um combustível cuja composição equivalente é
    \gls{chemicalGenericFuel} são produtos de combustão  \ch{CO}, \ch{NO},
    \ch{H2}, \ch{H}, \ch{O}, \ch{OH}?  Vamos nos fixar nestes dez produtos,
    embora possam haver muitos mais, pois estes já nos permitem o modelamento
    razoavelmente realista de grande parte dos sistemas de combustão em
    condições normais.

    Consideremos a reação química do combustível com uma determinada quantidade
    de ar como ponto de partida e se houver um equilíbrio a \gls{temperature} e
    \gls{pressure}, será obtido a partir dessa mistura, da seguinte forma:
    %
    \begin{equation} \label{eq:8.62}
        \begin{aligned}
        \ch{
            &\gls{chemicalGenericFuel}
            +
            $\frac{\gls{molarAirFuelRatio}}{\gls{equivalenceRatio}}$
            (
                O2
                +
                3,76 N2
            )
            ->\\
            &\propcomp{numberMoles}{1} CO2
            +
            \propcomp{numberMoles}{2} H2O
            +
            \propcomp{numberMoles}{3} N2
            +
            \propcomp{numberMoles}{4} O2
            +
            \propcomp{numberMoles}{5} CO
            +
            \propcomp{numberMoles}{6} H2
            +
            \propcomp{numberMoles}{7} H
            +
            \propcomp{numberMoles}{8} O
            +
            \propcomp{numberMoles}{9} OH
            +
            \propcomp{numberMoles}{10} NO
        }\,.
        \end{aligned}
    \end{equation}

    A função de Gibbs da mistura de produtos de combustão a \gls{temperature} e
    \gls{pressure} será dada por:
    %
    \begin{equation} \label{eq:8.63}
        \gls{GibbsFreeEnergy}
        =
        \sum\limits_{\gls{numberSpecies}}{
            \propcomp{chemicalPotential}{i}
            \propcomp{numberMoles}{i}
        }\,,
        \,\,\,\,
        \forall i\,,
    \end{equation}
    %
    onde %
    $
        \propcomp{chemicalPotential}{i}
        =
        \partmolalprop{GibbsFreeEnergy}{i}
        =
        \ddxconsty{
            \gls{GibbsFreeEnergy}
        }{
            \propcomp{numberMoles}{i}
        }{
            \gls{temperature},
            \gls{pressure},
            \propcomp{numberMoles}{j \neq i}
        }
    $ é o potencial (eletro-)químico do componente $i$ na mistura a
    \gls{temperature} e \gls{pressure}, o mesmo que a função de Gibbs parcial
    molar do componente $i$ na mistura a \gls{temperature} e \gls{pressure}.

    No caso da nossa reação, a \cref{eq:8.63} para \gls{GibbsFreeEnergy} ficará
    da seguinte forma:
    %
    \begin{equation} \label{eq:8.64}
        \gls{GibbsFreeEnergy}
        =
        \propcomp{chemicalPotential}{1}
        \propcomp{numberMoles}{1}
        +
        \propcomp{chemicalPotential}{2}
        \propcomp{numberMoles}{2}
        +
        \propcomp{chemicalPotential}{3}
        \propcomp{numberMoles}{3}
        +
        \propcomp{chemicalPotential}{4}
        \propcomp{numberMoles}{4}
        +
        \propcomp{chemicalPotential}{5}
        \propcomp{numberMoles}{5}
        +
        \propcomp{chemicalPotential}{6}
        \propcomp{numberMoles}{6}
        +
        \propcomp{chemicalPotential}{7}
        \propcomp{numberMoles}{7}
        +
        \propcomp{chemicalPotential}{8}
        \propcomp{numberMoles}{8}
        +
        \propcomp{chemicalPotential}{9}
        \propcomp{numberMoles}{9}
        +
        \propcomp{chemicalPotential}{10}
        \propcomp{numberMoles}{10}
    \end{equation}

    O equilíbrio da mistura dos produtos de combustão a \gls{temperature},
    \gls{pressure} será dado pela minimização da função de Gibbs, ou seja,
    $\diff{\gls{GibbsFreeEnergy}} = 0$.Portanto, diferenciando
    \gls{GibbsFreeEnergy} e levando-se em conta as equações de Gibbs-Duhem (por
    quê?), obtemos:
    %
    \begin{equation} \label{eq:8.65}
        \begin{aligned}
            \diff{\gls{GibbsFreeEnergy}}
            &=
            \propcomp{chemicalPotential}{1}
            \diff{\propcomp{numberMoles}{1}}
            +
            \propcomp{chemicalPotential}{2}
            \diff{\propcomp{numberMoles}{2}}
            +
            \propcomp{chemicalPotential}{3}
            \diff{\propcomp{numberMoles}{3}}
            +
            \propcomp{chemicalPotential}{4}
            \diff{\propcomp{numberMoles}{4}}
            +
            \propcomp{chemicalPotential}{5}
            \diff{\propcomp{numberMoles}{5}}\\
            &+
            \propcomp{chemicalPotential}{6}
            \diff{\propcomp{numberMoles}{6}}
            +
            \propcomp{chemicalPotential}{7}
            \diff{\propcomp{numberMoles}{7}}
            +
            \propcomp{chemicalPotential}{8}
            \diff{\propcomp{numberMoles}{8}}
            +
            \propcomp{chemicalPotential}{9}
            \diff{\propcomp{numberMoles}{9}}
            +
            \propcomp{chemicalPotential}{10}
            \diff{\propcomp{numberMoles}{10}}
            =
            0\,.
        \end{aligned}
    \end{equation}

    Pela conservação dos elementos químicos \ch{C}, \ch{H}, \ch{O} e \ch{N}
    envolvidos na reação química, deveremos ter à direita da reação o mesmo
    número que à esquerda, isto para cada elemento químico:
    %
    \begin{equation} \label{eq:8.66}
        \begin{aligned}
        \ch{C}&:
        \propcomp{numberMoles}{1}
        +
        \propcomp{numberMoles}{5} - \gls{carbonQuantity}
        =
        0\\
        \ch{H}&:
        2\propcomp{numberMoles}{2}
        +
        2\propcomp{numberMoles}{6}
        +
        \propcomp{numberMoles}{7}
        +
        \propcomp{numberMoles}{9}
        -
        \gls{hidrogenQuantity}
        =
        0 \\
        \ch{O}&:
        2\propcomp{numberMoles}{1}
        +
        \propcomp{numberMoles}{2}
        +
        2\propcomp{numberMoles}{4}
        +
        \propcomp{numberMoles}{5}
        +
        \propcomp{numberMoles}{8}
        +
        \propcomp{numberMoles}{9}
        +
        \propcomp{numberMoles}{10}
        -
        \left(
            \gls{oxygenQuantity}
            +
            2\frac{\gls{molarAirFuelRatio}}{\gls{equivalenceRatio}}
        \right)
        =
        0\\
        \ch{N}&:
        2\propcomp{numberMoles}{3}
        +
        \propcomp{numberMoles}{10}
        -
        \left[
            \gls{nitrogenQuantity}
            +
            2(3,76)\frac{\gls{molarAirFuelRatio}}{\gls{equivalenceRatio}}
        \right]
        =
        0\,.
        \end{aligned}
    \end{equation}

    Claramente não podemos variar independentemente os números de moles dos
    componentes da mistura na \cref{eq:8.65}, pois estão vinculados pelas
    quatro equações de conservação, \cref{eq:8.66}, de forma que a minimização
    da função de Gibbs da mistura deve satisfazer também às restrições,
    \cref{eq:8.66}.


    \section{O Método dos Multiplicadores de Lagrange}

    Dentre as várias maneiras de se resolver este problema, dentre as quais a
    substituição sucessiva, vamos utilizar aqui o método dos multiplicadores de
    Lagrange. Para isso, escrevemos a forma estendida da função
    \gls{GibbsFreeEnergy} a ser minimizada, $\tilde{\gls{GibbsFreeEnergy}}$,
    cujo mínimo deverá ser o mesmo de G, pois cada um dos
    \gls{LagrangeMultiplyer}, conhecidos como os \emph{multiplicadores de
    Lagrange}, é multiplicado pela sua respectiva equação de conservação,
    \cref{eq:8.66}:
    %
    \begin{equation} \label{eq:8.67}
        \begin{aligned}
        \tilde{\gls{GibbsFreeEnergy}}
        =
        \gls{GibbsFreeEnergy}
        +
        &\gls{LagrangeMultiplyer}_{\ch{C}}
        \left(
            \propcomp{numberMoles}{1}
            +
            \propcomp{numberMoles}{5}
            -
            \gls{carbonQuantity}
        \right)
        +\\
        &\gls{LagrangeMultiplyer}_{\ch{H}}
        \left(
            2\propcomp{numberMoles}{2}
            +
            2\propcomp{numberMoles}{6}
            +
            \propcomp{numberMoles}{7}
            +
            \propcomp{numberMoles}{9}
            -
            \gls{hidrogenQuantity}
        \right)
        +\\
        &\gls{LagrangeMultiplyer}_{\ch{O}}
        \left[
            2\propcomp{numberMoles}{1}
            +
            \propcomp{numberMoles}{2}
            +
            2\propcomp{numberMoles}{4}
            +
            \propcomp{numberMoles}{5}
            +
            \propcomp{numberMoles}{8}
            +
            \propcomp{numberMoles}{9}
            +
            \propcomp{numberMoles}{10}
            -
            \left(
                \gls{oxygenQuantity}
                +
                2\frac{\gls{molarAirFuelRatio}}{\gls{equivalenceRatio}}
            \right)
        \right]
        +\\
        &\gls{LagrangeMultiplyer}_{\ch{N}}
        \left[
            2\propcomp{numberMoles}{3}
            +
            \propcomp{numberMoles}{10}
            -
            \left(
                \gls{nitrogenQuantity}
                +
                2(3,76)\frac{\gls{molarAirFuelRatio}}{\gls{equivalenceRatio}}
            \right)
        \right]\,.
        \end{aligned}
    \end{equation}

    Diferenciamos, então, $\tilde{\gls{GibbsFreeEnergy}}$ e substituimos
    \diff{\gls{GibbsFreeEnergy}}, da seguinte forma:
    %
    \begin{equation} \label{eq:8.68}
        \begin{aligned}
        \diff{
            \tilde{\gls{GibbsFreeEnergy}}
        }
        =&\,\,\,
        \propcomp{chemicalPotential}{1}
        \diff{\propcomp{numberMoles}{1}}
        +
        \propcomp{chemicalPotential}{2}
        \diff{\propcomp{numberMoles}{2}}
        +
        \propcomp{chemicalPotential}{3}
        \diff{\propcomp{numberMoles}{3}}
        +
        \propcomp{chemicalPotential}{4}
        \diff{\propcomp{numberMoles}{4}}
        +
        \propcomp{chemicalPotential}{5}
        \diff{\propcomp{numberMoles}{5}}
        +\\
        &\propcomp{chemicalPotential}{6}
        \diff{\propcomp{numberMoles}{6}}
        +
        \propcomp{chemicalPotential}{7}
        \diff{\propcomp{numberMoles}{7}}
        +
        \propcomp{chemicalPotential}{8}
        \diff{\propcomp{numberMoles}{8}}
        +
        \propcomp{chemicalPotential}{9}
        \diff{\propcomp{numberMoles}{9}}
        +
        \propcomp{chemicalPotential}{10}
        \diff{\propcomp{numberMoles}{10}}
        +\\
        &\gls{LagrangeMultiplyer}_{\ch{C}}
        \left(
            \diff{\propcomp{numberMoles}{1}}
            +
            \diff{\propcomp{numberMoles}{5}}
        \right)
        +\\
        &\gls{LagrangeMultiplyer}_{\ch{H}}
        \left(
            2\diff\propcomp{numberMoles}{2}
            +
            2\diff\propcomp{numberMoles}{6}
            +
            \diff\propcomp{numberMoles}{7}
            +
            \diff\propcomp{numberMoles}{9}
        \right)
        +\\
        &\gls{LagrangeMultiplyer}_{\ch{O}}
        \left(
            2\diff\propcomp{numberMoles}{1}
            +
            \diff\propcomp{numberMoles}{2}
            +
            2\diff\propcomp{numberMoles}{4}
            +
            \diff\propcomp{numberMoles}{5}
            +
            \diff\propcomp{numberMoles}{8}
            +
            \diff\propcomp{numberMoles}{9}
            +
            \diff\propcomp{numberMoles}{10}
        \right)
        +\\
        &\gls{LagrangeMultiplyer}_{\ch{N}}
        \left(
            2\diff\propcomp{numberMoles}{3}
            +
            \diff\propcomp{numberMoles}{10}
        \right)
        +\\
        &\diff\gls{LagrangeMultiplyer}_{\ch{C}}
        \left(
            \propcomp{numberMoles}{1}
            +
            \propcomp{numberMoles}{5}
            -
            \gls{carbonQuantity}
        \right)
        +\\
        &\diff\gls{LagrangeMultiplyer}_{\ch{H}}
        \left(
            2\propcomp{numberMoles}{2}
            +
            2\propcomp{numberMoles}{6}
            +
            \propcomp{numberMoles}{7}
            +
            \propcomp{numberMoles}{9}
            -
            \gls{hidrogenQuantity}
        \right)
        +\\
        &\diff\gls{LagrangeMultiplyer}_{\ch{O}}
        \left[
            2\propcomp{numberMoles}{1}
            +
            \propcomp{numberMoles}{2}
            +
            2\propcomp{numberMoles}{4}
            +
            \propcomp{numberMoles}{5}
            +
            \propcomp{numberMoles}{8}
            +
            \propcomp{numberMoles}{9}
            +
            \propcomp{numberMoles}{10}
            -
            \left(
                \gls{oxygenQuantity}
                +
                2\frac{\gls{molarAirFuelRatio}}{\gls{equivalenceRatio}}
            \right)
        \right]
        +\\
        &\diff\gls{LagrangeMultiplyer}_{\ch{N}}
        \left[
            2\propcomp{numberMoles}{3}
            +
            \propcomp{numberMoles}{10}
            -
            \left(
                \gls{nitrogenQuantity}
                +
                2(3,76)\frac{\gls{molarAirFuelRatio}}{\gls{equivalenceRatio}}
            \right)
        \right]\,.
        \end{aligned}
    \end{equation}

    Coletando-se os termos das diferenciais nesta equação, obtemos:
    %
    \begin{equation} \label{eq:8.69}
        \begin{aligned}
        \diff{
            \tilde{\gls{GibbsFreeEnergy}}
        }
        =&\,\,\,
        \left(
            \propcomp{chemicalPotential}{1}
            +
            \gls{LagrangeMultiplyer}_{\ch{C}}
            +
            2\gls{LagrangeMultiplyer}_{\ch{O}}
        \right)
        \diff{\propcomp{numberMoles}{1}}
        +
        \left(
            \propcomp{chemicalPotential}{2}
            +
            2\gls{LagrangeMultiplyer}_{\ch{H}}
            +
            \gls{LagrangeMultiplyer}_{\ch{O}}
        \right)
        \diff{\propcomp{numberMoles}{2}}
        +
        \left(
            \propcomp{chemicalPotential}{3}
            +
            2\gls{LagrangeMultiplyer}_{\ch{N}}
        \right)
        \diff{\propcomp{numberMoles}{3}}
        +\\
        &
        \left(
            \propcomp{chemicalPotential}{4}
            +
            2\gls{LagrangeMultiplyer}_{\ch{O}}
        \right)
        \diff{\propcomp{numberMoles}{4}}
        +
        \left(
            \propcomp{chemicalPotential}{5}
            +
            \gls{LagrangeMultiplyer}_{\ch{C}}
            +
            \gls{LagrangeMultiplyer}_{\ch{O}}
        \right)
        \diff{\propcomp{numberMoles}{5}}
        +
        \left(
            \propcomp{chemicalPotential}{6}
            +
            2\gls{LagrangeMultiplyer}_{\ch{H}}
        \right)
        \diff{\propcomp{numberMoles}{6}}
        +
        \left(
            \propcomp{chemicalPotential}{7}
            +
            \gls{LagrangeMultiplyer}_{\ch{H}}
        \right)
        \diff{\propcomp{numberMoles}{7}}
        +\\
        &\left(
            \propcomp{chemicalPotential}{8}
            +
            \gls{LagrangeMultiplyer}_{\ch{O}}
        \right)
        \diff{\propcomp{numberMoles}{8}}
        +
        \left(
            \propcomp{chemicalPotential}{9}
            +
            \gls{LagrangeMultiplyer}_{\ch{O}}
            +
            \gls{LagrangeMultiplyer}_{\ch{H}}
        \right)
        \diff{\propcomp{numberMoles}{9}}
        +
        \left(
            \propcomp{chemicalPotential}{10}
            +
            \gls{LagrangeMultiplyer}_{\ch{N}}
            +
            \gls{LagrangeMultiplyer}_{\ch{O}}
        \right)
        \diff{\propcomp{numberMoles}{10}}
        +\\
        &
        \left(
            \propcomp{numberMoles}{1}
            +
            \propcomp{numberMoles}{5}
            -
            \gls{carbonQuantity}
        \right)
        \diff\gls{LagrangeMultiplyer}_{\ch{C}}
        +
        \left(
            2\propcomp{numberMoles}{2}
            +
            2\propcomp{numberMoles}{6}
            +
            \propcomp{numberMoles}{7}
            +
            \propcomp{numberMoles}{9}
            -
            \gls{hidrogenQuantity}
        \right)
        \diff\gls{LagrangeMultiplyer}_{\ch{H}}
        +\\
        &
        \left[
            2\propcomp{numberMoles}{1}
            +
            \propcomp{numberMoles}{2}
            +
            2\propcomp{numberMoles}{4}
            +
            \propcomp{numberMoles}{5}
            +
            \propcomp{numberMoles}{8}
            +
            \propcomp{numberMoles}{9}
            +
            \propcomp{numberMoles}{10}
            -
            \left(
                \gls{oxygenQuantity}
                +
                2\frac{\gls{molarAirFuelRatio}}{\gls{equivalenceRatio}}
            \right)
        \right]
        \diff\gls{LagrangeMultiplyer}_{\ch{O}}
        +\\
        &
        \left[
            2\propcomp{numberMoles}{3}
            +
            \propcomp{numberMoles}{10}
            -
            \left(
                \gls{nitrogenQuantity}
                +
                2(3,76)\frac{\gls{molarAirFuelRatio}}{\gls{equivalenceRatio}}
            \right)
        \right]
        \diff\gls{LagrangeMultiplyer}_{\ch{N}}
        =
        0\,.
        \end{aligned}
    \end{equation}

    Uma vez que agora na \cref{eq:8.69} cada diferencial pode variar
    arbitrariamente, para que $\diff{\tilde{\gls{GibbsFreeEnergy}}}$ se anule
    cada um dos respectivos coeficientes deve ser igualado a zero, produzindo
    assim o sistema:
    %
    \begin{equation} \label{eq:8.70}
        \begin{aligned}
            &\propcomp{chemicalPotential}{1}
            +
            \gls{LagrangeMultiplyer}_{\ch{C}}
            +
            2\gls{LagrangeMultiplyer}_{\ch{O}}
            = 0
            \,\,\,\,\,
            \text{(\ch{CO2})}\\
            %
            &\propcomp{chemicalPotential}{2}
            +
            2\gls{LagrangeMultiplyer}_{\ch{H}}
            +
            \gls{LagrangeMultiplyer}_{\ch{O}}
            = 0
            \,\,\,\,\,
            \text{(\ch{H2O})}\\
            %
            &\propcomp{chemicalPotential}{3}
            +
            2\gls{LagrangeMultiplyer}_{\ch{N}}
            = 0
            \,\,\,\,\,
            \text{(\ch{N2})}\\
            %
            &\propcomp{chemicalPotential}{4}
            +
            2\gls{LagrangeMultiplyer}_{\ch{O}}
            = 0
            \,\,\,\,\,
            \text{(\ch{O2})}\\
            %
            &\propcomp{chemicalPotential}{5}
            +
            \gls{LagrangeMultiplyer}_{\ch{C}}
            +
            \gls{LagrangeMultiplyer}_{\ch{O}}
            = 0
            \,\,\,\,\,
            \text{(\ch{CO})}\\
            %
            &\propcomp{chemicalPotential}{6}
            +
            2\gls{LagrangeMultiplyer}_{\ch{H}}
            = 0
            \,\,\,\,\,
            \text{(\ch{H2})}\\
            %
            &\propcomp{chemicalPotential}{7}
            +
            \gls{LagrangeMultiplyer}_{\ch{H}}
            = 0
            \,\,\,\,\,
            \text{(\ch{H})}\\
            %
            &\propcomp{chemicalPotential}{8}
            +
            \gls{LagrangeMultiplyer}_{\ch{O}}
            = 0
            \,\,\,\,\,
            \text{(\ch{O})}\\
            %
            &\propcomp{chemicalPotential}{9}
            +
            \gls{LagrangeMultiplyer}_{\ch{O}}
            +
            \gls{LagrangeMultiplyer}_{\ch{H}}
            = 0
            \,\,\,\,\,
            \text{(\ch{OH})}\\
            %
            &\propcomp{chemicalPotential}{10}
            +
            \gls{LagrangeMultiplyer}_{\ch{N}}
            +
            \gls{LagrangeMultiplyer}_{\ch{O}}
            = 0
            \,\,\,\,\,
            \text{(\ch{NO})}\\
            %
            &\propcomp{numberMoles}{1}
            +
            \propcomp{numberMoles}{5}
            -
            \gls{carbonQuantity}
            = 0\\
            %
            &2\propcomp{numberMoles}{2}
            +
            2\propcomp{numberMoles}{6}
            +
            \propcomp{numberMoles}{7}
            +
            \propcomp{numberMoles}{9}
            -
            \gls{hidrogenQuantity}
            = 0\\
            %
            &2\propcomp{numberMoles}{1}
            +
            \propcomp{numberMoles}{2}
            +
            2\propcomp{numberMoles}{4}
            +
            \propcomp{numberMoles}{5}
            +
            \propcomp{numberMoles}{8}
            +
            \propcomp{numberMoles}{9}
            +
            \propcomp{numberMoles}{10}
            -
            \left(
                \gls{oxygenQuantity}
                +
                2\frac{\gls{molarAirFuelRatio}}{\gls{equivalenceRatio}}
            \right)
            = 0\\
            %
            &2\propcomp{numberMoles}{3}
            +
            \propcomp{numberMoles}{10}
            -
            \left(
                \gls{nitrogenQuantity}
                +
                2(3,76)\frac{\gls{molarAirFuelRatio}}{\gls{equivalenceRatio}}
            \right)
            = 0\\
            %
            &\propcomp{numberMoles}{1}
            +
            \propcomp{numberMoles}{2}
            +
            \propcomp{numberMoles}{3}
            +
            \propcomp{numberMoles}{4}
            +
            \propcomp{numberMoles}{5}
            +
            \propcomp{numberMoles}{6}
            +
            \propcomp{numberMoles}{7}
            +
            \propcomp{numberMoles}{8}
            +
            \propcomp{numberMoles}{9}
            +
            \propcomp{numberMoles}{10}
            +
            \gls{numberMoles} = 0\,,
        \end{aligned}
    \end{equation}
    %
    onde \gls{numberMoles} é o número total de moles nos produtos.

    Por outro lado, cada um dos potenciais químicos pode ser escrito:
    %
    \begin{equation} \label{eq:8.71}
        \propcomp{chemicalPotential}{i}
        \functionOf{
            \gls{temperature},
            \gls{pressure},
            \propcomp{moleFraction}{1},
            \propcomp{moleFraction}{2},
            ...,
            \propcomp{moleFraction}{9}
        }
        =
        \partmolalprop{GibbsFreeEnergy}{i}
        =
        \propcomp{chemicalPotential}{i}^{\gls{standardState}}
        \functionOf{
            \gls{temperature},
            \gls{standardPressure}
        }
        +
        \gls{universalGasConstant}
        \gls{temperature}
        \ln{
            \left(
                \frac{
                    \molalpropcomp{fugacity}{i}
                    \functionOf{
                        \gls{temperature},
                        \gls{pressure},
                        \propcomp{moleFraction}{1},
                        \propcomp{moleFraction}{2},
                        ...,
                        \propcomp{moleFraction}{9}
                    }
                }{
                    \propcomp{fugacity}{i}^{\gls{standardState}}
                    \functionOf{
                        \gls{temperature},
                        \gls{standardPressure}
                    }
                }
            \right)
        }\,,
        \,\,\,\,
        \forall i = 1,2,...,10 \,,
    \end{equation}
    %
    onde
    $\propcomp{chemicalPotential}{i}^{\gls{standardState}}\functionOf{\gls{temperature},
    \gls{standardPressure}}$ é o potencial químico de $i$ puro a
    \gls{temperature} e \gls{standardPressure};
    $\molalpropcomp{fugacity}{i}\functionOf{\gls{temperature}, \gls{pressure},
    \propcomp{moleFraction}{1}, ..., \propcomp{moleFraction}{9}}$ é a
    fugacidade do componente $i$ na mistura a \gls{temperature} e
    \gls{pressure};
    $\propcomp{fugacity}{i}^{\gls{standardState}}\functionOf{\gls{temperature},
    \gls{standardPressure}}$ é a fugacidade do componente $i$ puro a
    \gls{temperature} e \gls{standardPressure}.

    Observe que a temperatura da \cref{eq:8.71} é \gls{temperature} e que a
    pressão \gls{standardPressure} refere-se a um estado padrão arbitrário,
    frequentemente adotado como a pressão do meio-ambiente padrão
    \gsub{pressure}{environmentState} para efeito de tabelas.

    Se cada um dos componentes puros $i$ for assumido como gás perfeito a
    \gls{temperature} e \gls{standardPressure} então a \cref{eq:8.71} ficará:
    %
    \begin{equation} \label{eq:8.72}
        \propcomp{chemicalPotential}{i}
        \functionOf{
            \gls{temperature},
            \gls{pressure},
            \propcomp{moleFraction}{1},
            \propcomp{moleFraction}{2},
            ...,
            \propcomp{moleFraction}{9}
        }
        =
        \partmolalprop{GibbsFreeEnergy}{i}
        =
        {\idealgasprop{chemicalPotential}}_{i}^{\gls{standardState}}
        \functionOf{
            \gls{temperature},
            \gls{standardPressure}
        }
        +
        \gls{universalGasConstant}
        \gls{temperature}
        \ln{
            \left(
                \frac{
                    \molalpropcomp{fugacity}{i}
                    \functionOf{
                        \gls{temperature},
                        \gls{pressure},
                        \propcomp{moleFraction}{1},
                        \propcomp{moleFraction}{2},
                        ...,
                        \propcomp{moleFraction}{9}
                    }
                }{
                    \gls{standardPressure}
                }
            \right)
        }\,,
        \,\,\,\,
        \forall i = 1,2,...,10\,,
    \end{equation}
    %
    pois, como $i$ é um gás perfeito a \gls{temperature} e
    \gls{standardPressure}, então
    $\propcomp{fugacity}{i}^{\gls{standardState}}\functionOf{\gls{temperature},
    \gls{standardPressure}} = \gls{standardPressure}$.

    Como em situações anteriormente encontradas, não poderemos avançar enquanto
    não fornecermos um modelo para a fugacidade de cada componente $i$ na
    mistura a \gls{temperature} e \gls{pressure}. Vamos, então, admitir que a
    mistura nessas condições é uma solução ideal. Aplicando-se a Lei de
    Lewis-Randall, a \cref{eq:8.72} torna-se:
    %
    \begin{equation} \label{eq:8.73}
        \propcomp{chemicalPotential}{i}
        \functionOf{
            \gls{temperature},
            \gls{pressure},
            \propcomp{moleFraction}{1},
            \propcomp{moleFraction}{2},
            ...,
            \propcomp{moleFraction}{9}
        }
        =
        \partmolalprop{GibbsFreeEnergy}{i}
        =
        {\idealgasprop{chemicalPotential}}_{i}^{\gls{standardState}}
        \functionOf{
            \gls{temperature},
            \gls{standardPressure}
        }
        +
        \gls{universalGasConstant}
        \gls{temperature}
        \ln{
            \left(
                \frac{
                    \propcomp{moleFraction}{i}
                    \propcomp{fugacity}{i}
                    \functionOf{
                        \gls{temperature},
                        \gls{pressure}
                    }
                }{
                    \gls{standardPressure}
                }
            \right)
        }\,,
        \,\,\,\,
        \forall i = 1,2,...,10\,,
    \end{equation}
    %
    pois $\molalpropcomp{fugacity}{i}\functionOf{\gls{temperature},
    \gls{pressure}, \propcomp{moleFraction}{1}, ...,
    \propcomp{moleFraction}{9}} =
    \propcomp{moleFraction}{i}\propcomp{fugacity}{i}\functionOf{\gls{temperature},
    \gls{pressure}}$, $\forall i = 1,2,...,10$. Cada uma das frações molares é
    $\propcomp{moleFraction}{i} =
    \dfrac{\propcomp{numberMoles}{i}}{\gls{numberMoles}}$, tal que
    $\sum\limits^{10}_{i = 1}{\propcomp{moleFraction}{i}} = 1$.

    Se, além disso, se a água nos produtos estiver toda ela na forma de vapor e
    a mistura de produtos de combustão a \gls{temperature} e \gls{pressure}
    puder ser considerada como de gases perfeitos, uma hipótese aliás bem
    razoável em processos de combustão comuns, então todos os
    $\propcomp{fugacity}{i}\functionOf{\gls{temperature}, \gls{pressure}} =
    \gls{pressure}$. Com isso, os potenciais químicos ficarão:

    \begin{equation} \label{eq:8.74}
        \propcomp{chemicalPotential}{i}
        =
        \partmolalprop{GibbsFreeEnergy}{i}
        =
        {\idealgasprop{chemicalPotential}}_{i}^{\gls{standardState}}
        \functionOf{
            \gls{temperature},
            \gls{standardPressure}
        }
        +
        \gls{universalGasConstant}
        \gls{temperature}
        \left[
            \ln{
                \frac{
                    \propcomp{numberMoles}{i}
                }{
                    \gls{numberMoles}
                }
            }
            +
            \ln{
                \frac{
                    \gls{pressure}
                }{
                    \gls{standardPressure}
                }
            }
        \right]\,,
        \,\,\,\,
        \forall i = 1,2,...,10\,.
    \end{equation}

    As \cref{eq:8.74} devem ser utilizadas em substituição aos potenciais
    químicos no sistema das \cref{eq:8.70}. Se como no nosso exemplo, forem dez
    os produtos de combustão, teremos uma equação para \gls{molarAirFuelRatio},
    \cref{eq:8.59}, dez equações referentes aos potenciais químicos
    \propcomp{chemicalPotential}{i}, quatro equações da conservação de \ch{C},
    \ch{H}, \ch{O} e \ch{N} e mais uma equação do número total de moles
    \gls{numberMoles}, representando um sistema com dezesseis equações. Quantas
    e quais seriam as incógnitas? Vejamos: Analisando o sistema,
    \cref{eq:8.70}, associado às \cref{eq:8.74}, descobrimos o conjunto de
    variáveis:
    %
    \begin{equation*}
        \gls{carbonQuantity},
        \gls{hidrogenQuantity},
        \gls{oxygenQuantity},
        \gls{nitrogenQuantity},
        \gls{equivalenceRatio},
        \gls{molarAirFuelRatio},
        \propcomp{numberMoles}{1},\propcomp{numberMoles}{2},
        \propcomp{numberMoles}{3},\propcomp{numberMoles}{4},
        \propcomp{numberMoles}{5},\propcomp{numberMoles}{6},
        \propcomp{numberMoles}{7},\propcomp{numberMoles}{8},
        \propcomp{numberMoles}{9},\propcomp{numberMoles}{10},
        \gls{LagrangeMultiplyer}_{\ch{C}},
        \gls{LagrangeMultiplyer}_{\ch{H}},
        \gls{LagrangeMultiplyer}_{\ch{O}},\gls{LagrangeMultiplyer}_{\ch{N}},
        \gls{numberMoles},\gls{temperature},\gls{pressure}\,,
    \end{equation*}
    %
    o que perfaz um total de  vinte e três incógnitas, sete a mais do que o
    número de equações no sistema.  Portanto, para que este tenha uma solução
    determinada, ou são necessárias mais sete equações, ou então que sete das
    variáveis sejam conhecidas de antemão. Podemos afirmar de um modo geral que
    as sete equações-extra ou os meios para a sua obtenção devem provir dos
    dados e especificações estabelecidas para cada problema.

    Deixamos para discutir à parte a determinação dos valores para
    ${\molalidealgasprop{chemicalPotential}}_{i}^{\gls{standardState}}\functionOf{\gls{temperature},\gls{standardPressure}}
    =
    {\molalidealgasprop{intGibbsFreeEnergy}}_i^{\gls{standardPressure}}\functionOf{\gls{temperature},
    \gls{standardPressure}}$, nas \cref{eq:8.71,eq:8.72,eq:8.73}, assumindo-se
    gases perfeitos a cada um dos componentes puros $i$ na temperatura
    \gls{temperature} e pressão padrão \gls{standardPressure}.


    \section{A Determinação dos Potenciais Químicos}

    Vimos que na expressão da minimização da Função de Gibbs de uma mistura de
    \gls{numberSpecies} componentes em equilíbrio químico a \gls{temperature} e
    \gls{pressure}, aparece a função de Gibbs parcial molar de cada componente
    $i$ nas condições da mistura, ou seja, $\partmolalprop{GibbsFreeEnergy}{i}
    = \propcomp{chemicalPotential}{i}\functionOf{\gls{temperature},
    \gls{pressure}, \propcomp{moleFraction}{j \neq i}}$ também chamado de
    potencial químico do componente $i$ na mistura a \gls{temperature} e
    \gls{pressure}.

    Mas como determinamos os valores destes potenciais?

    Vamos recuperar o que vimos a respeito, agora de forma geral:
    %
    \begin{equation*}
        \propcomp{chemicalPotential}{i}
        \functionOf{
            \gls{temperature},
            \gls{pressure},
            \propcomp{moleFraction}{1},
            \propcomp{moleFraction}{2},
            ...,
            \propcomp{moleFraction}{9}
        }
        =
        \partmolalprop{GibbsFreeEnergy}{i}
        =
        \propcomp{chemicalPotential}{i}^{\gls{standardState}}
        \functionOf{
            \gls{temperature},
            \gls{standardPressure}
        }
        +
        \gls{universalGasConstant}
        \gls{temperature}
        \ln{
            \left(
                \frac{
                    \molalpropcomp{fugacity}{i}
                    \functionOf{
                        \gls{temperature},
                        \gls{pressure},
                        \propcomp{moleFraction}{1},
                        \propcomp{moleFraction}{2},
                        ...,
                        \propcomp{moleFraction}{9}
                    }
                }{
                    \propcomp{fugacity}{i}^{\gls{standardState}}
                    \functionOf{
                        \gls{temperature},
                        \gls{standardPressure}
                    }
                }
            \right)
        }\,,
        \,\,\,\,
        \forall i = 1,2,...,10 \,.
    \end{equation*}

    Se $i$ puro for gás perfeito a \gls{temperature} e \gls{standardPressure}:
    %
    \begin{equation}
        \propcomp{chemicalPotential}{i}
        \functionOf{
            \gls{temperature},
            \gls{pressure},
            \propcomp{moleFraction}{1},
            \propcomp{moleFraction}{2},
            ...,
            \propcomp{moleFraction}{9}
        }
        =
        \partmolalprop{GibbsFreeEnergy}{i}
        =
        {\idealgasprop{chemicalPotential}}_{i}^{\gls{standardState}}
        \functionOf{
            \gls{temperature},
            \gls{standardPressure}
        }
        +
        \gls{universalGasConstant}
        \gls{temperature}
        \ln{
            \left(
                \frac{
                    \molalpropcomp{fugacity}{i}
                    \functionOf{
                        \gls{temperature},
                        \gls{pressure},
                        \propcomp{moleFraction}{1},
                        \propcomp{moleFraction}{2},
                        ...,
                        \propcomp{moleFraction}{9}
                    }
                }{
                    \gls{standardPressure}
                }
            \right)
        }\,,
        \,\,\,\,
        \forall i = 1,2,...,10\,.
    \end{equation}

    Por outro lado, se em \gls{temperature} e \gls{pressure} a mistura
    homogênea for solução ideal:
    %
    \begin{equation*}
        \propcomp{chemicalPotential}{i}
        \functionOf{
            \gls{temperature},
            \gls{pressure},
            \propcomp{moleFraction}{1},
            \propcomp{moleFraction}{2},
            ...,
            \propcomp{moleFraction}{9}
        }
        =
        \partmolalprop{GibbsFreeEnergy}{i}
        =
        {\idealgasprop{chemicalPotential}}_{i}^{\gls{standardState}}
        \functionOf{
            \gls{temperature},
            \gls{standardPressure}
        }
        +
        \gls{universalGasConstant}
        \gls{temperature}
        \ln{
            \left(
                \frac{
                    \propcomp{moleFraction}{i}
                    \propcomp{fugacity}{i}
                    \functionOf{
                        \gls{temperature},
                        \gls{pressure}
                    }
                }{
                    \gls{standardPressure}
                }
            \right)
        }\,,
        \,\,\,\,
        \forall i = 1,2,...,10\,.
    \end{equation*}

    Ainda, se a solução ideal for uma mistura de gases perfeitos:
    %
    \begin{equation*}
        \propcomp{chemicalPotential}{i}
        \functionOf{
            \gls{temperature},
            \gls{pressure},
            \propcomp{moleFraction}{1},
            \propcomp{moleFraction}{2},
            ...,
            \propcomp{moleFraction}{9}
        }
        =
        \partmolalprop{GibbsFreeEnergy}{i}
        =
        {\idealgasprop{chemicalPotential}}_{i}^{\gls{standardState}}
        \functionOf{
            \gls{temperature},
            \gls{standardPressure}
        }
        +
        \gls{universalGasConstant}
        \gls{temperature}
        \ln{
            \left(
                \frac{
                    \propcomp{moleFraction}{i}
                    \gls{pressure}
                }{
                    \gls{standardPressure}
                }
            \right)
        }\,,
        \,\,\,\,
        \forall i = 1,2,...,10\,.
    \end{equation*}

    Muito bem, até agora, mas como determinarmos o valor de
    ${\molalidealgasprop{intGibbsFreeEnergy}}^{\gls{standardTemperature}}_{i}\functionOf{\gls{temperature},
    \gls{pressure}}$?

    Se estivermos lidando com reações químicas, já sabemos que todas as
    propriedades deverão ser trazidas para uma mesma base, comum a todos os
    integrantes da reação. Neste caso, por conveniência, usaremos a base da
    função de Gibbs de formação, \gls{formationGibbsFreeEnergy} cuja definição
    é idêntica a da base da entalpia de formação \gls{formationEnthalpy}, ou
    seja, onde se nesta lê entalpia substitua-se naquela por função de Gibbs.

    Podemos expressar a função de Gibbs de qualquer substância pura $i$ a
    \gls{temperature}, \gls{pressure} por uma expressão análoga a que
    desenvolvemos para a entalpia:

    \begin{equation} \label{eq:8.75}
        \molalpropcomp{intGibbsFreeEnergy}{i}
        \functionOf{
            \gls{temperature},
            \gls{pressure}
        }
        =
        \propcomp{formationGibbsFreeEnergy}{i}
        +
        \underset{
            \gls{standardTemperature},
            \gls{standardPressure}
            \rightarrow
            \gls{temperature},
            \gls{pressure}
        }{
            \Delta\molalpropcomp{intGibbsFreeEnergy}{i}
        }{}\,.
    \end{equation}

    Nesta última equação, o termo da esquerda que queremos determinar e o
    primeiro termo da direita estão ambos na base da função de Gibbs de
    formação. O último termo da direita poderá estar, como sabemos, em uma base
    qualquer. O valor de \propcomp{formationGibbsFreeEnergy}{i}, por sua vez,
    pode ser obtido, quando possível, das tabelas termodinâmicas.  Muitas vezes
    $i$ é assumido como gás perfeito em \gls{standardTemperature},
    \gls{standardPressure}, o que, como já discutimos, não implica em perda de
    generalidade. Quanto aos demais termos da direita, podemos escrever:
    %
    \begin{equation} \label{eq:8.76}
        \underset{
            \gls{standardTemperature},
            \gls{standardPressure}
            \rightarrow
            \gls{temperature},
            \gls{pressure}
        }{
            \Delta\molalpropcomp{intGibbsFreeEnergy}{i}
        }{}
        =
        \molalpropcomp{intGibbsFreeEnergy}{i}
        \functionOf{
            \gls{temperature},
            \gls{pressure}
        }
        -
        \molalpropcomp{intGibbsFreeEnergy}{i}
        \functionOf{
            \gls{standardTemperature},
            \gls{standardPressure}
        }
        =
        \left[
            \molar{\gls{intEnthalpy}}
            \functionOf{
                \gls{standardTemperature}
            }
            -
            \molar{\gls{intEnthalpy}}
            \functionOf{
                \gls{temperature}
            }
        \right]_{i}
        -
        \gls{temperature}
        \molalpropcomp{intEntropy}{i}
        \functionOf{
            \gls{temperature},
            \gls{pressure}
        }
        +
        \gls{standardTemperature}
        \molalpropcomp{intEntropy}{i}
        \functionOf{
            \gls{standardTemperature},
            \gls{standardPressure}
        }\,.
    \end{equation}

    Para escrever esta equação, lançamos mão da definição de função de Gibbs
    para substância pura $\gls{intGibbsFreeEnergy} = \gls{intEnthalpy} -
    \gls{temperature}\gls{intEntropy}$. A diferença de entalpias, bem como as
    entropias, podem ser obtidas como já estudamos. Um caso interessante e
    importante é quando a base é adotada como gás perfeito. Então,
    substituindo-se nas \cref{eq:8.75,eq:8.76} os termos que devem ser gases
    perfeitos, a expressão completa para a função de Gibbs molar (ou o
    potencial químico) de uma substância pura $i$ a \gls{temperature},
    \gls{pressure} será dada por
    %
    \begin{equation} \label{eq:8.76}
        \begin{aligned}
        \molalpropcomp{intGibbsFreeEnergy}{i}
        \functionOf{
            \gls{temperature},
            \gls{pressure}
        }
        =
        \propcomp{chemicalPotential}{i}
        \functionOf{
            \gls{temperature},
            \gls{pressure}
        }
        &=
        \propcomp{formationGibbsFreeEnergy}{i}
        +
        \left[
            \molalpropcomp{intGibbsFreeEnergy}{i}
            \functionOf{
                \gls{temperature},
                \gls{pressure}
            }
            -
            \molalidealgasprop{intGibbsFreeEnergy}_{i}
            \functionOf{
                \gls{standardTemperature},
                \gls{standardPressure}
            }
        \right]\\
        &=
        \propcomp{formationGibbsFreeEnergy}{i}
        +
        \left[
            \molar{\gls{intEnthalpy}}
            \functionOf{
                \gls{temperature}
            }
            -
            \molalidealgasprop{intEnthalpy}
            \functionOf{
                \gls{standardTemperature}
            }
        \right]_{i}
        -
        \gls{temperature}
        \molalpropcomp{intEntropy}{i}
        \functionOf{
            \gls{temperature},
            \gls{pressure}
        }
        +
        \gls{standardTemperature}
        \molalidealgasprop{intEntropy}_{i}
        \functionOf{
            \gls{standardTemperature},
            \gls{standardPressure}
        }\,.
        \end{aligned}
    \end{equation}

    No caso de nossa reação química, precisávamos determinar o valor de
    ${\molalidealgasprop{chemicalPotential}}_{i}^{\gls{standardState}}\functionOf{\gls{temperature},
    \gls{standardPressure}} =
    {\molalidealgasprop{intGibbsFreeEnergy}}^{\gls{standardTemperature}}_{i}\functionOf{\gls{temperature},
    \gls{standardPressure}}$. Pois bem, basta substituirmos adequadamente na
    \cref{eq:8.76}:
    %
    \begin{equation} \label{eq:8.77}
        \begin{aligned}
        {\molalidealgasprop{intGibbsFreeEnergy}}_{i}^{\gls{standardState}}
        \functionOf{
            \gls{temperature},
            \gls{standardPressure}
        }
        =
        {\idealgasprop{chemicalPotential}}_{i}^{\gls{standardState}}
        \functionOf{
            \gls{temperature},
            \gls{standardPressure}
        }
        &=
        \propcomp{formationGibbsFreeEnergy}{i}
        +
        \left[
            \molalidealgasprop{intGibbsFreeEnergy}_{i}
            \functionOf{
                \gls{temperature},
                \gls{standardPressure}
            }
            -
            \molalidealgasprop{intGibbsFreeEnergy}_{i}
            \functionOf{
                \gls{standardTemperature},
                \gls{standardPressure}
            }
        \right]\\
        &=
        \propcomp{formationGibbsFreeEnergy}{i}
        +
        \left[
            \molalidealgasprop{intEnthalpy}
            \functionOf{
                \gls{temperature}
            }
            -
            \molalidealgasprop{intEnthalpy}
            \functionOf{
                \gls{standardTemperature}
            }
        \right]_{i}
        -
        \gls{temperature}
        \molalidealgasprop{intEntropy}_{i}
        \functionOf{
            \gls{temperature},
            \gls{standardPressure}
        }
        +
        \gls{standardTemperature}
        \molalidealgasprop{intEntropy}_{i}
        \functionOf{
            \gls{standardTemperature},
            \gls{standardPressure}
        }\,.
        \end{aligned}
    \end{equation}

    A \cref{eq:8.77} é claramente muito conveniente para o usuário, pois
    envolve apenas valores do gás perfeito para a substância pura $i$.

    A estas alturas você já deve estar dominando com tranquilidade todas estas
    transformações, não é verdade? Como ficaria a Eq. 8.76 com todas as
    expansões em coordenadas generalizadas e nas bases adequadas? Com algum
    cuidado e atenção (faça!), você deve chegar a

    \begin{equation} \label{eq:8.78}
        \begin{aligned}
        \molalpropcomp{intGibbsFreeEnergy}{i}
        \functionOf{
            \gls{temperature},
            \gls{pressure}
        }
        =
        \propcomp{chemicalPotential}{i}
        \functionOf{
            \gls{temperature},
            \gls{pressure}
        }
        &=
        \propcomp{formationGibbsFreeEnergy}{i}
        -
        \left(
            \molalidealgasprop{intEnthalpy}
            -
            \molar{\gls{intEnthalpy}}
        \right)_{i}
        \functionOf{
            \gsub{temperature}{reduced},
            \gsub{pressure}{reduced},
            \gls{PitzerAcentricFactor},
            \gls{WuPolarFactor}
        }
        +
        \left[
            \molalidealgasprop{intEnthalpy}
            \functionOf{
                \gls{temperature}
            }
            -
            \molalidealgasprop{intEnthalpy}
            \functionOf{
                \gls{standardTemperature}
            }
        \right]_{i}
        +\\
        &\gls{temperature}
        \left[
            \left(
                \molalidealgasprop{intEntropy}
                -
                \molar{\gls{intEntropy}}
            \right)_{i}
            \functionOf{
                \gsub{temperature}{reduced},
                \gsub{pressure}{reduced},
                \gls{PitzerAcentricFactor},
                \gls{WuPolarFactor}
            }
            -
            \left[
                \molalidealgasprop{intEntropy}
                \functionOf{
                    \gls{temperature},
                    \gls{pressure}
                }
                -
                \molalidealgasprop{intEntropy}
                \functionOf{
                    \gls{standardTemperature},
                    \gls{standardPressure}
                }
            \right]_{i}
        \right]
        +
        \left(
            \gls{standardTemperature}
            -
            \gls{temperature}
        \right)
        \propcomp{absoluteEntropy}{i}\,.
        \end{aligned}
    \end{equation}
    %
    na qual $\propcomp{absoluteEntropy}{i}$ é o valor da entropia
    absoluta da substância pura $i$ como gás perfeito no estado padrão a
    $(\gls{standardTemperature},\gls{standardPressure})$. Como ficaria a
    \cref{eq:8.78} se a substância pura $i$ fosse gás perfeito também a
    $(\gls{temperature}, \gls{pressure})$? E como ficaria a \cref{eq:8.77},
    que, como você percebeu, é avaliada inteiramente a \gls{standardPressure}?


    \section{O Método das Constantes de Equilíbrio}

    Vamos repetir aqui a reação química dada pela \cref{eq:8.62}:

    \begin{equation} \label{eq:8.79}
        \begin{aligned}
        \ch{
            &\gls{chemicalGenericFuel}
            +
            $\frac{\gls{molarAirFuelRatio}}{\gls{equivalenceRatio}}$
            (
                O2
                +
                3,76 N2
            )
            ->\\
            &\propcomp{numberMoles}{1} CO2
            +
            \propcomp{numberMoles}{2} H2O
            +
            \propcomp{numberMoles}{3} N2
            +
            \propcomp{numberMoles}{4} O2
            +
            \propcomp{numberMoles}{5} CO
            +
            \propcomp{numberMoles}{6} H2
            +
            \propcomp{numberMoles}{7} H
            +
            \propcomp{numberMoles}{8} O
            +
            \propcomp{numberMoles}{9} OH
            +
            \propcomp{numberMoles}{10} NO
        }\,.
        \end{aligned}
    \end{equation}

    Vimos também que o equilíbrio químico da mistura homogênea dos produtos de
    combustão será dado pela minimização da função de Gibbs da mistura a
    \gls{temperature}, \gls{pressure}:

    \begin{equation} \label{eq:8.80}
        \begin{aligned}
            \diff{\gls{GibbsFreeEnergy}}
            &=
            \propcomp{chemicalPotential}{1}
            \diff{\propcomp{numberMoles}{1}}
            +
            \propcomp{chemicalPotential}{2}
            \diff{\propcomp{numberMoles}{2}}
            +
            \propcomp{chemicalPotential}{3}
            \diff{\propcomp{numberMoles}{3}}
            +
            \propcomp{chemicalPotential}{4}
            \diff{\propcomp{numberMoles}{4}}
            +
            \propcomp{chemicalPotential}{5}
            \diff{\propcomp{numberMoles}{5}}\\
            &+
            \propcomp{chemicalPotential}{6}
            \diff{\propcomp{numberMoles}{6}}
            +
            \propcomp{chemicalPotential}{7}
            \diff{\propcomp{numberMoles}{7}}
            +
            \propcomp{chemicalPotential}{8}
            \diff{\propcomp{numberMoles}{8}}
            +
            \propcomp{chemicalPotential}{9}
            \diff{\propcomp{numberMoles}{9}}
            +
            \propcomp{chemicalPotential}{10}
            \diff{\propcomp{numberMoles}{10}}
            =
            0\,,
        \end{aligned}
    \end{equation}
    %
    sujeita esta às restrições estabelecidas pela conservação dos elementos
    químicos, no nosso caso \ch{C}, \ch{H}, \ch{O} e \ch{N}. Vimos em uma seção
    anterior a incorporação (isto é, a satisfação) dessas restrições utilizando
    o método dos multiplicadores de Lagrange. Havíamos comentado a respeito de
    outros métodos para isso.  Dentre eles, nessa seção apresentaremos \emph{o
        método das constantes de equilíbrio}.

    Vamos então expressar a produção dos dez produtos de combustão \ch{CO2},
    \ch{H2O}, \ch{N2}, \ch{O2}, \ch{CO}, \ch{H2}, \ch{H}, \ch{O}, \ch{OH},
    \ch{NO}, na forma das seguintes seis reações químicas:
    %
    \begin{equation} \label{eq:8.81}
        \begin{aligned}
            &\ch{1/2 H2 <-> H} \\
            &\ch{1/2 O2 <-> O} \\
            &\ch{1/2 H2 + 1/2 O2 <-> OH} \\
            &\ch{1/2 N2 + 1/2 O2 <-> NO} \\
            &\ch{H2 + 1/2 O2 <-> H2O} \\
            &\ch{CO + 1/2 O2 <-> CO2}\,.
        \end{aligned}
    \end{equation}

    Perceba que todos os produtos de combustão estão representados nestas seis
    equações e poderíamos também demonstrar que quaisquer outras reações seriam
    necessariamente combinações lineares dessas seis. Pois bem, como vimos
    anteriormente a cada uma destas reações podemos atribuir um grau de reação
    $\gls{reactionExtent}_k$, $\forall k = 1,...,6$. A variação total do número
    de moles \diff{\propcomp{numberMoles}{i}}, $\forall i = 1,...,10$ para cada
    um dos produtos pode ser descrita em função da proporção do avanço da
    variação do grau de reação (ou seja, do deslocamento de cada reação para a
    direita na direção dos \enquote{produtos}) de cada uma das reações nas
    quais cada um dos produtos aparecem, da seguinte forma:
    %
    \begin{equation} \label{eq:8.82}
    \begin{aligned}
        (\ch{CO2})\,\,\,\,\,
        \diff{\propcomp{numberMoles}{1}}
        &=
        \diff{\propcomp{reactionExtent}{6}}\\
        %
        (\ch{H2O})\,\,\,\,\,
        \diff{\propcomp{numberMoles}{2}}
        &=
        \diff{\propcomp{reactionExtent}{5}}\\
        %
        (\ch{N2})\,\,\,\,\,
        \diff{\propcomp{numberMoles}{3}}
        &=
        -\frac{1}{2}\diff{\propcomp{reactionExtent}{4}}\\
        %
        (\ch{O2})\,\,\,\,\,
        \diff{\propcomp{numberMoles}{4}}
        &=
        -\frac{1}{2}\diff{\propcomp{reactionExtent}{2}}
        -\frac{1}{2}\diff{\propcomp{reactionExtent}{3}}
        -\frac{1}{2}\diff{\propcomp{reactionExtent}{4}}
        -\frac{1}{2}\diff{\propcomp{reactionExtent}{5}}
        -\frac{1}{2}\diff{\propcomp{reactionExtent}{6}}
        \\
        %
        (\ch{CO})\,\,\,\,\,
        \diff{\propcomp{numberMoles}{5}}
        &=
        -\diff{\propcomp{reactionExtent}{6}}\\
        %
        (\ch{H2})\,\,\,\,\,
        \diff{\propcomp{numberMoles}{6}}
        &=
        -\frac{1}{2}\diff{\propcomp{reactionExtent}{1}}
        -\frac{1}{2}\diff{\propcomp{reactionExtent}{3}}
        -\diff{\propcomp{reactionExtent}{5}}
        \\
        %
        (\ch{H})\,\,\,\,\,
        \diff{\propcomp{numberMoles}{7}}
        &=
        \diff{\propcomp{reactionExtent}{1}}\\
        %
        (\ch{O})\,\,\,\,\,
        \diff{\propcomp{numberMoles}{8}}
        &=
        \diff{\propcomp{reactionExtent}{2}}\\
        %
        (\ch{OH})\,\,\,\,\,
        \diff{\propcomp{numberMoles}{9}}
        &=
        \diff{\propcomp{reactionExtent}{3}}\\
        %
        (\ch{NO})\,\,\,\,\,
        \diff{\propcomp{numberMoles}{10}}
        &=
        \diff{\propcomp{reactionExtent}{4}}\,.\\
        %
    \end{aligned}
    \end{equation}

    Você deve ter percebido que o objetivo das relações, \cref{eq:8.82}, é fazer com
    que as variações dos números de moles dos produtos satisfaçam as restrições
    impostas pela conservação dos elementos químicos \ch{C}, \ch{H}, \ch{O} e
    \ch{N}. Portanto,
    já deve ter concluído que o próximo passo é substituirmos as variações dos
    números de moles dos produtos na \cref{eq:8.80}:

    \begin{equation} \label{eq:8.83}
        \begin{aligned}
            \diff{\gls{GibbsFreeEnergy}}
            =&\,\,\,
            \propcomp{chemicalPotential}{1}
            \diff{\propcomp{reactionExtent}{6}}
            +
            \propcomp{chemicalPotential}{2}
            \diff{\propcomp{reactionExtent}{5}}
            -
            \propcomp{chemicalPotential}{3}
            \frac{1}{2}\diff{\propcomp{reactionExtent}{4}}\\
            &+
            \propcomp{chemicalPotential}{4}
            \left(
                -\frac{1}{2}\diff{\propcomp{reactionExtent}{2}}
                -\frac{1}{2}\diff{\propcomp{reactionExtent}{3}}
                -\frac{1}{2}\diff{\propcomp{reactionExtent}{4}}
                -\frac{1}{2}\diff{\propcomp{reactionExtent}{5}}
                -\frac{1}{2}\diff{\propcomp{reactionExtent}{6}}
            \right)
            -
            \propcomp{chemicalPotential}{5}
            \diff{\propcomp{reactionExtent}{6}}\\
            &+
            \propcomp{chemicalPotential}{6}
            \left(
                -\frac{1}{2}\diff{\propcomp{reactionExtent}{1}}
                -\frac{1}{2}\diff{\propcomp{reactionExtent}{3}}
                -\diff{\propcomp{reactionExtent}{5}}
            \right)
            +
            \propcomp{chemicalPotential}{7}
            \diff{\propcomp{reactionExtent}{1}}
            +
            \propcomp{chemicalPotential}{8}
            \diff{\propcomp{reactionExtent}{2}}
            +
            \propcomp{chemicalPotential}{9}
            \diff{\propcomp{reactionExtent}{3}}
            +
            \propcomp{chemicalPotential}{10}
            \diff{\propcomp{reactionExtent}{4}}\,.
        \end{aligned}
    \end{equation}

    Colocando as diferenciais dos graus de reação em evidência e igualando a
    zero, obtemos:
    %
    \begin{equation} \label{eq:8.84}
        \begin{aligned}
            \diff{\gls{GibbsFreeEnergy}}
            =&\,\,\,
            \left(
                \propcomp{chemicalPotential}{7}
                -
                \frac{1}{2}\propcomp{chemicalPotential}{6}
            \right)
            \diff{\propcomp{reactionExtent}{1}}
            +
            \left(
                \propcomp{chemicalPotential}{8}
                -
                \frac{1}{2}\propcomp{chemicalPotential}{4}
            \right)
            \diff{\propcomp{reactionExtent}{2}}\\
            &+
            \left(
                \propcomp{chemicalPotential}{9}
                -
                \frac{1}{2}\propcomp{chemicalPotential}{6}
                -
                \frac{1}{2}\propcomp{chemicalPotential}{4}
            \right)
            \diff{\propcomp{reactionExtent}{3}}
            +
            \left(
                \propcomp{chemicalPotential}{10}
                -
                \frac{1}{2}\propcomp{chemicalPotential}{3}
                -
                \frac{1}{2}\propcomp{chemicalPotential}{4}
            \right)
            \diff{\propcomp{reactionExtent}{4}}\\
            &+
            \left(
                \propcomp{chemicalPotential}{2}
                -
                \propcomp{chemicalPotential}{6}
                -
                \frac{1}{2}\propcomp{chemicalPotential}{4}
            \right)
            \diff{\propcomp{reactionExtent}{5}}
            +
            \left(
                \propcomp{chemicalPotential}{1}
                -
                \propcomp{chemicalPotential}{5}
                -
                \frac{1}{2}\propcomp{chemicalPotential}{4}
            \right)
            \diff{\propcomp{reactionExtent}{6}}
            =
            0\,.
        \end{aligned}
    \end{equation}

    Uma vez que todas as variações dos graus de reação na \cref{eq:8.84} são
    arbitrárias, cada um dos seus coeficientes deve ser zero, para que ela seja
    satisfeita. Deve ficar claro que, uma vez alcançado o equilíbrio a
    \gls{temperature}, \gls{pressure}, a composição da mistura não se alterará
    mais, de forma que tentativas de deslocamento de quaisquer das seis
    reações, \cref{eq:8.81}, para a esquerda ou para a direita implicará
    imediatamente aumento da função de Gibbs total da mistura e
    consequentemente haverá um movimento de restituição no sentido contrário
    até se restabelecer o equilíbrio.

    Repetindo aqui a \cref{eq:8.71}, mas com \gls{numberSpecies} componentes:
    %
    \begin{equation*}
        \propcomp{chemicalPotential}{i}
        \functionOf{
            \gls{temperature},
            \gls{pressure},
            \propcomp{moleFraction}{j \neq i}
        }
        =
        \partmolalprop{GibbsFreeEnergy}{i}
        =
        \propcomp{chemicalPotential}{i}^{\gls{standardState}}
        \functionOf{
            \gls{temperature},
            \gls{standardPressure}
        }
        +
        \gls{universalGasConstant}
        \gls{temperature}
        \ln{
            \left(
                \frac{
                    \molalpropcomp{fugacity}{i}
                    \functionOf{
                        \gls{temperature},
                        \gls{pressure},
                        \propcomp{moleFraction}{j \neq i}
                    }
                }{
                    \propcomp{fugacity}{i}^{\gls{standardState}}
                    \functionOf{
                        \gls{temperature},
                        \gls{standardPressure}
                    }
                }
            \right)
        }\,,
        \,\,\,\,
        \forall i = 1,2,...,\gls{numberSpecies} \,,
    \end{equation*}

    Para simplificar a notação, vamos definir a \emph{atividade do componente
    $i$} em qualquer mistura homogênea de \gls{numberSpecies} componentes a
    \gls{temperature}, \gls{pressure} em relação ao estado padrão a
    \gls{temperature},\gls{standardPressure} como:
    %
    \begin{equation} \label{eq:8.85}
        \propcomp{activity}{i}
        \equiv
        \frac{
            \molalpropcomp{fugacity}{i}
        }{
            \propcomp{fugacity}{i}^{\gls{standardState}}
            \functionOf{
                \gls{temperature},
                \gls{standardPressure}
            }
        }\,,
        \,\,\,\,
        \forall i = 1,2,...,\gls{numberSpecies}
    \end{equation}
    %
    e a \cref{eq:8.71} pode ser rescrita como
    %
    \begin{equation} \label{eq:8.86}
        \propcomp{chemicalPotential}{i}
        =
        \partmolalprop{GibbsFreeEnergy}{i}
        =
        \molalpropcomp{intGibbsFreeEnergy}{i}^{\gls{standardState}}
        \functionOf{
            \gls{temperature},
            \gls{standardPressure}
        }
        +
        \gls{universalGasConstant}
        \gls{temperature}
        \ln{
            \propcomp{activity}{i}
        }\,,
        \,\,\,\,
        \forall i = 1,2,...,\gls{numberSpecies} \,,
    \end{equation}

    Substituimos os potenciais químicos dados pelas \cref{eq:8.86} na
    \cref{eq:8.84} e obtemos:
    %
    \begin{equation} \label{eq:8.87}
        \begin{aligned}
            \diff{\gls{GibbsFreeEnergy}}
            =&\,\,\,\,\,\,
            \left[
                \left(
                    \molalpropcomp{intGibbsFreeEnergy}{7}^{\gls{standardState}}
                    +
                    \gls{universalGasConstant}
                    \gls{temperature}
                    \ln{
                        \propcomp{activity}{7}
                    }
                \right)
                -
                \frac{1}{2}
                \left(
                    \molalpropcomp{intGibbsFreeEnergy}{6}^{\gls{standardState}}
                    +
                    \gls{universalGasConstant}
                    \gls{temperature}
                    \ln{
                        \propcomp{activity}{6}
                    }
                \right)
            \right]
            \diff{\propcomp{reactionExtent}{1}}\\
            &+
            \left[
                \left(
                    \molalpropcomp{intGibbsFreeEnergy}{8}^{\gls{standardState}}
                    +
                    \gls{universalGasConstant}
                    \gls{temperature}
                    \ln{
                        \propcomp{activity}{8}
                    }
                \right)
                -
                \frac{1}{2}
                \left(
                    \molalpropcomp{intGibbsFreeEnergy}{4}^{\gls{standardState}}
                    +
                    \gls{universalGasConstant}
                    \gls{temperature}
                    \ln{
                        \propcomp{activity}{4}
                    }
                \right)
            \right]
            \diff{\propcomp{reactionExtent}{2}}\\
            &+
            \left[
                \left(
                    \molalpropcomp{intGibbsFreeEnergy}{9}^{\gls{standardState}}
                    +
                    \gls{universalGasConstant}
                    \gls{temperature}
                    \ln{
                        \propcomp{activity}{9}
                    }
                \right)
                -
                \frac{1}{2}
                \left(
                    \molalpropcomp{intGibbsFreeEnergy}{6}^{\gls{standardState}}
                    +
                    \gls{universalGasConstant}
                    \gls{temperature}
                    \ln{
                        \propcomp{activity}{6}
                    }
                \right)
                -
                \frac{1}{2}
                \left(
                    \molalpropcomp{intGibbsFreeEnergy}{4}^{\gls{standardState}}
                    +
                    \gls{universalGasConstant}
                    \gls{temperature}
                    \ln{
                        \propcomp{activity}{4}
                    }
                \right)
            \right]
            \diff{\propcomp{reactionExtent}{3}}\\
            &+
            \left[
                \left(
                    \molalpropcomp{intGibbsFreeEnergy}{10}^{\gls{standardState}}
                    +
                    \gls{universalGasConstant}
                    \gls{temperature}
                    \ln{
                        \propcomp{activity}{10}
                    }
                \right)
                -
                \frac{1}{2}
                \left(
                    \molalpropcomp{intGibbsFreeEnergy}{3}^{\gls{standardState}}
                    +
                    \gls{universalGasConstant}
                    \gls{temperature}
                    \ln{
                        \propcomp{activity}{3}
                    }
                \right)
                -
                \frac{1}{2}
                \left(
                    \molalpropcomp{intGibbsFreeEnergy}{4}^{\gls{standardState}}
                    +
                    \gls{universalGasConstant}
                    \gls{temperature}
                    \ln{
                        \propcomp{activity}{4}
                    }
                \right)
            \right]
            \diff{\propcomp{reactionExtent}{4}}\\
            &+
            \left[
                \left(
                    \molalpropcomp{intGibbsFreeEnergy}{2}^{\gls{standardState}}
                    +
                    \gls{universalGasConstant}
                    \gls{temperature}
                    \ln{
                        \propcomp{activity}{2}
                    }
                \right)
                -
                \left(
                    \molalpropcomp{intGibbsFreeEnergy}{6}^{\gls{standardState}}
                    +
                    \gls{universalGasConstant}
                    \gls{temperature}
                    \ln{
                        \propcomp{activity}{6}
                    }
                \right)
                -
                \frac{1}{2}
                \left(
                    \molalpropcomp{intGibbsFreeEnergy}{4}^{\gls{standardState}}
                    +
                    \gls{universalGasConstant}
                    \gls{temperature}
                    \ln{
                        \propcomp{activity}{4}
                    }
                \right)
            \right]
            \diff{\propcomp{reactionExtent}{5}}\\
            &+
            \left[
                \left(
                    \molalpropcomp{intGibbsFreeEnergy}{1}^{\gls{standardState}}
                    +
                    \gls{universalGasConstant}
                    \gls{temperature}
                    \ln{
                        \propcomp{activity}{1}
                    }
                \right)
                -
                \left(
                    \molalpropcomp{intGibbsFreeEnergy}{5}^{\gls{standardState}}
                    +
                    \gls{universalGasConstant}
                    \gls{temperature}
                    \ln{
                        \propcomp{activity}{5}
                    }
                \right)
                -
                \frac{1}{2}
                \left(
                    \molalpropcomp{intGibbsFreeEnergy}{4}^{\gls{standardState}}
                    +
                    \gls{universalGasConstant}
                    \gls{temperature}
                    \ln{
                        \propcomp{activity}{4}
                    }
                \right)
            \right]
            \diff{\propcomp{reactionExtent}{6}}
            =
            0\,.
        \end{aligned}
    \end{equation}

    Colocando em evidência o termo \gls{universalGasConstant}\gls{temperature},
    rearranjamos a \cref{eq:8.87}, e, lembrando que a soma de logaritmos
    corresponde ao logaritmo de produtos, chegamos a seguinte expressão:
    %
    \begin{equation} \label{eq:8.88}
        \begin{aligned}
            \diff{\gls{GibbsFreeEnergy}}
            =&\,\,\,\,\,\,
            \left[
                \left(
                    \molalpropcomp{intGibbsFreeEnergy}{7}^{\gls{standardState}}
                    -
                    \frac{1}{2}
                    \molalpropcomp{intGibbsFreeEnergy}{6}^{\gls{standardState}}
                \right)
                +
                \gls{universalGasConstant}
                \gls{temperature}
                \ln{
                    \left(
                        \frac{
                            \propcomp{activity}{7}
                        }{
                            \propcomp{activity}{6}^{\sfrac{1}{2}}
                        }
                    \right)
                }
            \right]
            \diff{\propcomp{reactionExtent}{1}}\\
            &+
            \left[
                \left(
                    \molalpropcomp{intGibbsFreeEnergy}{8}^{\gls{standardState}}
                    -
                    \frac{1}{2}
                    \molalpropcomp{intGibbsFreeEnergy}{4}^{\gls{standardState}}
                \right)
                +
                \gls{universalGasConstant}
                \gls{temperature}
                \ln{
                    \left(
                        \frac{
                            \propcomp{activity}{8}
                        }{
                            \propcomp{activity}{4}^{\sfrac{1}{2}}
                        }
                    \right)
                }
            \right]
            \diff{\propcomp{reactionExtent}{2}}\\
            &+
            \left[
                \left(
                    \molalpropcomp{intGibbsFreeEnergy}{9}^{\gls{standardState}}
                    -
                    \frac{1}{2}
                    \molalpropcomp{intGibbsFreeEnergy}{6}^{\gls{standardState}}
                    -
                    \frac{1}{2}
                    \molalpropcomp{intGibbsFreeEnergy}{4}^{\gls{standardState}}
                \right)
                +
                \gls{universalGasConstant}
                \gls{temperature}
                \ln{
                    \left(
                        \frac{
                            \propcomp{activity}{9}
                        }{
                            \propcomp{activity}{6}^{\sfrac{1}{2}}
                            \propcomp{activity}{4}^{\sfrac{1}{2}}
                        }
                    \right)
                }
            \right]
            \diff{\propcomp{reactionExtent}{3}}\\
            &+
            \left[
                \left(
                    \molalpropcomp{intGibbsFreeEnergy}{10}^{\gls{standardState}}
                    -
                    \frac{1}{2}
                    \molalpropcomp{intGibbsFreeEnergy}{3}^{\gls{standardState}}
                    -
                    \frac{1}{2}
                    \molalpropcomp{intGibbsFreeEnergy}{4}^{\gls{standardState}}
                \right)
                +
                \gls{universalGasConstant}
                \gls{temperature}
                \ln{
                    \left(
                        \frac{
                            \propcomp{activity}{10}
                        }{
                            \propcomp{activity}{3}^{\sfrac{1}{2}}
                            \propcomp{activity}{4}^{\sfrac{1}{2}}
                        }
                    \right)
                }
            \right]
            \diff{\propcomp{reactionExtent}{4}}\\
            &+
            \left[
                \left(
                    \molalpropcomp{intGibbsFreeEnergy}{2}^{\gls{standardState}}
                    -
                    \molalpropcomp{intGibbsFreeEnergy}{6}^{\gls{standardState}}
                    -
                    \frac{1}{2}
                    \molalpropcomp{intGibbsFreeEnergy}{4}^{\gls{standardState}}
                \right)
                +
                \gls{universalGasConstant}
                \gls{temperature}
                \ln{
                    \left(
                        \frac{
                            \propcomp{activity}{2}
                        }{
                            \propcomp{activity}{6}
                            \propcomp{activity}{4}^{\sfrac{1}{2}}
                        }
                    \right)
                }
            \right]
            \diff{\propcomp{reactionExtent}{5}}\\
            &+
            \left[
                \left(
                    \molalpropcomp{intGibbsFreeEnergy}{1}^{\gls{standardState}}
                    -
                    \molalpropcomp{intGibbsFreeEnergy}{5}^{\gls{standardState}}
                    -
                    \frac{1}{2}
                    \molalpropcomp{intGibbsFreeEnergy}{4}^{\gls{standardState}}
                \right)
                +
                \gls{universalGasConstant}
                \gls{temperature}
                \ln{
                    \left(
                        \frac{
                            \propcomp{activity}{1}
                        }{
                            \propcomp{activity}{5}
                            \propcomp{activity}{4}^{\sfrac{1}{2}}
                        }
                    \right)
                }
            \right]
            \diff{\propcomp{reactionExtent}{6}}
            =
            0\,.
        \end{aligned}
    \end{equation}

    Cada um dos termos entre parenteses na \cref{eq:8.88} é equivalente à
    variação da função de Gibbs %
    $
        \Delta \gsuper{GibbsFreeEnergy}{standardState}_k
        \functionOf{
            \gls{temperature},
            \gls{standardPressure}
        }
        =
        {\gsupsub{GibbsFreeEnergy}{standardState}{chProducts}}_{k}
        -
        {\gsupsub{GibbsFreeEnergy}{standardState}{chReactants}}_{k}
    $, $\forall k = 1,...,6$ por convenção, dos \enquote{reagentes} (esquerda)
    para os \enquote{produtos} (direita), com cada \enquote{reagente} e cada
    \enquote{produto} assumido estar a
    \gls{temperature},\gls{standardPressure}, para cada uma das seis reações,
    da seguinte forma:

    \begin{equation} \label{eq:8.89}
        \begin{array}{l l l}
            \ch{1/2 H2 <-> H}
        &   \ch{1/2 H2 -> H}
        &   \Delta \gsuper{GibbsFreeEnergy}{standardState}_1 =
            \molalpropcomp{intGibbsFreeEnergy}{7}^{\gls{standardState}}
            -
            \frac{1}{2}
            \molalpropcomp{intGibbsFreeEnergy}{6}^{\gls{standardState}}
        \\
            \ch{1/2 O2 <-> O}
        &   \ch{1/2 O2 -> O}
        &   \Delta \gsuper{GibbsFreeEnergy}{standardState}_2  =
            \molalpropcomp{intGibbsFreeEnergy}{8}^{\gls{standardState}}
            -
            \frac{1}{2}
            \molalpropcomp{intGibbsFreeEnergy}{4}^{\gls{standardState}}
        \\
            \ch{1/2 H2 + 1/2 O2 <-> OH}
        &   \ch{1/2 H2 + 1/2 O2 -> OH}
        &   \Delta \gsuper{GibbsFreeEnergy}{standardState}_3 =
            \molalpropcomp{intGibbsFreeEnergy}{9}^{\gls{standardState}}
            -
            \frac{1}{2}
            \molalpropcomp{intGibbsFreeEnergy}{6}^{\gls{standardState}}
            -
            \frac{1}{2}
            \molalpropcomp{intGibbsFreeEnergy}{4}^{\gls{standardState}}
        \\
            \ch{1/2 N2 + 1/2 O2 <-> NO}
        &   \ch{1/2 N2 + 1/2 O2 -> NO}
        &   \Delta \gsuper{GibbsFreeEnergy}{standardState}_4 =
            \molalpropcomp{intGibbsFreeEnergy}{10}^{\gls{standardState}}
            -
            \frac{1}{2}
            \molalpropcomp{intGibbsFreeEnergy}{3}^{\gls{standardState}}
            -
            \frac{1}{2}
            \molalpropcomp{intGibbsFreeEnergy}{4}^{\gls{standardState}}
        \\
            \ch{H2 + 1/2 O2 <-> H2O}
        &   \ch{H2 + 1/2 O2 -> H2O}
        &   \Delta \gsuper{GibbsFreeEnergy}{standardState}_5 =
            \molalpropcomp{intGibbsFreeEnergy}{2}^{\gls{standardState}}
            -
            \molalpropcomp{intGibbsFreeEnergy}{6}^{\gls{standardState}}
            -
            \frac{1}{2}
            \molalpropcomp{intGibbsFreeEnergy}{4}^{\gls{standardState}}
        \\
            \ch{CO + 1/2 O2 <-> CO2}
        &   \ch{CO + 1/2 O2 -> CO2}
        &   \Delta \gsuper{GibbsFreeEnergy}{standardState}_6 =
            \molalpropcomp{intGibbsFreeEnergy}{1}^{\gls{standardState}}
            -
            \molalpropcomp{intGibbsFreeEnergy}{5}^{\gls{standardState}}
            -
            \frac{1}{2}
                \molalpropcomp{intGibbsFreeEnergy}{4}^{\gls{standardState}}
            \\
        \end{array}
    \end{equation}

    Observe como cada um dos $\Delta
    \gsuper{GibbsFreeEnergy}{standardState}_k\functionOf{\gls{temperature},
    \gls{standardPressure}}$ nas \cref{eq:8.89} está vinculado a sua respectiva
    reação $k$ não apenas na direção --- da esquerda para a direita, mas também
    no valor de seus coeficientes estequiométricos. Podemos então afirmar que
    se multiplicarmos qualquer das equações $k$ por um número positivo ou
    negativo, o seu respectivo $\Delta
    \gsuper{GibbsFreeEnergy}{standardState}_k\functionOf{\gls{temperature},
    \gls{standardPressure}}$ será multiplicado pelo mesmo número. Além disso,
    qualquer combinação linear entre as seis equações incorrerá na mesma
    combinação linear para os respectivos $\Delta
    \gsuper{GibbsFreeEnergy}{standardState}_k$. Vejamos um exemplo importante.
    Peguemos a equação 6 e multipliquemos por -1:
    %
    \begin{equation} \label{eq:8.90}
        \begin{array}{l l l}
            \ch{CO2 <-> CO + 1/2 O2}
        &   \ch{CO2 -> CO + 1/2 O2}
        &   \Delta \gsuper{GibbsFreeEnergy}{standardState}_7 =
            -\molalpropcomp{intGibbsFreeEnergy}{1}^{\gls{standardState}}
            +
            \molalpropcomp{intGibbsFreeEnergy}{5}^{\gls{standardState}}
            +
            \frac{1}{2}
            \molalpropcomp{intGibbsFreeEnergy}{4}^{\gls{standardState}} \\
        \end{array}
    \end{equation}

    Se somarmos a equação 7 com a equação 5, resultará na equação 8:
    %
    \begin{equation} \label{eq:8.91}
        \begin{array}{l l l}
            \ch{H2 + CO2 <-> H2O + CO}
        &   \ch{H2 + CO2 <-> H2O + CO}
        &   \Delta \gsuper{GibbsFreeEnergy}{standardState}_8 =
            \molalpropcomp{intGibbsFreeEnergy}{2}^{\gls{standardState}}
            +
            \molalpropcomp{intGibbsFreeEnergy}{5}^{\gls{standardState}}
            -
            \molalpropcomp{intGibbsFreeEnergy}{6}^{\gls{standardState}}
            -
            \molalpropcomp{intGibbsFreeEnergy}{1}^{\gls{standardState}} \\
        \end{array}
    \end{equation}

    A reação química reversível da \cref{eq:8.91} é extremamente importante e
    ocorre com qualquer temperatura dos produtos sempre que a razão de
    equivalência for maior do que 1, ou seja o ar utilizado na combustão é
    menor do que o ar estequiométrico para o combustível. Ela é chamada de
    \emph{reação água-gás}.

    Como inicialmente estabelecemos, poderíamos ter escolhido as reações 7 ou
    8, ou mesmo ambas, para substituir uma ou duas reações do sistema de seis,
    \cref{eq:8.81}. Seja qual for a nossa escolha, enfatizamos que só podemos
    encontrar seis reações que são independentes entre si, nem mais e nem
    menos.

    Vamos definir a constante de equilíbrio
    $\gls{chEquilibriumConstant}_{k}\functionOf{\gls{temperature}}$ para uma
    reação reversível $k$ como sendo:
    %
    \begin{equation} \label{eq:8.92}
        \ln{
            \gls{chEquilibriumConstant}_k
            \functionOf{
                \gls{temperature}
            }
        }
        =
        -\frac{
            \Delta \gsuper{GibbsFreeEnergy}{standardState}_k
            \functionOf{
                \gls{temperature},
                \gls{standardPressure}
            }
        }{
            \gls{universalGasConstant}
            \gls{temperature}
        }\,,
        \,\,\,\,
        \forall k = 1,...,6\,.
    \end{equation}

    Observe que as constantes de equilíbrio não são realmente constantes, mas
    uma função da temperatura da reação química \gls{temperature}. Entretanto,
    este nome acabou permanecendo por razões históricas. De que maneira a
    constante de equilíbrio varia com a temperatura? Voltaremos a essa questão,
    mas antes vamos obter as equações de equilíbrio que utilizam as constantes.

    A partir da \cref{eq:8.88}, igualando-se cada coeficiente a zero e
    substituindo-se para as constantes de equilíbrio, obteremos equações que
    representam para cada reação química reversível a relação entre a
    respectiva constante de equilíbrio e as atividades e as fugacidades dos
    componentes que participam daquela reação, dentro da mistura homogênea de
    produtos de combustão em equilíbrio a \gls{temperature}, \gls{pressure}.
    Assumiremos, além disso, por conveniência que no estado padrão a
    \gls{temperature} e \gls{standardPressure} os componentes puros $i$ são
    gases perfeitos, isto é,
    $\propcomp{fugacity}{i}\functionOf{\gls{temperature},
    \gls{standardPressure}} = \gls{standardPressure}$. Obtemos assim:
    %
    \begin{equation} \label{eq:8.93}
        \begin{aligned}
        &\propcomp{chEquilibriumConstant}{1}
        \functionOf{
            \gls{temperature}
        }
        =
        \frac{
            \propcomp{activity}{7}
        }{
            \propcomp{activity}{6}^{\sfrac{1}{2}}
        }
        =
        \frac{
            \sfrac{
                \molalpropcomp{fugacity}{7}
            }{
                \gls{pressure}
            }
        }{
            \left(
                \sfrac{
                    \molalpropcomp{fugacity}{6}
                }{
                    \gls{pressure}
                }
            \right)^{\sfrac{1}{2}}
        }
        \left(
            \frac{
                \gls{pressure}
            }{
                \gls{standardPressure}
            }
        \right)^{\sfrac{1}{2}} \\
        &\propcomp{chEquilibriumConstant}{2}
        \functionOf{
            \gls{temperature}
        }
        =
        \frac{
            \propcomp{activity}{8}
        }{
            \propcomp{activity}{4}^{\sfrac{1}{2}}
        }
        =
        \frac{
            \sfrac{
                \molalpropcomp{fugacity}{8}
            }{
                \gls{pressure}
            }
        }{
            \left(
                \sfrac{
                    \molalpropcomp{fugacity}{4}
                }{
                    \gls{pressure}
                }
            \right)^{\sfrac{1}{2}}
        }
        \left(
            \frac{
                \gls{pressure}
            }{
                \gls{standardPressure}
            }
        \right)^{\sfrac{1}{2}} \\
        &\propcomp{chEquilibriumConstant}{3}
        \functionOf{
            \gls{temperature}
        }
        =
        \frac{
            \propcomp{activity}{9}
        }{
            \propcomp{activity}{6}^{\sfrac{1}{2}}
            \propcomp{activity}{4}^{\sfrac{1}{2}}
        }
        =
        \frac{
            \sfrac{
                \molalpropcomp{fugacity}{9}
            }{
                \gls{pressure}
            }
        }{
            \left(
                \sfrac{
                    \molalpropcomp{fugacity}{6}
                }{
                    \gls{pressure}
                }
            \right)^{\sfrac{1}{2}}
            \left(
                \sfrac{
                    \molalpropcomp{fugacity}{4}
                }{
                    \gls{pressure}
                }
            \right)^{\sfrac{1}{2}}
        }
        \left(
            \frac{
                \gls{pressure}
            }{
                \gls{standardPressure}
            }
        \right)^{0} \\
        &\propcomp{chEquilibriumConstant}{4}
        \functionOf{
            \gls{temperature}
        }
        =
        \frac{
            \propcomp{activity}{10}
        }{
            \propcomp{activity}{3}^{\sfrac{1}{2}}
            \propcomp{activity}{4}^{\sfrac{1}{2}}
        }
        =
        \frac{
            \sfrac{
                \molalpropcomp{fugacity}{10}
            }{
                \gls{pressure}
            }
        }{
            \left(
                \sfrac{
                    \molalpropcomp{fugacity}{3}
                }{
                    \gls{pressure}
                }
            \right)^{\sfrac{1}{2}}
            \left(
                \sfrac{
                    \molalpropcomp{fugacity}{4}
                }{
                    \gls{pressure}
                }
            \right)^{\sfrac{1}{2}}
        }
        \left(
            \frac{
                \gls{pressure}
            }{
                \gls{standardPressure}
            }
        \right)^{0} \\
        &\propcomp{chEquilibriumConstant}{5}
        \functionOf{
            \gls{temperature}
        }
        =
        \frac{
            \propcomp{activity}{2}
        }{
            \propcomp{activity}{6}
            \propcomp{activity}{4}^{\sfrac{1}{2}}
        }
        =
        \frac{
            \sfrac{
                \molalpropcomp{fugacity}{9}
            }{
                \gls{pressure}
            }
        }{
            \left(
                \sfrac{
                    \molalpropcomp{fugacity}{2}
                }{
                    \gls{pressure}
                }
            \right)
            \left(
                \sfrac{
                    \molalpropcomp{fugacity}{4}
                }{
                    \gls{pressure}
                }
            \right)^{\sfrac{1}{2}}
        }
        \left(
            \frac{
                \gls{pressure}
            }{
                \gls{standardPressure}
            }
        \right)^{-\sfrac{1}{2}} \\
        &\propcomp{chEquilibriumConstant}{6}
        \functionOf{
            \gls{temperature}
        }
        =
        \frac{
            \propcomp{activity}{1}
        }{
            \propcomp{activity}{5}
            \propcomp{activity}{4}^{\sfrac{1}{2}}
        }
        =
        \frac{
            \sfrac{
                \molalpropcomp{fugacity}{1}
            }{
                \gls{pressure}
            }
        }{
            \left(
                \sfrac{
                    \molalpropcomp{fugacity}{5}
                }{
                    \gls{pressure}
                }
            \right)
            \left(
                \sfrac{
                    \molalpropcomp{fugacity}{4}
                }{
                    \gls{pressure}
                }
            \right)^{\sfrac{1}{2}}
        }
        \left(
            \frac{
                \gls{pressure}
            }{
                \gls{standardPressure}
            }
        \right)^{-\sfrac{1}{2}}\,. \\
        \end{aligned}
    \end{equation}

    Como sempre, pouco poderemos avançar se não modelarmos as fugacidades dos
    componentes na mistura. Pois bem, se a mistura de produtos de combustão
    puder ser considerada como solução ideal a \gls{temperature},
    \gls{pressure}, então as Eqs. 8.93 se desdobrarão em:
    %
    \begin{equation} \label{eq:8.94}
        \begin{aligned}
        &\propcomp{chEquilibriumConstant}{1}
        \functionOf{
            \gls{temperature}
        }
        =
        \frac{
            \propcomp{activity}{7}
        }{
            \propcomp{activity}{6}^{\sfrac{1}{2}}
        }
        =
        \left(
            \frac{
                \propcomp{moleFraction}{7}
            }{
                \propcomp{moleFraction}{6}^{\sfrac{1}{2}}
            }
        \right)
        \frac{
            \sfrac{
                \propcomp{fugacity}{7}
            }{
                \gls{pressure}
            }
        }{
            \left(
                \sfrac{
                    \propcomp{fugacity}{6}
                }{
                    \gls{pressure}
                }
            \right)^{\sfrac{1}{2}}
        }
        \left(
            \frac{
                \gls{pressure}
            }{
                \gls{standardPressure}
            }
        \right)^{\sfrac{1}{2}} \\
        &\propcomp{chEquilibriumConstant}{2}
        \functionOf{
            \gls{temperature}
        }
        =
        \frac{
            \propcomp{activity}{8}
        }{
            \propcomp{activity}{4}^{\sfrac{1}{2}}
        }
        =
        \left(
            \frac{
                \propcomp{moleFraction}{8}
            }{
                \propcomp{moleFraction}{4}^{\sfrac{1}{2}}
            }
        \right)
        \frac{
            \sfrac{
                \propcomp{fugacity}{8}
            }{
                \gls{pressure}
            }
        }{
            \left(
                \sfrac{
                    \propcomp{fugacity}{4}
                }{
                    \gls{pressure}
                }
            \right)^{\sfrac{1}{2}}
        }
        \left(
            \frac{
                \gls{pressure}
            }{
                \gls{standardPressure}
            }
        \right)^{\sfrac{1}{2}} \\
        &\propcomp{chEquilibriumConstant}{3}
        \functionOf{
            \gls{temperature}
        }
        =
        \frac{
            \propcomp{activity}{9}
        }{
            \propcomp{activity}{6}^{\sfrac{1}{2}}
            \propcomp{activity}{4}^{\sfrac{1}{2}}
        }
        =
        \left(
            \frac{
                \propcomp{moleFraction}{9}
            }{
                \propcomp{moleFraction}{6}^{\sfrac{1}{2}}
                \propcomp{moleFraction}{4}^{\sfrac{1}{2}}
            }
        \right)
        \frac{
            \sfrac{
                \propcomp{fugacity}{9}
            }{
                \gls{pressure}
            }
        }{
            \left(
                \sfrac{
                    \propcomp{fugacity}{6}
                }{
                    \gls{pressure}
                }
            \right)^{\sfrac{1}{2}}
            \left(
                \sfrac{
                    \propcomp{fugacity}{4}
                }{
                    \gls{pressure}
                }
            \right)^{\sfrac{1}{2}}
        }
        \left(
            \frac{
                \gls{pressure}
            }{
                \gls{standardPressure}
            }
        \right)^{0} \\
        &\propcomp{chEquilibriumConstant}{4}
        \functionOf{
            \gls{temperature}
        }
        =
        \frac{
            \propcomp{activity}{10}
        }{
            \propcomp{activity}{3}^{\sfrac{1}{2}}
            \propcomp{activity}{4}^{\sfrac{1}{2}}
        }
        =
        \left(
            \frac{
                \propcomp{moleFraction}{10}
            }{
                \propcomp{moleFraction}{3}^{\sfrac{1}{2}}
                \propcomp{moleFraction}{4}^{\sfrac{1}{2}}
            }
        \right)
        \frac{
            \sfrac{
                \propcomp{fugacity}{10}
            }{
                \gls{pressure}
            }
        }{
            \left(
                \sfrac{
                    \propcomp{fugacity}{3}
                }{
                    \gls{pressure}
                }
            \right)^{\sfrac{1}{2}}
            \left(
                \sfrac{
                    \propcomp{fugacity}{4}
                }{
                    \gls{pressure}
                }
            \right)^{\sfrac{1}{2}}
        }
        \left(
            \frac{
                \gls{pressure}
            }{
                \gls{standardPressure}
            }
        \right)^{0} \\
        &\propcomp{chEquilibriumConstant}{5}
        \functionOf{
            \gls{temperature}
        }
        =
        \frac{
            \propcomp{activity}{2}
        }{
            \propcomp{activity}{6}
            \propcomp{activity}{4}^{\sfrac{1}{2}}
        }
        =
        \left(
            \frac{
                \propcomp{moleFraction}{2}
            }{
                \propcomp{moleFraction}{6}
                \propcomp{moleFraction}{4}^{\sfrac{1}{2}}
            }
        \right)
        \frac{
            \sfrac{
                \propcomp{fugacity}{9}
            }{
                \gls{pressure}
            }
        }{
            \left(
                \sfrac{
                    \propcomp{fugacity}{2}
                }{
                    \gls{pressure}
                }
            \right)
            \left(
                \sfrac{
                    \propcomp{fugacity}{4}
                }{
                    \gls{pressure}
                }
            \right)^{\sfrac{1}{2}}
        }
        \left(
            \frac{
                \gls{pressure}
            }{
                \gls{standardPressure}
            }
        \right)^{-\sfrac{1}{2}} \\
        &\propcomp{chEquilibriumConstant}{6}
        \functionOf{
            \gls{temperature}
        }
        =
        \frac{
            \propcomp{activity}{1}
        }{
            \propcomp{activity}{5}
            \propcomp{activity}{4}^{\sfrac{1}{2}}
        }
        =
        \left(
            \frac{
                \propcomp{moleFraction}{1}
            }{
                \propcomp{moleFraction}{5}
                \propcomp{moleFraction}{4}^{\sfrac{1}{2}}
            }
        \right)
        \frac{
            \sfrac{
                \propcomp{fugacity}{1}
            }{
                \gls{pressure}
            }
        }{
            \left(
                \sfrac{
                    \propcomp{fugacity}{5}
                }{
                    \gls{pressure}
                }
            \right)
            \left(
                \sfrac{
                    \propcomp{fugacity}{4}
                }{
                    \gls{pressure}
                }
            \right)^{\sfrac{1}{2}}
        }
        \left(
            \frac{
                \gls{pressure}
            }{
                \gls{standardPressure}
            }
        \right)^{-\sfrac{1}{2}}\,. \\
        \end{aligned}
    \end{equation}

    Nas \cref{eq:8.94} os termos
    $\dfrac{\propcomp{fugacity}{i}}{\gls{pressure}}$ são os coeficientes de
    fugacidade do componente $i$ puro a \gls{temperature}, \gls{pressure}, que
    poderiam ser obtidos através das coordenadas generalizadas.

    Finalmente, se a solução ideal dos produtos de combustão a
    \gls{temperature}, \gls{pressure} puder ser considerada como uma mistura de
    gases perfeitos, os coeficientes de fugacidade serão todos iguais a unidade
    e obteremos as equações das constantes de equilíbrio na sua forma mais
    conhecida:
    %
    \begin{equation} \label{eq:8.95}
        \begin{aligned}
        &\propcomp{chEquilibriumConstant}{1}
        \functionOf{
            \gls{temperature}
        }
        =
        \left(
            \frac{
                \propcomp{moleFraction}{7}
            }{
                \propcomp{moleFraction}{6}^{\sfrac{1}{2}}
            }
        \right)
        \left(
            \frac{
                \gls{pressure}
            }{
                \gls{standardPressure}
            }
        \right)^{\sfrac{1}{2}} \\
        &\propcomp{chEquilibriumConstant}{2}
        \functionOf{
            \gls{temperature}
        }
        =
        \left(
            \frac{
                \propcomp{moleFraction}{8}
            }{
                \propcomp{moleFraction}{4}^{\sfrac{1}{2}}
            }
        \right)
        \left(
            \frac{
                \gls{pressure}
            }{
                \gls{standardPressure}
            }
        \right)^{\sfrac{1}{2}} \\
        &\propcomp{chEquilibriumConstant}{3}
        \functionOf{
            \gls{temperature}
        }
        =
        \left(
            \frac{
                \propcomp{moleFraction}{9}
            }{
                \propcomp{moleFraction}{6}^{\sfrac{1}{2}}
                \propcomp{moleFraction}{4}^{\sfrac{1}{2}}
            }
        \right)
        \left(
            \frac{
                \gls{pressure}
            }{
                \gls{standardPressure}
            }
        \right)^{0} \\
        &\propcomp{chEquilibriumConstant}{4}
        \functionOf{
            \gls{temperature}
        }
        =
        \left(
            \frac{
                \propcomp{moleFraction}{10}
            }{
                \propcomp{moleFraction}{3}^{\sfrac{1}{2}}
                \propcomp{moleFraction}{4}^{\sfrac{1}{2}}
            }
        \right)
        \left(
            \frac{
                \gls{pressure}
            }{
                \gls{standardPressure}
            }
        \right)^{0} \\
        &\propcomp{chEquilibriumConstant}{5}
        \functionOf{
            \gls{temperature}
        }
        =
        \left(
            \frac{
                \propcomp{moleFraction}{2}
            }{
                \propcomp{moleFraction}{6}
                \propcomp{moleFraction}{4}^{\sfrac{1}{2}}
            }
        \right)
        \left(
            \frac{
                \gls{pressure}
            }{
                \gls{standardPressure}
            }
        \right)^{-\sfrac{1}{2}} \\
        &\propcomp{chEquilibriumConstant}{6}
        \functionOf{
            \gls{temperature}
        }
        =
        \left(
            \frac{
                \propcomp{moleFraction}{1}
            }{
                \propcomp{moleFraction}{5}
                \propcomp{moleFraction}{4}^{\sfrac{1}{2}}
            }
        \right)
        \left(
            \frac{
                \gls{pressure}
            }{
                \gls{standardPressure}
            }
        \right)^{-\sfrac{1}{2}}\,.\\
        \end{aligned}
    \end{equation}

    Chamamos a sua atenção para o fato de que as conhecidas \cref{eq:8.95} só
    podem ser empregadas quando a mistura em equilíbrio puder ser considerada
    como de gases perfeitos a \gls{temperature}, \gls{pressure}, uma hipótese,
    aliás, bem razoável nos processos de combustão encontrados na prática.

    Em cada uma das reações de equilíbrio a reação se desloca para a esquerda
    ou direita conforme alteramos a pressão ou a temperatura da reação.

    Como poderemos saber a direção em que uma determinada reação se deslocará
    se alterarmos a sua pressão, enquanto mantemos a sua temperatura
    inalterada?

    Vamos estudar a primeira (ou a segunda) das \cref{eq:8.95}. Como o produto
    do lado direito é constante, se aumentarmos a pressão \gls{pressure}, o
    numerador da relação de frações molares deve diminuir e o denominador
    aumentar, ou seja, a reação se desloca em direção aos reagentes, para a
    esquerda.

    Vamos ver agora a reação 5 (ou a reação 6): o aumento da pressão  implicará
    o aumento do numerador (por quê?) e a diminuição do denominador, deslocando
    a reação para a direita, no sentido dos produtos.

    O que podemos dizer das reações 3 ou 4? Claramente a alteração da pressão
    não terá efeito sobre estas reações (por quê?).

    Então, já podemos estabelecer uma regra: se o número de moles dos produtos
    for maior do que o número de moles dos reagentes, então o aumento da
    pressão causará um deslocamento para os reagentes e vice-versa. Este é o
    chamado \emph{princípio de Le Chatelier}.

    O que podemos dizer a respeito do efeito da temperatura? Se o valor da
    constante de equilíbrio aumentar com a temperatura, então um aumento da
    temperatura implica um deslocamento da reação no sentido dos produtos (por
    quê?). Ao contrário, se o valor da constante de equilíbrio decrescer com o
    aumento da temperatura, então a reação correspondente se deslocará para os
    reagentes. Ou seja, não poderemos decidir qual é o efeito enquanto não
    analisarmos com mais detalhes o comportamento da constante de equilíbrio
    com a temperatura.

    Para uma substância pura $i$ podemos escrever:
    %
    \begin{equation*}
        \diff\molalpropcomp{intGibbsFreeEnergy}{i}
        =
        -\molalpropcomp{entropy}{i}
        \diff{\gls{temperature}}
        +
        \molalpropcomp{specificVolume}{i}
        \diff{\gls{pressure}}\,.
    \end{equation*}

    A diferencial calculada a pressão constante ficará:
    %
    \begin{equation} \label{eq:8.96}
        \diff\molalpropcomp{intGibbsFreeEnergy}{i}\big|_{\gls{pressure}}
        =
        -\molalpropcomp{entropy}{i}
        \diff{\gls{temperature}}\big|_{\gls{pressure}}
        =
        -
        \frac{
            \left(
                \molalpropcomp{intEnthalpy}{i}
                -
                \molalpropcomp{intGibbsFreeEnergy}{i}
            \right)
        }{
            \gls{temperature}
        }
        \diff{\gls{temperature}}\big|_{\gls{pressure}}\,.
    \end{equation}

    Rearranjando a \cref{eq:8.96} e dividindo-se por $\gls{temperature}^2$:
    %
    \begin{equation} \label{eq:8.97}
        \frac{
            \gls{temperature}
            \diff\molalpropcomp{intGibbsFreeEnergy}{i}\big|_{\gls{pressure}}
            -
            \molalpropcomp{intGibbsFreeEnergy}{i}
            \diff{\gls{temperature}}\big|_{\gls{pressure}}\,.
        }{
            \gls{temperature}^2
        }
        =
        -
        \frac{
            \molalpropcomp{intEnthalpy}{i}
        }{
            \gls{temperature}^2
        }
        \diff{\gls{temperature}}\big|_{\gls{pressure}}\,,
    \end{equation}
    %
    de onde obtemos:

    \begin{equation} \label{eq:8.98}
        \diff{}
        \left(
            \frac{
                \molalpropcomp{intGibbsFreeEnergy}{i}
            }{
                \gls{temperature}
            }
        \right)_{\gls{pressure}}
        =
        -
        \frac{
            \molalpropcomp{intEnthalpy}{i}
        }{
            \gls{temperature}^2
        }
        \diff{\gls{temperature}}\big|_{\gls{pressure}}\,.
    \end{equation}

    Vamos rescrever a mesma expressão para a substância pura $i$ mas no estado
    padrão a \gls{temperature}, \gls{standardPressure}, dividindo
    adicionalmente por \gls{universalGasConstant}:
    %
    \begin{equation} \label{eq:8.99}
        \diff{}
        \left(
            \frac{
                \molalpropcomp{intGibbsFreeEnergy}{i}
            }{
                \gls{universalGasConstant}
                \gls{temperature}
            }
        \right)_{\gls{standardPressure}}
        =
        -
        \frac{
            \molalpropcomp{intEnthalpy}{i}
        }{
            \gls{universalGasConstant}
            \gls{temperature}^2
        }
        \diff{\gls{temperature}}\big|_{\gls{standardPressure}}\,.
    \end{equation}

    Então, para qualquer uma das reações de equilíbrio $k$ podemos escrever:
    %
    \begin{equation} \label{eq:8.100}
        \diff{}
        \left(
            \frac{
                \Delta \gsuper{GibbsFreeEnergy}{standardState}_k
            }{
                \gls{universalGasConstant}
                \gls{temperature}
            }
        \right)_{\gls{standardPressure}}
        =
        -
        \frac{
            \Delta \gsuper{enthalpy}{standardState}_k
        }{
            \gls{universalGasConstant}
            \gls{temperature}^2
        }
        \diff{\gls{temperature}}\big|_{\gls{standardPressure}}\,,
    \end{equation}
    %
    ou ainda, da definição de constante de equilíbrio:
    %
    \begin{equation} \label{eq:8.101}
        \diff{}
        \left(
            \ln{
                \gls{chEquilibriumConstant}_k
            }
        \right)
        =
        -
        \frac{
            \Delta \gsuper{enthalpy}{standardState}_k
        }{
            \gls{universalGasConstant}
            \gls{temperature}^2
        }
        \diff{\gls{temperature}}\big|_{\gls{standardPressure}}
        =
        -
        \frac{
            \Delta \gsuper{enthalpy}{standardState}_k
        }{
            \gls{universalGasConstant}
        }
        \diff{\left(
            \frac{1}{\gls{temperature}}
        \right)}_{\gls{standardPressure}}\,.
    \end{equation}

    Assim como a variação da função de Gibbs associada a cada uma das $k$
    reações, podemos também definir $\Delta
    \gsuper{enthalpy}{standardState}_k\functionOf{\gls{temperature},
    \gls{standardPressure}}$, a entalpia de reação no estado padrão a
    \gls{temperature},\gls{standardPressure}. De fato, por analogia com as
    \cref{eq:8.89}, podemos escrever:

    \begin{equation} \label{eq:8.102}
        \begin{array}{l l l}
            \ch{1/2 H2 <-> H}
        &   \ch{1/2 H2 -> H}
        &   \Delta \gsuper{enthalpy}{standardState}_1 =
            \molalpropcomp{intEnthalpy}{7}^{\gls{standardState}}
            -
            \frac{1}{2}
            \molalpropcomp{intEnthalpy}{6}^{\gls{standardState}}
        \\
            \ch{1/2 O2 <-> O}
        &   \ch{1/2 O2 -> O}
        &   \Delta \gsuper{enthalpy}{standardState}_2  =
            \molalpropcomp{intEnthalpy}{8}^{\gls{standardState}}
            -
            \frac{1}{2}
            \molalpropcomp{intEnthalpy}{4}^{\gls{standardState}}
        \\
            \ch{1/2 H2 + 1/2 O2 <-> OH}
        &   \ch{1/2 H2 + 1/2 O2 -> OH}
        &   \Delta \gsuper{enthalpy}{standardState}_3 =
            \molalpropcomp{intEnthalpy}{9}^{\gls{standardState}}
            -
            \frac{1}{2}
            \molalpropcomp{intEnthalpy}{6}^{\gls{standardState}}
            -
            \frac{1}{2}
            \molalpropcomp{intEnthalpy}{4}^{\gls{standardState}}
        \\
            \ch{1/2 N2 + 1/2 O2 <-> NO}
        &   \ch{1/2 N2 + 1/2 O2 -> NO}
        &   \Delta \gsuper{enthalpy}{standardState}_4 =
            \molalpropcomp{intEnthalpy}{10}^{\gls{standardState}}
            -
            \frac{1}{2}
            \molalpropcomp{intEnthalpy}{3}^{\gls{standardState}}
            -
            \frac{1}{2}
            \molalpropcomp{intEnthalpy}{4}^{\gls{standardState}}
        \\
            \ch{H2 + 1/2 O2 <-> H2O}
        &   \ch{H2 + 1/2 O2 -> H2O}
        &   \Delta \gsuper{enthalpy}{standardState}_5 =
            \molalpropcomp{intEnthalpy}{2}^{\gls{standardState}}
            -
            \molalpropcomp{intEnthalpy}{6}^{\gls{standardState}}
            -
            \frac{1}{2}
            \molalpropcomp{intEnthalpy}{4}^{\gls{standardState}}
        \\
            \ch{CO + 1/2 O2 <-> CO2}
        &   \ch{CO + 1/2 O2 -> CO2}
        &   \Delta \gsuper{enthalpy}{standardState}_6 =
            \molalpropcomp{intEnthalpy}{1}^{\gls{standardState}}
            -
            \molalpropcomp{intEnthalpy}{5}^{\gls{standardState}}
            -
            \frac{1}{2}
                \molalpropcomp{intEnthalpy}{4}^{\gls{standardState}}
            \\
        \end{array}
    \end{equation}

    A característica principal da entalpia de reação é que o seu valor é
    positivo para as reações endotérmicas, ou seja, que absorvem energia, como
    as de dissociação 1 e 2 e negativo para as reações exotérmicas, ou seja,
    que liberam energia, como as de oxidação 3, 4, 5 e 6. Assim como a função
    de Gibbs do estado padrão, a entalpia de reação também está estreitamente
    associada a uma determinada reação de equilíbrio. Se multiplicarmos a
    reação por um número positivo ou negativo, a entalpia de reação será
    afetada da mesma forma.

    Da análise da \cref{eq:8.101} podemos concluir duas coisas. A primeira, que
    o logaritmo da constante de equilíbrio varia linearmente com o inverso da
    temperatura absoluta, se a entalpia de reação for razoavelmente constante
    no intervalo de temperatura considerado. A segunda, que a variação da
    constante de equilíbrio com a temperatura depende apenas do sinal da
    entalpia de reação. Se a reação considerada for endotérmica, o valor da
    constante de equilíbrio aumenta com a temperatura, ou seja, a dissociação
    aumenta com a temperatura (por quê?), ao passo que o valor decresce com o
    aumento da temperatura para as reações exotérmicas, ou seja, a reação se
    desloca no sentido dos reagentes (por quê?). A propósito, a \cref{eq:8.101}
    é denominada \emph{equação de Van't Hoff}.

    Retornando ao equacionamento do problema original, vamos assumir que os
    produtos de combustão formam uma mistura de gases perfeitos a
    \gls{temperature}, \gls{pressure}. Expressando as \cref{eq:8.95} em termos
    de número de moles:

    \begin{equation} \label{eq:8.103}
        \begin{aligned}
        &\propcomp{chEquilibriumConstant}{1}
        \functionOf{
            \gls{temperature}
        }
        =
        \left(
            \frac{
                \propcomp{numberMoles}{7}
            }{
                \propcomp{numberMoles}{6}^{\sfrac{1}{2}}
                \gls{numberMoles}^{\sfrac{1}{2}}
            }
        \right)
        \left(
            \frac{
                \gls{pressure}
            }{
                \gls{standardPressure}
            }
        \right)^{\sfrac{1}{2}} \\
        &\propcomp{chEquilibriumConstant}{2}
        \functionOf{
            \gls{temperature}
        }
        =
        \left(
            \frac{
                \propcomp{numberMoles}{8}
            }{
                \propcomp{numberMoles}{4}^{\sfrac{1}{2}}
                \gls{numberMoles}^{\sfrac{1}{2}}
            }
        \right)
        \left(
            \frac{
                \gls{pressure}
            }{
                \gls{standardPressure}
            }
        \right)^{\sfrac{1}{2}} \\
        &\propcomp{chEquilibriumConstant}{3}
        \functionOf{
            \gls{temperature}
        }
        =
        \left(
            \frac{
                \propcomp{numberMoles}{9}
            }{
                \propcomp{numberMoles}{6}^{\sfrac{1}{2}}
                \propcomp{numberMoles}{4}^{\sfrac{1}{2}}
            }
        \right)
        \left(
            \frac{
                \gls{pressure}
            }{
                \gls{standardPressure}
            }
        \right)^{0} \\
        &\propcomp{chEquilibriumConstant}{4}
        \functionOf{
            \gls{temperature}
        }
        =
        \left(
            \frac{
                \propcomp{numberMoles}{10}
            }{
                \propcomp{numberMoles}{3}^{\sfrac{1}{2}}
                \propcomp{numberMoles}{4}^{\sfrac{1}{2}}
            }
        \right)
        \left(
            \frac{
                \gls{pressure}
            }{
                \gls{standardPressure}
            }
        \right)^{0} \\
        &\propcomp{chEquilibriumConstant}{5}
        \functionOf{
            \gls{temperature}
        }
        =
        \left(
            \frac{
                \propcomp{numberMoles}{2}
                \gls{numberMoles}^{\sfrac{1}{2}}
            }{
                \propcomp{numberMoles}{6}
                \propcomp{numberMoles}{4}^{\sfrac{1}{2}}
            }
        \right)
        \left(
            \frac{
                \gls{pressure}
            }{
                \gls{standardPressure}
            }
        \right)^{-\sfrac{1}{2}} \\
        &\propcomp{chEquilibriumConstant}{6}
        \functionOf{
            \gls{temperature}
        }
        =
        \left(
            \frac{
                \propcomp{numberMoles}{1}
                \gls{numberMoles}^{\sfrac{1}{2}}
            }{
                \propcomp{numberMoles}{5}
                \propcomp{numberMoles}{4}^{\sfrac{1}{2}}
            }
        \right)
        \left(
            \frac{
                \gls{pressure}
            }{
                \gls{standardPressure}
            }
        \right)^{-\sfrac{1}{2}} \\
        \end{aligned}
    \end{equation}

    Quantas e quais são afinal as equações do sistema? Vejamos: são as seis
    equações com as constantes de equilíbrio, \cref{eq:8.103}, adicionadas das
    quatro equações de conservação dos elementos químicos, \cref{eq:8.66}, a
    soma do número de moles \gls{numberMoles} encontrada na última das
    \cref{eq:8.70} e uma equação para \gls{molarAirFuelRatio}, \cref{eq:8.59},
    em um total de doze equações.  Analisando as
    \cref{eq:8.59,eq:8.66,eq:8.70,eq:8.103}, chegamos ao seguinte  conjunto de
    dezenove variáveis:
    %
    \begin{equation*}
         \gls{carbonQuantity}, \gls{hidrogenQuantity},
         \gls{oxygenQuantity}, \gls{nitrogenQuantity},
         \gls{equivalenceRatio},
         \gls{molarAirFuelRatio},
         \propcomp{numberMoles}{1}, \propcomp{numberMoles}{2},
         \propcomp{numberMoles}{3}, \propcomp{numberMoles}{4},
         \propcomp{numberMoles}{5}, \propcomp{numberMoles}{6},
         \propcomp{numberMoles}{7}, \propcomp{numberMoles}{8},
         \propcomp{numberMoles}{9}, \propcomp{numberMoles}{10},
         \gls{numberMoles}, \gls{temperature}, \gls{pressure}\,,
    \end{equation*}
    %
    ou seja, sete a mais do que o número de equações do sistema, consistente
    com o número que obtivemos com o método dos multiplicadores de Lagrange. Da
    mesma forma que com aquele método, precisamos de sete equações ou dos meios
    para obter o valor de sete variáveis, para que possamos buscar soluções
    para o problema. Em geral, \gls{carbonQuantity}, \gls{hidrogenQuantity},
    \gls{oxygenQuantity}, \gls{nitrogenQuantity} e consequentemente
    \gls{molarAirFuelRatio}, são valores dados pela escolha do combustível e a
    razão de equivalência \gls{equivalenceRatio} é um parâmetro a ser
    investigado. Resta-nos então a temperatura \gls{temperature} e a pressão
    \gls{pressure} dos produtos de combustão.
