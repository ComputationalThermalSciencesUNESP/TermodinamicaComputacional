\documentclass[brazilian,a4paper,12pt,two-sided]{book}

    \usepackage{titling}

    \title{%
        Termodinâmica Computacional%
    }

    \author{%
        Prof. Dr. Emanuel Rocha Woiski%
    }

    \usepackage[%
        hidelinks,
        %pdftex,
        xetex,
        unicode,
        pdfauthor   = {\theauthor},
        pdftitle    = {\thetitle},
        pdfsubject  = {Termodinâmica Clássica Computacional},
        pdfproducer = {LaTeX with hyperref},
        pdfcreator  = {pdflatex}%
    ]   {hyperref}

    \usepackage{book-thermo-style}

    % List of folders with the figures and plots
    \graphicspath{
        {img/}
        {img/introduction/}
        {img/pureSubstances/}
        {img/theFirstLaw/}
        {img/theSecondLaw/}
        {img/exergyAnalaysis/}
        {img/thermodynamicProperties/}
        {img/thermodynamicPropComputation/}
        {img/homogeneousMixtures/}
        {img/heterogeneousMixtures/}
        {img/annexes/}
    }

    % New commands
    \usepackage{thermo-commands}
    \usepackage{tikz-local-config}
    \usepackage{tensor-operators}

    \makeglossaries

    \loadglsentries{thermo-glossary.tex}
    \loadglsentries{thermo-acronyms.tex}

    \usepackage[%
        type={CC},
        modifier={by-nc-sa},
        version={4.0},
        lang=brazilian
    ]{doclicense}

    \usepackage[capitalise]{cleveref}

\begin{document}

\frontmatter

    \begin{titlepage}
\begin{center}
    \begin{tikzpicture}[%
        remember picture,
        overlay,
        line width=0mm
    ]
        % draw the orange strip
        \fill[
            draw=black,
            shade,
            top color=blue!70!black,
            bottom color=blue!50
        ]   (current page.south west)
            --
            ++(0.3\paperwidth,0)
            -- coordinate (pinMainTitle)
            ++(0,\paperheight)
            --
            (current page.north west)
            --
            cycle;

        \node[%
            %draw,
            anchor=west,
            text width=0.7\paperwidth,
            align=center,
            font=\fontsize{40}{52}\selectfont,
            yshift=0.1\paperheight
        ]   (mainTitle) at (pinMainTitle)
            {
                {\textbf{%
                    \sffamily
                    Termodinâmica\\
                    Computacional%
                    }
                }
            };

        \node[%
            draw,
            fill=white,
            anchor=south east,
            text width=0.5\paperwidth,
            align=center,
            outer sep=0.1\paperwidth,
            inner sep=0.02\paperwidth,
            blur shadow={shadow blur steps=5}
        ]   (subTitle) at (current page.south east)
            {
                {
                    \sffamily
                    \fontsize{30}{20}\selectfont
                    \textbf{Termodinâmica\\[\baselineskip]
                    Aplicada}
                }\\[2\baselineskip]
                {
                    \sffamily
                    \fontsize{14}{11}\selectfont
                    \textit{%
                        \textbf{Emanuel Rocha Woiski}\\[\baselineskip]
                        UNESP - Ilha Solteira
                    }
                }
            };

    \end{tikzpicture}
\end{center}
\end{titlepage}


    \begin{dedication}
        À Tânia, \\
        meu amor, \\
        pela infinita paciência e compreensão
    \end{dedication}

    % Termodinamica Computacional © 2022 by Emanuel Rocha Woiski is licensed
    % under CC BY-NC-SA 4.0. To view a copy of this license, visit
    % http://creativecommons.org/licenses/by-nc-sa/4.0/
    \doclicenseThis

    \doublespacing
    \tableofcontents

\mainmatter

    \import{tex/}{foreword.tex}
    \import{tex/}{introduction.tex}
    \import{tex/}{pureSubstances.tex}
    \import{tex/}{theFirstLaw.tex}
    \import{tex/}{theSecondLaw.tex}
    \import{tex/}{exergyAnalysis.tex}
    \import{tex/}{thermodynamicProperties.tex}
    \import{tex/}{thermodynamicPropComputation.tex}

    \import{tex/}{equationsList.tex}

    \import{tex/}{homogeneousMixtures.tex}
    \import{tex/}{homogeneousReactiveMixtures.tex}
    \import{tex/}{heterogeneousMixtures.tex}
    \import{tex/}{thermodynamicCycles.tex}

    \includepdf[%
        pages=-,
        pagecommand={\pagestyle{fancy}},
        scale=0.9
    ]   {tex/annexes.pdf}

\backmatter

    % To include all entries in the bibliography
    \nocite{*}

    \defbibnote{commentOnBibliography}{%
        Todos os livros de Termodinâmica são muito parecidos entre si, embora a
        abordagem deste ou daquele tópico seja feita com maior ou menor
        aprofundamento. Em nível de graduação em Engenharia Mecânica,
        destacamos o \textcite{VanWylen1994} em sua antiga quarta edição, a
        qual vem com um disquete cujo conteúdo não apenas substitui com
        vantagem todas as tabelas do livro, mas também inclui um aplicativo que
        facilita tremendamente a utilização das coordenadas generalizadas de
        Lee-Kesler, com fator acêntrico e tudo e, até mesmo, rotinas em Fortran
        77, para incluir em programas desenvolvidos pelo leitor. Mas por que a
        quarta edição? Porque esta é a última edição que ainda continha a mais
        completa abordagem de misturas ideais que já encontramos em graduação
        em Engenharia Mecânica. Temos ainda \textcite{Moran1992}, um
        livro-texto que tem sido muito adotado nas escolas. Três outros livros,
        para além da graduação, são: \textcite{Callen1960}, com a sua abordagem
        axiomática geral, \textcite{Modell1983}, que tratam muito bem a questão
        das restrições internas a um sistema e \textcite{Bejan1997}, que, além
        de uma revisão rigorosa da Termodinâmica Clássica, analisa a eficiência
        do ponto de vista exergético e da minimização da geração irreversível
        da entropia, para modelos de sistemas de potência e refrigeração
        típicos.
    }

    \printbibliography[%
        prenote=commentOnBibliography,
        title={Bibliografia Comentada},
        heading=bibintoc
    ]

\end{document}
