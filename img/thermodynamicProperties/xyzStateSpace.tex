% Author Iago Lessa de Oliveira
% Based on code published in 
% https://tex.stackexchange.com/questions/479814/a-diagram-about-partial-derivatives-of-fx-y
\documentclass[tikz,border=3.14mm]{standalone}

    \usepackage{siunitx}
    \usepackage{../../tikz-local-config}

    \usepackage{glossaries-extra}
    \usepackage{glossaries-ext}
    \usepackage{tensor-operators}

        \loadglsentries{../../thermo-glossary.tex}
        \loadglsentries{../../thermo-acronyms.tex}

\begin{document}
    \begin{tikzpicture}[
        bullet/.style={circle,fill,inner sep=1pt},
        declare function={%
            f(\x,\y)=2 - 0.5*pow(\x-1.25,2) - 0.5*pow(\y-1,2);
        }
    ]

    \begin{axis}[%
        view={150}{45},
        colormap/blackwhite,
        axis lines=middle,
        zmax=2.2,
        zmin=0,
        xmin=-0.2,
        xmax=2.4,
        ymin=-0.2,
        ymax=2,
        xlabel=$x$,
        ylabel=$y$,
        zlabel=$z$,
        xtick=\empty,
        ytick=\empty,
        ztick=\empty
    ]

        \def\xmin{0.6};
        \def\xmax{2.0};

        \def\ymin{0.5};
        \def\ymed{1.2};
        \def\ymax{2.0};

        \def\xpoint{1.75};
        \def\ypoint{1.2};

        % Part of the surface
        \addplot3[
            surf,
            %shader=faceted interp,
            %shader=interp,
            domain=\xmin:\xmax,
            domain y=\ymin:\ymed,
            opacity=0.4
        ]   {f(x,y)};

        % Coordinate lines
        \draw[%
            thin,
            dashed
        ]   (\xpoint,0,0)
            node[above left] {$x_0$}
            --
            (\xpoint,\ymed,0)
            node[bullet] (b1) {}
            --
            (0,\ymed,0)
            node[above right] {$y_0$}
            (\xpoint,\ymed,0)
            --
            (\xpoint,\ymed,{f(\xpoint,\ymed)})
            node[bullet] {};

        % Angled line
        \draw[%
            shorten <= -1.25cm,
            shorten >= -1.25cm,
        ]   (\xpoint,\ymed,{f(\xpoint,\ymed)})
            coordinate (thePoint)
            --
            (0.75,\ymed,{f(1.75,1.2)+0.5}) coordinate[pos=0.5] (aux1);

        % Constant y plane
        \draw[%
            %black,
            dashed,
            %upper left=gray!80!black,
            %upper right=gray!60,
            %lower left=gray!60,
            %lower right=gray!80!black
        ]   (\xmax,\ymed,{f(\xmax,\ymed)})
            --
            (\xmax,\ymed,0)
            --
            (\xpoint,\ymed,0)
            (\xmin,\ymed,0)
            --
            (\xmin,\ymed,{f(\xmin,\ymed)});


        \draw[thin, dashed]
            (thePoint)
            --
            ++(-1.75,0,0) coordinate (tipParallelXAxis);

        \pic[%
            draw,
            <->,
            %anchor=west,
            %"$\propto \theta$",
            %angle eccentricity=3.0
        ]   {angle=tipParallelXAxis--thePoint--aux1};

        % Rest of the surface
        \addplot3[%
            surf,
            %shader=interp,
            domain=\xmin:\xmax,
            domain y=\ymed:\ymax,
            opacity=0.4
        ]   {f(x,y)};

        % Cut line on the surface
        \addplot3[
            thick,
            domain=\xmin:\xmax,
            samples y=1
        ]   ({x},\ymed,{f(x,\ymed)});

    \end{axis}

    \draw [%
        latex-
    ]   (aux1)
        --
        ++(-1,1)
        node[%
            above,
            align=center
        ]   {%
                Inclinação na direção $x$\\
                $\dfrac{\partial z}{\partial x}\bigg|_{(x_0,y_0)}$
            };

    \node[%
        anchor=north west
    ]   at (b1) {$(x_0,y_0)$};


    \begin{axis}[%
        xshift=7.5cm,
        view={150}{45},
        colormap/blackwhite,
        axis lines=middle,
        zmax=2.2,
        zmin=0,
        xmin=-0.2,
        xmax=2.4,
        ymin=-0.2,
        ymax=2,
        xlabel=$x$,
        ylabel=$y$,
        zlabel=$z$,
        xtick=\empty,
        ytick=\empty,
        ztick=\empty
    ]

        \def\xmin{0.6};
        \def\xmed{1.2};
        \def\xmax{2.0};

        \def\ymin{0.5};
        \def\ymax{2.0};

        \def\xpoint{1.75};
        \def\ypoint{1.2};

        % Part of the surface
        \addplot3[
            surf,
            %shader=faceted interp,
            %shader=interp,
            domain=\xmin:\xmed,
            domain y=\ymin:\ymax,
            opacity=0.4
        ]   {f(x,y)};

        % Coordinate lines
        \draw[%
            thin,
            dashed
        ]   (\xmed,0,0)
            node[above left] {$x_0$}
            --
            (\xmed,\ypoint,0)
            node[bullet] (b2) {}
            --
            (0,\ypoint,0)
            node[above right] {$y_0$}
            (\xmed,\ypoint,0)
            --
            (\xmed,\ypoint,{f(\xmed,\ypoint)})
            node[bullet] {};

        % Angled line
        \draw[%
            shorten <= -1.50cm,
            shorten >= -1.25cm,
        ]   (\xmed,\ypoint,{f(\xmed,\ypoint)})
            coordinate (thePoint2)
            --
            (\xmed,0.2,{f(\xmed,\ypoint)+0.2}) coordinate[pos=0.5] (aux2);

        % Constant x plane
        \draw[%
            %black,
            dashed,
            %upper left=gray!80!black,
            %upper right=gray!60,
            %lower left=gray!60,
            %lower right=gray!80!black
        ]   (\xmed,\ymax,{f(\xmed,\ymax)})
            --
            (\xmed,\ymax,0)
            --
            (\xmed,\ypoint,0)
            (\xmed,\ymin,0)
            --
            (\xmed,\ymin,{f(\xmed,\ymin)});

        \draw[thin, dashed]
            (thePoint2)
            --
            ++(0,-1.5,0) coordinate (tipParallelYAxis);

        \pic[%
            %draw,
            <->,
            %anchor=west,
            %"$\propto \theta$",
            %angle eccentricity=3.0
        ]   {angle=aux2--thePoint2--tipParallelYAxis};

        % Rest of the surface
        \addplot3[%
            surf,
            %shader=interp,
            domain=\xmed:\xmax,
            domain y=\ymin:\ymax,
            opacity=0.4
        ]   {f(x,y)};

        % Cut line on the surface
        \addplot3[
            thick,
            %domain=\xmin:\xmax,
            domain=\ymin:\ymax,
            samples y=1,
        ]   (\xmed,{x},{f(\xmed,x)});

    \end{axis}

    \draw [%
        latex-
    ]   (aux2)
        --
        ++(0.3,1)
        node[%
            above,
            align=center
        ]   {%
                Inclinação na direção $y$\\
                $\dfrac{\partial z}{\partial y}\bigg|_{(x_0,y_0)}$
            };

    \node[%
        anchor=north east
    ]   at (b2) {$(x_0,y_0)$};

    \end{tikzpicture}
\end{document}
