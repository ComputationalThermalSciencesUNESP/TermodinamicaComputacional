% Author: Iago Lessa de Oliveira (February, 2022)
\documentclass[margin=5pt]{standalone}

    \usepackage{siunitx}
    \usepackage{../../tikz-local-config}

    \usepackage{glossaries-extra}
    \usepackage{glossaries-ext}
    \usepackage{tensor-operators}

        \loadglsentries{../../thermo-glossary.tex}
        \loadglsentries{../../thermo-acronyms.tex}

\begin{document}

    \begin{tikzpicture}[%
        %scale=0.5
    ]

        \begin{scope}[%
            % name local scope (can be used to place node)
            local bounding box=system,
            scale=1.5
        ]

            \draw[%
                thick,
                %dashed,
                shade,
                top color=yellow!50,
                bottom color=yellow!10,
                %fill=yellow!50,
                rotate=55,
                postaction={decorate},
                decoration={%
                    markings,
                    mark=at position 0.1 with{%
                        \coordinate (pinHeat);
                    },
                    mark=at position 0.3 with{%
                        \coordinate (pinWork);
                    },
                    mark=at position 0.9 with{%
                        \coordinate (pinMass);
                    }
                }
            ]   plot[%
                    smooth cycle,
                    tension=0.9
                ]   coordinates {% polar coordinates
                        (0:0.5)
                        (45:1.2)
                        (90:1.5)
                        (135:1.2)
                        (180:1)
                        (225:1)
                        (270:1)
                        (315:1)
                    } coordinate (pinBoundary);
        \end{scope}

        \node[%
            yshift=-0.35cm,
            text width=2.5cm,
            align=center
        ]   (labelSystem) at (system)
            {%
                \textbf{Sistema}\\
                \diff{\gls{mass}}, \diff{\gls{temperature}},
                \diff{\gls{pressure}}, \diff{\gls{volume}},
                \diff{\gls{specificVolume}}
            };

        % boundary label
        \node[%
            xshift=0.55cm,
            yshift=0.35cm,
            anchor=west,
            %text width=2.5cm,
            align=center,
            inner sep=1pt,
        ]   (labelBoundary) at (pinBoundary)
            {%
                \textbf{Fronteira}
            };

        \draw[%
            indication
        ]   (labelBoundary.south west)
            to
            (pinBoundary);

        % Interactions
        \draw[%
            green!50!black,
            thick,
            -latex,
            line width=1.5pt,
            %bend left=10
        ]   (pinMass)++(-0.4cm,-0.4cm)
            to[out=90,in=200]
            node[
                pos=1,
                anchor=south west,
                inner sep=1pt
            ]   {\large\idiff{\gls{mass}}}
            ++(1.0cm,1.0cm);

        \draw[%
            red!70,
            thick,
            -latex,
            line width=1.5pt,
        ]   (pinHeat)++(-0.4cm,0.4cm)
            to[out=0,in=100]
            node[
                pos=0,
                anchor=south east,
                inner sep=1pt
            ]   {\large\idiff{\gls{heatTransfer}}}
            ++(1.0cm,-1.0cm);

        \draw[%
            blue!70,
            thick,
            latex-,
            line width=1.5pt,
        ]   (pinWork)++(-0.4cm,-0.4cm)
            to[out=90,in=190]
            node[
                pos=0,
                anchor=north,
                inner sep=1pt
            ]   {\large\idiff{\gls{workTransfer}}}
            ++(1.0cm,1.0cm);

        \node[%
            xshift=2.5cm,
            yshift=0.3cm,
            text width=3cm,
            align=center,
            text=green!50!black,
        ]   (labelEnv) at (system.south)
            {\textbf{Meio-ambiente}};

        %\begin{scope}[on background layer]
        %    \fill[%
        %        shading=radial,
        %        inner color=green
        %    ]   (current bounding box.south west)
        %        rectangle
        %        (current bounding box.north east);
        %\end{scope}

    \end{tikzpicture}
\end{document}
